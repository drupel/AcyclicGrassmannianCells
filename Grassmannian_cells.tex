\documentclass{amsart}

% Commands for marginal notes
\usepackage[draft]{say}
\newcommand{\sayD}[1]{\say[D]{#1}}
\newcommand{\sayT}[1]{\say[T]{#1}}

\usepackage{amsmath,amssymb,latexsym,color}
\usepackage[bookmarks=true,colorlinks=true, linkcolor=blue, citecolor=cyan]{hyperref}
\usepackage{tikz}
\usetikzlibrary{arrows}
\usetikzlibrary{fit}
\usepackage[margin=1in,marginparwidth=0.8in, marginparsep=0.1in]{geometry}
\usepackage{extarrows}
\usepackage{verbatim}
\input xy
\xyoption{all}

\usepackage[inline]{showlabels}
%\usepackage[marginal]{showlabels}

\newtheorem{theorem}{Theorem}[section]
\newtheorem{claim}[theorem]{Claim}
\newtheorem{conjecture}[theorem]{Conjecture}
\newtheorem{corollary}[theorem]{Corollary}
\newtheorem{definition}[theorem]{Definition}
\newtheorem{lemma}[theorem]{Lemma}
\newtheorem{question}[theorem]{Question}
\newtheorem{proposition}[theorem]{Proposition}
\newtheorem{remark}[theorem]{Remark}
\newtheorem{example}[theorem]{Example}
\newtheorem{thm}{Theorem}
\numberwithin{equation}{section}

\renewcommand{\AA}{\mathbb{A}}
\newcommand{\C}{\mathbb{C}}
\newcommand{\CC}{\mathbb{C}}
\newcommand{\kk}{\Bbbk}
\newcommand{\NN}{\mathbb{N}}
\newcommand{\ZZ}{\mathbb{Z}}

\newcommand{\bfe}{\mathbf{e}}
\newcommand{\bff}{\mathbf{f}}
\newcommand{\bfg}{\mathbf{g}}
\newcommand{\bfh}{\mathbf{h}}
\newcommand{\bfi}{\mathbf{i}}
\newcommand{\bfI}{\mathbf{I}}
\newcommand{\bfJ}{\mathbf{J}}
\newcommand{\bfs}{\mathbf{s}}
\newcommand{\bft}{\mathbf{t}}

\newcommand{\tbfe}{{\tilde\bfe}}
\newcommand{\tbff}{{\tilde\bff}}
\newcommand{\tbfg}{{\tilde\bfg}}
\newcommand{\tbfh}{{\tilde\bfh}}
\newcommand{\tbfs}{{\tilde\bfs}}
\newcommand{\tbft}{{\tilde\bft}}

\newcommand{\cA}{\mathcal{A}}
\newcommand{\cB}{\mathcal{B}}
\newcommand{\cC}{\mathcal{C}}

\newcommand{\cF}{\mathcal{F}}
\newcommand{\cG}{\mathcal{G}}
\newcommand{\cH}{\mathcal{H}}
\newcommand{\cI}{\mathcal{I}}
\newcommand{\cP}{\mathcal{P}}
\newcommand{\cQ}{\mathcal{Q}}
\newcommand{\cT}{\mathcal{T}}

\newcommand{\cU}{\mathcal{U}}
\newcommand{\cV}{\mathcal{V}}
\newcommand{\cW}{\mathcal{W}}

\newcommand{\ui}{{\underline i}}
\newcommand{\uj}{{\underline j}}
\newcommand\udim{{\underline{\dim}\, }}

\newcommand{\into}{\hookrightarrow}
\newcommand{\onto}{\to\!\!\!\!\!\to}
\newcommand{\xonto}[1]{\xrightarrow{#1}\!\!\!\!\!\to}

\newcommand{\Att}{\operatorname{Att}}
\newcommand{\codim}{\operatorname{codim}}
\newcommand{\End}{\operatorname{End}}
\newcommand{\Ext}{\operatorname{Ext}}
\newcommand{\Gr}{\mathrm{Gr}}
\newcommand{\GL}{\mathrm{GL}}
\newcommand{\Hom}{\operatorname{Hom}}
\newcommand{\im}{\mathrm{im}}
\newcommand{\Ind}{\mathrm{Ind}}
\newcommand{\Irr}{\operatorname{Irr}}
\newcommand{\pt}{\mathrm{pt}}
\newcommand{\Rad}{\operatorname{Rad}}
\newcommand{\Rep}{\mathrm{Rep}}
\newcommand{\rep}{\operatorname{rep}}
\newcommand{\rk}{\mathrm{rk}}
\newcommand{\Sc}[2]{\langle #1,#2\rangle}
\newcommand{\ses}[3]{\xymatrix@C15pt{0\ar[r] & #1\ar[r] & #2\ar[r] & #3 \ar[r] & 0}}
\newcommand{\sesm}[4]{\xymatrix{0\ar[r] & #1\ar[r] & #2\ar^{#4}[r] & #3 \ar[r] & 0}}
\newcommand{\sesM}[4]{\xymatrix{0\ar[r] & #1\ar^{#4}[r] & #2\ar[r] & #3 \ar[r] & 0}}
\newcommand{\Ses}[5]{\xymatrix{0\ar[r] & #1\ar^{#4}[r] & #2\ar^{#5}[r] & #3 \ar[r] & 0}}

\newcommand{\spn}{\operatorname{span}}
\newcommand{\supp}{\operatorname{supp}}
\newcommand{\vs}{\vspace{0.2cm}}

\hyphenation{endo-functors}
\setcounter{MaxMatrixCols}{20}


\title{Cell Decompositions for Acyclic Quiver Grassmannians}

\author{Dylan Rupel}
\address[Dylan Rupel]{Michigan State University, Department of Mathematics, 619 Red Cedar Road, C212 Wells Hall, East Lansing, MI 48824}
\email{drupel@nd.edu}
\author{Thorsten Weist}
\address[Thorsten Weist]{Bergische Universit\"at Wuppertal, Gau\ss str.\ 20, 42097 Wuppertal, Germany}
\email{weist@uni-wuppertal.de}

\begin{document}
\begin{abstract}
  We prove that all quiver Grassmannians for exceptional representations of an acyclic quiver admit a cell decomposition.  
  In the process, we introduce a class of regular representations which arise as quotients of consecutive exceptional representations in an exceptional sequence.
  Cell decompositions for quiver Grassmannians of these ``truncated exceptionals'' are also established. 
  The cells admit a natural combinatorial labeling in terms of 2-quivers.
\end{abstract}

\setcounter{tocdepth}{2}

\maketitle
\tableofcontents

%%%%%%%%%%%%%%%%%%%%%%

\section{Introduction}

\subsubsection*{Acknowledgements}
We would like to thank Giovanni Cerulli Irelli, Hans Franzen, Oliver Lorscheid and Markus Reineke for very fruitful discussions related to this project.


\subsection{Notation}
\begin{itemize}
  \item Given two tuples $x=(x_1,\ldots,x_n)$ and $y=(y_1,\ldots,y_m)$, we write $x\smile y$ for their concatenation $(x_1,\ldots,x_n,y_1,\ldots,y_m)$. (better notation?)
  \item Given two tuples of vector spaces $\cU=(U_1,\ldots,U_n)$ and $\cW=(W_1,\ldots,W_n)$, we write $\cU\subset\cW$ if $U_i$ is a subspace of $W_i$ for each $i$ and $\cU\subsetneq\cW$ if at least one of these subspaces is proper.
    In this case, we write $\cW/\cU=(W_1/U_1,\ldots,W_n/U_n)$ and set $\codim_\cW(\cU)=\sum\limits_{i=1}^n \dim(W_i/U_i)$.
  \item Given two tuples of vector spaces $\cU=(U_1,\ldots,U_n)$ and $\cW=(W_1,\ldots,W_n)$, we write $\cU\bullet\cW$ for $\bigoplus_{i=1}^n U_i\otimes W_i$. (better notation?)
\end{itemize}


%%%%%%%%%%%%%%%%%%%%%%%%%%%%%%%

\section{Exceptional sequences}

A representation $X$ is \emph{exceptional} if $X$ is indecomposable and rigid, i.e.\ $\Ext(X,X)=0$.
A sequence of exceptional representations $(X_1,\ldots,X_k)$ is \emph{exceptional} itself if $\Hom(X_i,X_j)=0=\Ext(X_i,X_j)$ for $i<j$.
An immediate consequence of the definition is that an exceptional sequence must consist of non-isomorphic representations. 
An exceptional sequence $(X_1,\ldots,X_k)$ is \emph{$\Ext$-orthogonal} (resp.\ \emph{$\Hom$-orthogonal}) if $\Ext(X_i,X_j)=0$ (resp. $\Hom(X_i,X_j)=0$) for all $i,j$, it is \emph{pairwise-orthogonal} if it is both $\Hom$- and $\Ext$-orthogonal.

An exceptional sequence $(X_1,\ldots,X_n)$ consisting of $n=|Q_0|$ exceptional representations is called \emph{complete}.
A (non-)complete $\Ext$-orthogonal exceptional sequence $(X_1,\ldots,X_k)$ determines a (partial) tilting representation $X_1\oplus\cdots\oplus X_k$. 
As any exceptional representation $X$ can be completed to tilting representation \cite{???}, every exceptional representation can be found in a complete exceptional sequence.
We will see below that this may be done with $X$ in any position of the sequence. 

Without loss of generality we will assume $Q_0=\{1,\ldots,n\}$ is labeled along a \emph{sink adapted sequence}, i.e.\ vertex 1 is a sink in $Q$ and $(2,\ldots,n)$ is sink adapted for the quiver obtained from $Q$ by removing vertex 1 and all arrows incident to this vertex.
Then we have a canonical $\Hom$-orthogonal exceptional sequence $(S_1,\ldots,S_n)$.
A remarkable result of Crawley-Boevey shows that all complete exceptional sequences may be obtained from this standard sequence by an action of the braid group.

Given an exceptional sequence $(X,Y)$, write $\cC(X,Y)$ for the full subcategory of $\rep Q$ containing $X$ and $Y$ which is closed under taking direct sums, kernels, and cokernels.
It is well-known \cite{???} that the category $\cC(X,Y)$ is isomorphic to the category of representations for a generalized Kronecker quiver.
In this subcategory the exceptional sequences are easy to describe, namely they are either the canonical $\Hom$-orthogonal exceptional sequence or a slice of either the preprojetive of preinjective component of the Auslander-Reiten quiver of $\cC(X,Y)$.
The braid group action is then given by either replacing an odd slice by an even slice or by replacing the projective or injective slice with the $\Hom$-orthogonal exceptional sequence.


\subsection{Quiver Grassmannian Fibrations from Exceptional Sequences}

Fix a complete exceptional sequence $(X_1,\ldots,X_n)$.
Write $S_1^{(i)}$ and $S_2^{(i)}$ for the simple projective and simple injective respectively in the category $\cC(X_i,X_{i+1})$.
Given a representation $X\in\cC(X_i,X_{i+1})$, write $\Gr^{(i)}_{e_1,e_2}(X)$ for the projective variety of subrepresentations $E\subset X$ with $E\in\cC(X_i,X_{i+1})$.
Then there is a map
\[\Gr_\bfe(X)\to\bigsqcup_{e_1,e_2} \Gr^{(i)}_{e_1,e_2}(X)\]
given by forgetting the summands of a subrepresentation $E\subset X$ which do not lie in $\cC(X_i,X_{i+1})$.

What can we say about the fibers of this map?


%%%%%%%%%%%%%%%%%%%%%%%%%%%%
\section{General statements}
For two representations $M$ and $N$ write $[M,N]=\dim\Hom(M,N)$ and $[M,N]^1=\dim\Ext(N,M)=1$

\begin{lemma}\label{lem:indecomposable}
  Let $M_1,\ldots,M_n$ be pairwise orthogonal exceptional representations, i.e. $\Hom(M_i,M_j)=\Ext(M_i,M_j)=0$ if $i\neq j$.
  Let $N\in {^\perp} M_i$ for $i=1,\ldots,n$ and let $\Hom(M_i,N)=\langle f_i\rangle=\kk$. 
  Let $f=(f_1,\ldots,f_n):\bigoplus_{i=1}^n M_i\to N$.
  Then $f$ is either surjective or injective. In the first case $\ker(f)$ is exceptional, in the second case $\mathrm{coker}(f)$ is exceptional. 

  Finally, $f$ is injective if and only if $\sum_{i=1}^n\udim M_i\geq \udim N$.
\end{lemma}
\begin{proof}
We proceed by induction on $n$. Let $n=1$. As $\Ext(N,M_1)=0$, the map $f_1$ is either surjective or injective by the Happel-Ringel lemma. In the first case, we obtain a short exact sequence $\ses{K}{M_1}{N}$ such that $\Hom(K,N)=0$ as the induced map $\Hom(M_1,M_1)\to\Hom(M_1,N)$ is an isomorphism. Applying $\Hom(-,M_1)$ shows $\Ext(K,M_1)=0$ as $M_1$ is exceptional. Thus applying $\Ext(K,-)$ shows that $K$ is exceptional. The second case follows in the same way.

Thus assume that the claim is true for $n-1$. In the first case, write $K$ for the kernel of $f':=(f_1,\ldots,f_{n-1})$. Then we have $ \Hom(M_n,N)\cong \Ext(M_n,K)$, $\Hom(M_n,K)=0$ and $K\in{^\perp} M_n$ for $n.$ Write $\tilde K_n$ for the middle term of the unique exact sequence $e_n\in \Ext(M_n,K)$ which is easily seen to be exceptional. Considering the corresponding pullback diagram shows that $\ker(f)=\tilde K_n$.

If $f'$ is injective, write $C=\mathrm{coker}(f')$. Then we have $\Hom(M_n,C)\cong\Hom(M_n,N)$, $\Ext(M_n,C)=0$ and $C\in{^\perp}M_n$. Thus the claim follows in the same way as for $n=1$. 

The last statement is clear.
\end{proof}

\begin{lemma}\label{lem:indecomposable2}\sayT{does this follow from some general stuff?}
Let $(N,M_1,\ldots,M_n)$ be an exceptional sequence as in Lemma \ref{lem:indecomposable} Assume that each $M_i$ and $N$ are not injective. Moreover, assume that $f$ is injective and that $\mathrm{coker}(f)$ is the injective indecomposable representation $I_q$. Then $\tau^{-1}f:\bigoplus_{i=1}^n\tau^{-1}M_i\to \tau^{-1}N$ is surjective with $\mathrm{ker}(\tau^{-1}f)=P_q$.  
\end{lemma}
\begin{proof}\sayT{notation needs to be improved and adjusted}
Applying reflection functors, as all representation except the cokernel are not injective, we may assume that $I_q=S_q$, i.e. $q$ is a source. Then we have
$$\udim N-\sum_{i=1}^n\udim M_i=q.$$
It follows that
$$\sum_{i=1}^n\udim \sigma_qM_i-\udim \sigma_q N=q$$
and thus $\sigma_qf$ is surjective with $\mathrm{coker}(f)=P_q$. Note that $q$ is a sink of $\mu_qQ$. Furthermore, $\sigma_q$ and $\tau^{-1}$ preserve all properties of the exceptional sequence  necessary for Lemma \ref{lem:indecomposable}. Now write $\tau^{-1}=\sigma_{q_1}\ldots\sigma_{q_m}\sigma_q$ for some source adapting sequence $(q,q_m,\ldots,q_1)$. As $\sigma_{q_1}\ldots\sigma_{q_m}P_q$ is the projective representation corresponding to $q$ in the quiver $\mu_{q_1}\ldots\mu_{q_m}\mu_q Q=Q$, we obtain a short exact sequence $$\ses{P_q}{\bigoplus_{i=1}^n\tau^{-1}M_i}{\tau^{-1}N}$$ which shows the claim.

When applying reflection functors again, the general statement is obtained.


\end{proof}


%%%%%%%%%%%%%%%%%%%%%%%%%%%%%%%%%%%%%%%%%
\section{Results on Quiver Grassmannians}

\noindent We recall several results concerning Grassmannians which are needed to construct cell decompositions.
\begin{definition}
We say that a representation $M$ has \emph{property (C)} if every non-empty quiver Grassmannian $\Gr_\bfe(M)$ admits a cell decomposition into affine spaces.


\end{definition}


\subsection{Torus action}
  
\subsection{Caldero-Chapoton map}

Let $\xi:\ses{M}{B}{N}$ be a short exact sequence of representations.
For a fixed dimension vector $\bfe$, this induces the so-called ``Caldero-Chapoton map'' between quiver Grassmannians
\begin{align*}
  \Psi:\Gr_\bfe(B)&\to\bigsqcup_{\bff+\bfg=\bfe} \Gr_\bff(M)\times \Gr_\bfg(N)\\
  E&\mapsto \big(E\cap M,(E+M)/M\big).
\end{align*}
Following \cite[Section 3]{cc}, any non-empty fiber of $\Psi$ satisfies $\Psi^{-1}(U,W)\cong\Hom(W,M/U)$, in particular each non-empty fiber is an affine space.%\mathbb{A}^{\dim\Hom(W,M/U)}$.

For $\cG_{\bff,\bfg}:=\Psi^{-1}\big(\Gr_\bff(M)\times \Gr_\bfg(N)\big)$, we have the stratification
\begin{equation}
  \label{eq:grassmannian decomposition}
  \Gr_\bfe(B)=\bigsqcup_{\bff+\bfg=\bfe} \cG_{\bff,\bfg}.
\end{equation}
\sayD{ToDo: Identify closure order.}
Then $\Psi$ restricts to a map
\[\Psi_{\bff,\bfg}:\cG_{\bff,\bfg}\to \Gr_\bff(M)\times \Gr_\bfg(N).\]
The following results are proven in \cite[Section 3]{cefr}.
\begin{theorem}\label{thm:ccbundle}
  \label{vb}
  If $\im\big(\Psi_{\bff,\bfg}\big)$ is locally closed and the fiber dimension of $\Psi_{\bff,\bfg}$ is constant over $\im\big(\Psi_{\bff,\bfg}\big)$, then $\Psi_{\bff,\bfg}:\cG_{\bff,\bfg}\to\im\big(\Psi_{\bff,\bfg}\big)$ is an affine bundle.
  \sayD{Random question: Can the identification maps between adjacent fibers $\Hom(W,M/U)$ and $\Hom(W',M/U')$ be realized in terms of maps relating $W,W'$ or $M/U,M/U'$?}
  In particular, the existence of a cell decomposition of $\im\big(\Psi_{\bff,\bfg}\big)$ implies a cell decomposition of $\cG_{\bff,\bfg}$ in this case.
\end{theorem}

For two representations $M$ and $N$ with $[N,M]=1$, we define
$$M_N=\max\{U\subset M\mid [N,M/U]^1=1\},\quad N^M=\min\{W\subset N\mid [W,M]^1=1\}.$$
By \cite[Lemma 27]{cefr}, these representations are well-defined.
\begin{lemma}
  \label{lem:ccbundle}
  Let $M$ and $N$ be representations such that $[N,M]=1$ and let $f:M\to \tau N$ be the induced morphism.
  Then we have $M_N=\ker(f)$ and $N^M=\im(\tau^{-1}f:\tau^{-1}M\to N)$. 
\end{lemma}

Note that we have natural embeddings $\Gr_{\bff}(M_N)\subset\Gr_{\bff}(M)$ and $\Gr_{\bfg-\udim N^M}(N/N^M)\subset \Gr_{\bfg}(N)$.
Using the homological diagram
\[
  \xymatrix{ \Ext(N,M) \ar[r]\ar[d] & \Ext(N,M/U) \ar[r]\ar[d] & 0\\
    \Ext(W,M) \ar[r]\ar[d] & \Ext(W,M/U) \ar[r]\ar[d] & 0\\
    0 & 0 & }
\]
we see that $\Gr_\bff(M_N)\times\Gr_{\bfg-\udim N^M}(N/N^M)$ is precisely the locus of points $(U,W)$ inside $\Gr_\bff(M)\times\Gr_\bfg(N)$ over which the dimension of $\Hom(W,M/U)$ jumps by 1 from its generic value.
\sayD{These points are not actually in the image of $\Psi$, but it elicits the right geometric picture.  What geometric understanding can be gained by completing the quiver Grassmannian using such an auxiliary affine bundle over $\Gr_\bff(M_N)\times\Gr_{\bfg-\udim N^M}(N/N^M)$?  Does this say anything about closure order of cells?}

\begin{theorem}
  \label{thm:ccbundle2}
  If $[N,M]^1=1$ and $\xi\in\Ext(N,M)$ does not split, we have
  $$\im\big(\Psi_{\bff,\bfg}\big)=\{(U,W)\mid \Ext(W,M/U)=0\}=\Gr_\bff(M)\times \Gr_\bfg(N)\backslash(\Gr_\bff(M_N)\times\Gr_{\bfg-\udim N^M}(N/N^M)).$$
  Moreover, $\Psi_{\bff,\bfg}:\cG_{\bff,\bfg}\to\im\big(\Psi_{\bff,\bfg}\big)$ is an affine bundle of rank $\Sc{\bfg}{\udim M-\bff}$.
\end{theorem}

\begin{definition}
  Let $B$ be a representation with property (C) and $M\subset B$.
  We call the pair $(M,B)$ \emph{compatible} if $M$ and $B/M$ have property (C), $\im(\Psi_{\bff,\bfg})\subset \Gr_\bff(M)\times\Gr_\bfg(B/M)$ is a union of cells and the fiber dimension is constant over $\im(\Psi_{\bff,\bfg})$ for each $\bff,\bfg$.
\end{definition}
We will frequently make use of the following version of these results.
\begin{corollary}
  \label{cor:ccbundle}
  \sayT{this should be the version we need}
  Let $M$ and $N$ be representations with property (C) such that $[N,M]=1$, i.e. there exists a unique non-split exact sequence $\ses{M}{B}{N}$.
  If $(M_N,M)$ and $(N^M,N)$ are compatible pairs, then $B$ has property (C).
  Furthermore, $(M,B)$ is a compatible pair.
\end{corollary}
\begin{proof}
  For $\bfe=\bff+\bfg$, consider the maps $$\Psi_{\bff,\bfg}:\cG_{\bff,\bfg}\to\Gr_\bff(M)\times\Gr_\bfg(N),\quad \Psi_{\bff',\bff''}:\cH_{\bff',\bff''}\to\Gr_{\bff'}(M_N)\times\Gr_{\bff''}(M/M_N),$$$$\Psi_{\bfg',\bfg''}:\cI_{\bfg',\bfg''}\to\Gr_{\bfg'}(N^M)\times\Gr_{\bfg''}(N/N^M)$$
  and let
  $$\Gr_\bfe(B)=\bigsqcup_{\bff+\bfg=\bfe} \cG_{\bff,\bfg},\quad \Gr_\bff(M)=\bigsqcup_{\bff'+\bff''=\bff} \cH_{\bff',\bff''},\quad \Gr_\bfe(N)=\bigsqcup_{\bfg'+\bfg''=\bfg} \cI_{\bfg',\bfg''}$$
  be the corresponding decompositions.

  By Theorem \ref{thm:ccbundle2}, we have 
  $$\im(\Psi_{\bff,\bfg})=\left(\bigsqcup_{\substack{\bff'+\bff''=\bff\\\bff'\neq \bff}} \cH_{\bff',\bff''}\right)\times\left(\bigsqcup_{\substack{\bfg'+\bfg''=\bfg\\\bfg'\neq\udim N^M}} \cI_{\bfg',\bfg''}\right).$$
  As $(M_N,M)$ and $(N^M,N)$ are compatible pairs, each $\cH_{\bff',\bff''}$ and each $\cI_{\bfg',\bfg''}$ has a cell decomposition into affine spaces.

  Again by Theorem \ref{thm:ccbundle2}, we see that each $\cG_{\bff,\bfg}$ has a cell decomposition into affine spaces. Furthermore, $B$ has property (C) and by construction $(M,B)$ is a compatible pair.
\end{proof}


%%%%%%%%%%%%%%%%%%%%%%%%%%%%%%%%%%
\section{Truncated Preprojectives}

Let $Q=(Q_0,Q_1,s,t)$ be a connected acyclic quiver, to avoid special cases we assume $Q$ is not of finite representation type.
Given a sink or source $k\in Q_0$, write $\mu_k Q$ for the quiver obtained from $Q$ by reversing all arrows incident to vertex $k$.
Without loss of generality, we will assume the vertex set $Q_0=\{1,\ldots,n\}$ is labeled along a \emph{sink adapted sequence}, i.e.~vertex 1 is a sink in $Q$, vertex $2$ is a sink in $\mu_1 Q$, vertex $3$ is a sink in $\mu_2\mu_1 Q$, etc.
Observe that this implies vertex $n$ is a source in $Q$, vertex $n-1$ is a source in $\mu_n Q$, vertex~$n-2$ is a source in $\mu_{n-1}\mu_n Q$, etc.

For $i,j\in Q_0$, we introduce the following notation:
\begin{itemize}
  \item $Q_1(i,-)$ is the set of arrows $\alpha\in Q_1$ with source $s(\alpha)=i$ and $Q_0(i,-)$ is the set of targets $t(\alpha)$ for all such arrows;
  \item $Q_1(-,j)$ is the set of arrows $\alpha\in Q_1$ with target $t(\alpha)=j$ and $Q_0(-,j)$ is the set of sources $s(\alpha)$ for all such arrows;
  \item $Q_1(i,j)=Q_1(i,-)\cap Q_1(-,j)$ is the set of arrows $\alpha\in Q_1$ with $s(\alpha)=i$ and $t(\alpha)=j$;
  \item write $b_{ij}=|Q_1(i,j)|$ for the number of arrows from vertex $i$ to vertex $j$.
\end{itemize}

A \emph{leaf} of $Q$ is a vertex $\ell\in Q_0$ with $|Q_1(\ell,-)\cup Q_1(-,\ell)|=1$.
For $i,\ell\in Q_0$, an arrow $\alpha\in Q_1(i,\ell)\cup Q_1(\ell,i)$ is called a \emph{limb arrow} and $\ell$ is a \emph{limb vertex} with respect to vertex $i$ if removing $\alpha$ from $Q$ creates a disconnected quiver in which the component containing vertex $\ell$ is an orientation of a Dynkin diagram of type $A$ with $\ell$ a leaf in this component.

Note that each limb vertex $\ell\in Q_0$ has a uniquely associated leaf vertex $\ell'$ of~$Q$ lying at the opposite end of its type $A$ component, if $\ell$ itself is a leaf of $Q$ then $\ell'=\ell$.
Once a limb vertex $\ell$ is specified, the type $A$ component associated to $\ell$ is uniquely determined (we assume $Q$ itself is not of type $A$).
In the type $A$ subquiver $Q(\ell)\subset Q$ obtained by adjoining the limb arrow $\alpha$ to this component, each arrow is directed either toward vertex $i$ or toward vertex $\ell'$.
We write $h(\ell)$ for the total number of arrows in the quiver $Q(\ell)$ directed toward vertex $i$.

For a sink or source vertex $k\in Q_0$, write $\Sigma_k:\rep Q\to\rep \mu_k Q$ for the BGP-reflection functor \cite{BGP??} (see also \cite{APR??} for an equivalent formulation known as APR-tilting).
Note that we use the same symbol $\Sigma$ to denote the reflection functor associated to a sink or to a source and also for the reflection functors associated to any reflection of $Q$.
In particular, the functor $\Sigma_k^2$ is naturally isomorphic to the identity functor when restricted to the full subcategory $\rep^{\langle k\rangle} Q\subset\rep Q$ consisting of representations with no summand isomorphic to the vertex simple $S_k$.
Moreover, the reflection functor $\Sigma_k:\rep^{\langle k\rangle} Q\to\rep^{\langle k\rangle} \mu_k Q$ is an exact equivalence of categories \cite{DR76}.
Recall that the \emph{Auslander-Reiten translation} $\tau:\rep Q\to\rep Q$ is naturally isomorphic to the product of reflection functors $\Sigma_n\cdots\Sigma_1$ \cite{BB??}.

For $i\in Q_0$, write $P_i$ for the projective cover of the simple $S_i$ and write $I_i$ for its injective hull.
These representations are constructed inductively via $P_i=\Sigma_1\cdots\Sigma_{i-1}S_i$ (resp.~via $I_i=\Sigma_n\cdots\Sigma_{i+1}S_i$), where $S_i$ here denotes the vertex simple of the quiver $\mu_{i-1}\cdots\mu_1 Q$ (resp.~of the quiver $\mu_{i+1}\cdots\mu_n Q$).
%For $i,j\in Q_0$, write $\Irr(P_i,P_j)\subset\Hom(P_i,P_j)$ for the subspace of morphisms which have no nontrivial factorization through a projective $P_k$ with $k\ne i,j$. 
%\begin{lemma}
%  For $i,j\in Q_0$, we have $\dim\Irr(P_j,P_i)=b_{ij}$.
%\end{lemma}
The \emph{preprojective} (resp.~\emph{preinjective}) representations $P_{i,m}$ (resp.~$I_{i,m}$) for $i\in Q_0$ and $m\in\ZZ_{\ge0}$ are defined inductively by
\[P_{i,0}=P_i,\qquad P_{i,m+1}=\tau^{-1} P_{i,m},\qquad I_{i,0}=I_i,\qquad I_{i,m+1}=\tau I_{i,m}.\]
More generally, we say a representation $M\in\rep Q$ is \emph{preprojective} (resp.~\emph{preinjective}) if $\tau^r M=0$ (resp. if~$\tau^{-r} M=0$) for some $r>0$.

\begin{lemma}
  \label{le:preprojective ext groups}
  For $i,j,k\in Q_0$ with $|Q_1(i,j)|\ne0$ and $m\ge0$, we have 
  \[\Ext(P_{k,m},P_{i,m})=0 \qquad \text{and} \qquad \Ext(P_{j,m+1},P_{i,m})=0.\]
\end{lemma}
%\begin{proof}
%  The first claim for $m=0$ is immediate since the representations $P_{i,0}$ are projective, for $m>0$ this follows by applying Auslander-Reiten translations.
%  The second claim may be reduced to the case $m=0$ and $j$ a sink in $Q$ using Auslander-Reiten translations and reflection functors.
%\end{proof}
For two dimension vectors $\alpha$ and $\beta$ we denote by $\hom(\alpha,\beta)$ the minimal and thus general value if $\dim\Hom:R_\alpha(Q)\times R_\beta(Q)\to \NN$, see \cite{sch} for more details.
\begin{lemma}
  \label{lem: non-preinjective}
  If $M,N\in\rep(Q)$ are indecomposable such that $\udim M$ is an imaginary root and such that $\Hom(N,M)\neq 0$, then $N$ is not preinjective.
  In particular, if $\hom(\alpha,\beta)\neq 0$ and $\beta$ is an imaginary Schur root, then $\alpha$ is not the dimension vector of a preinjective representation.
\end{lemma}
\begin{proof}
  As $\udim M$ is imaginary, $M$ is a regular representation.
  Indeed all preinjectives and preprojective indecomposable representations are exceptional.
  But if $N$ were preinjective, we had $\Hom(N,M)\neq 0$ only for preinjectives $M$.

  Finally, if $\beta$ is a Schur root, then a general representation is indecomposable which gives the second part.
\end{proof}
For a dimension vector $\alpha\in\NN^{Q_0}$, we write $\alpha=\sum_{q\in Q_0}\alpha_q q$.
\begin{proposition}
  \label{pro:simpleregular}
  For $i\in Q_0$, the vertex simple $S_i$ is preinjective if and only if one of the following holds: 
  \begin{enumerate}
    \item $|Q_1(-,i)|=0$;
    \item $|Q_1(-,i)|=1$ and $\ell\in Q_0(-,i)$ is a limb vertex with respect to vertex $i$.
  \end{enumerate}
  In particular, the vertex simple $S_i$ is not preinjective if $|Q_1(-,i)|\geq 2$.			
\end{proposition}
\begin{proof}
  If $Q_1(-,i)=0$, then vertex $i$ is a source of $Q$ and so $S_i$ is injective.

  Assume $|Q_1(-,i)|=1$, say $\ell\in Q_0(-,i)$, and that $\ell$ is a limb vertex with respect to $i$ with associated leaf $\ell'$ in $Q(\ell)$.
  We work by induction on $h(\ell)\ge1$ to show that $S_i=\tau^{h(\ell)}I_{\ell',0}$ is preinjective.
  Indeed, when $h(\ell)=1$, the quiver $Q(\ell)$ has the form
  \[\xymatrix{\bullet^{\ell'} & \hdots \ar[l] & \bullet^{\ell''} \ar[l] & \bullet^{\ell} \ar[l] \ar[r] & \bullet^{i} }\]
  and an easy calculation shows that $S_i=\tau I_{\ell',0}$ in this case.
  Now for $h(\ell)>1$, there are two possibilities.
  If $\ell$ is not a source, then by induction $S_\ell=\tau^{h(\ell)-1}I_{\ell',0}$ is preinjective, but $S_i=\tau S_\ell$ in this case and so $S_i=\tau^{h(\ell)}I_{\ell',0}$ as desired.
  If $\ell$ is a source, then there exists a vertex $i'$ in $Q(\ell)$ so that $S_i=\tau^2 S_{i'}$ with $S_{i'}=\tau^{h(\ell)-2} I_{\ell',0}$ by induction, this again gives $S_i=\tau^{h(\ell)} I_{\ell',0}$ in this case.

  To finish, we aim to show in the remaining cases, namely when $|Q_1(-,i)|\ge2$ or when $|Q_1(-,i)|=1$ but $\ell\in Q_0(-,i)$ is not a limb vertex with respect to vertex $i$, that $S_i$ is not preinjective by applying Lemma~\ref{lem: non-preinjective}. 
  Let $Q'$ be the connected component containing $i$ after removing all arrows in $Q_1(i,-)$ from the quiver $Q$.
  Note that vertex $i$ is a sink in $Q'$.

  If $Q'$ is wild or of extended Dynkin type, let $M$ be any indecomposable representation of $Q'$ with $M_i\neq 0$ such that $\beta=\udim M$ is an imaginary root of $Q'$, and hence an imaginary root of $Q$. 
  We can naturally identify $M$ with a representation of $Q$ by setting $M_j=0$ for all $j\in Q_0\setminus Q'_0$ (in particular, $M_\alpha=0$ for all $\alpha\in Q_1(i,-)$).
  This yields $\Hom_Q(S_i,M)\neq 0$ and thus the claim follows from Lemma \ref{lem: non-preinjective} in this case. 

  Thus in the following we assume that $Q'$ is of Dynkin type $A$, $D$, or $E$.
  In particular, this implies that $|Q_0(-,i)|=|Q_1(-,i)|$, i.e.~there are no parallel arrows with target $i$.
  Moreover, since $Q$ is not of Dynkin type itself, there exists a vertex $j\in Q_0(i,-)$.  
	
  First consider the case $|Q_0(-,i)|\geq 2$, i.e.~there exist distinct vertices $q,q'\in Q_0(-,i)$.
  Note that $q,q'\neq j$ as $Q$ is acyclic. 
  If $|Q_1(i,j)|\geq 2$, we have for the imaginary root $\beta=q+q'+3i+j$ that $\hom(i,\beta)\geq 1$.
  It follows that $S_i$ is not preinjective by Lemma~\ref{lem: non-preinjective}.
  Thus we may additionally assume that $|Q_0(i,-)|=|Q_1(i,-)|$, i.e.~there are no parallel arrows with source $i$.

  If $|Q_0(i,-)|\ge2$, i.e.~there exists $j'\in Q_0(i,-)$ distinct from vertex $j$, then there would exists a (not necessarily full) subquiver $Q''$ of $Q$ with vertices $\{i,j,j',q,q'\}$ of extended Dynkin type $\tilde D_4$ with middle vertex $i$ and two each of incoming/outgoing arrows.
  Then $S_i$ is regular as a representation of $Q''$.
  Indeed, in this case there exists an indecomposable imaginary root representation $X$ of $Q''$ in the same tube as $S_i$ which is of dimension $\delta=q+q'+2i+j+j'$ and such that $\Hom(S_i,X)\neq 0$.
  As we have $\dim\Ext_Q(X,X)\geq\dim\Ext_{Q''}(X,X)\neq 0$, $X$ is also an imaginary root representation of $Q$ and we can again apply Lemma~\ref{lem: non-preinjective} to conclude that $S_i$ is not preinjective.

  Thus we may assume that $|Q_0(i,-)|=|Q_1(i,-)|=1$ in the case $|Q_0(-,i)|\ge2$.
  If there were a vertex $q''\notin \{j,q,q'\}$ such that $|Q_1(q'',i)|\geq 1$, then we may again consider the (not necessarily full) subquiver $Q''$ of $Q$ with vertices $\{i,j,q,q',q''\}$ which will again be of extended Dynkin type $\tilde D_4$ with middle vertex $i$, only now there are three incoming arrows and a single outgoing arrow.
  Then for an indecomposable imaginary root representation $X$ of $Q''$ of dimension $\delta=q+q'+q''+2i+j$ which lies in a homogeneous tube, we again have $\Hom(S_i,X)\neq 0$.
  As in the previous case, we may again apply Lemma~\ref{lem: non-preinjective} to conclude that $S_i$ is not preinjective.

  When $|Q_0(-,i)|\ge2$, it remains to consider the case where vertex $i$ has precisely the three neighbors $q$, $q'$, and $j$.
  Note that, as $Q'$ is of Dynkin type, $q$ is not connected to $q'$.
  If $j\in Q'_0$, then vertex $j$ is connected to $q$ (or analogously to $q'$) by a sequence $j=q_1,q_2,\ldots,q_r=q$ with each $q_t\notin\{q',i\}$ and the root $\beta=i+\sum_{t=1}^r q_t$ is imaginary.
  Denoting the arrow from $i$ to $j$ by $\alpha$, there exists a Schurian representation of dimension $\beta$ with $X_\alpha=0$ which implies $\Hom(S_i,X_\alpha)\neq 0$. 
  Thus Lemma~\ref{lem: non-preinjective} again yields the claim in this case.

  Thus we may additionally assume that $j\notin Q'_0$ and so there exists precisely one path from each of the vertices $q$ or $q'$ to $j$, namely the path passing through vertex $i$.
  If there exists $i'\in Q_0$ different from vertex~$i$ such that $|Q_0(i',-)\cup Q_0(-,i')|\geq 3$, then $Q$ will contain a subquiver $Q''$ of extended Dynkin type $D$ in which precisely two vertices out of $\{q,q',j\}$ are leaves.
  Arguing as in the $\tilde D_4$ cases above, we may again find an indecomposable imaginary root representation $X$ of $Q''$ with $\Hom(S_i,X)\neq 0$ and apply Lemma~\ref{lem: non-preinjective} to conclude that $S_i$ is not preinjective.

  Thus it remains to consider the case when $Q$ is a flag quiver with central vertex $i$.
  If $Q$ is not a tree, i.e. there exist vertices $s,s'$ with $|Q_1(s,s')|\geq 2$, the root 
  $$\beta=2i+\sum_{q\in Q_0\backslash\{i\}}q$$
  is imaginary with $\hom(i,\beta)\neq 0$ and Lemma \ref{lem: non-preinjective} applies.
  
  Thus assume that $Q$ is a tree.
  As it is not of Dynkin type, it contains an extended Dynkin quiver of type $E$ as a full subquiver.
  But then it is straightforward that $\hom(i,\delta)\neq 0$ for the respective imaginary Schur roots $\delta$, see \cite[Section 4]{CB} for a list.
  Thus again Lemma \ref{lem: non-preinjective} applies and we see that $S_i$ is not preinjective.
	
  It remains to consider the case when $|Q_0(-,i)|=1$ and $\ell\in Q_0(-,i)$ is not a limb with respect to vertex~$i$.
  From above, we can assume that $Q'$ is of Dynkin type $D$ or $E$ (with leaf $i$).
  Applying reflection functors we can assume that all arrows in $Q'$ are oriented towards $\ell$, i.e. $\ell$ is a sink of $Q'$.
  Let $\gamma:\ell\to i$ denote the unique arrow with sink $i$. 
By assumption, we either have $|Q_0(-,\ell)|=2$ or there is a vertex $\ell'\in Q'_0$ such that $|Q_0(-,\ell')|=2$ and $|Q_0(\ell',-)|=1$.
Moreover, there is a unique path from $\ell'$ to $i$ where we denote the vertices in between by $q_1,\ldots,q_r$.
Now let $Q(\ell',i)$ be the quiver obtained from $Q$ when identifying $\ell'$ and $i$ and removing all arrows and vertices along the path from $\ell'$ to $i$.
Denote the vertex obtained via this identification by $(\ell',i)$.
It is straightforward to check that roots $\beta$ of $Q$ with $\beta_i=\beta_{q_t}=\beta_\ell'$ for $t=1,\ldots,r$ are imaginary if and only if the corresponding root of $Q(\ell',i)$ with $\beta_{(\ell',i)}=\beta_i$ is imaginary.
Thus if $Q(\ell',i)$ is not of Dynkin type, we can make use of the observations in the case $|Q(-,i)|=2$.
Note that $|Q(\ell',i)_1(-,(\ell',i))|\geq 2$.

Thus we may finally assume that $Q(\ell',i)$ is of Dynkin type and $Q$ is not.
This is only possible if $Q$ is of type $E$.
But then for the imaginary Schur root $\delta$ we have $\delta_i<\delta_\ell$.
But this yields $\hom(i,\delta)\neq 0$ and thus the claim.
\end{proof}

\begin{corollary}
  \label{cor:simpleregular}
  For $i\in Q_0$, the vertex simple $S_i$ is preprojective if and only if one of the following holds: 
  \begin{enumerate}
    \item $|Q_1(i,-)|=0$;
    \item $|Q_1(i,-)|=1$ and $\ell\in Q_0(i,-)$ is a limb vertex with respect to vertex $i$. %such that the associated leaf is not equal to $i$.
  \end{enumerate}
  In particular, the vertex simple $S_i$ is not preprojective if $|Q_1(i,-)|\geq 2$.
\end{corollary}




\begin{lemma}
  \label{lem:regular support}
  If $\alpha\in\NN Q_0$ is a regular root of the subquiver $\supp(\alpha)\subset Q$, then it is a regular root of $Q$.
\end{lemma}
\begin{proof}Idea: For every regular representation $M$ of $\supp(\alpha)$ there is a regular representation $N$ of $\supp(\alpha)$ with $\Hom(M,N)\neq 0$. But $N$ can be naturally extended to a regular representation $\tilde N$ of $Q$ with $\Hom_Q(M,\tilde N)\neq 0$ which shows that $M$ is regular.
\end{proof}

\begin{lemma}
  \label{le:nonlimb extensions}
  Suppose $\ell\in Q_0(n,-)$ is not a limb vertex with respect to the source vertex $n$.
  Then any nontrivial extension $E\in\Ext(S_n,P_{\ell,0})$ is indecomposable and not preinjective.
\end{lemma}
\begin{proof}
  Recall that $\dim (P_{\ell,0})_n$ is the number of paths from $\ell$ to $n$ in $Q$.
  As there is an arrow from $n$ to $\ell$ and as $Q$ is acyclic, we have $\dim (P_{\ell,0})_n=0$.
  By assumption, we have a non-split exact sequence
  \[\ses{P_{\ell,0}}{E}{S_n}.\]
  Since $\Hom(P_{\ell,0},S_n)=0$, applying $\Hom(P_{\ell,0},-)$ to this sequence shows $\Hom(P_{\ell,0},E)\cong\Hom(P_{\ell,0},P_{\ell,0})$ is one-dimensional.
  Since $S_n$ is injective, we have $\Hom(S_n,E)=0$ and thus applying $\Hom(-,E)$ to the sequence above shows that $\End(E)$ is also one-dimensional, in particular $E$ must be indecomposable.

  Using that $P_{\ell,0}$ and $S_n$ are exceptional and $\Sc{P_{\ell,0}}{S_n}=0$, we obtain
  $$\Sc{\udim E}{\udim E}=\Sc{\udim P_{\ell,0}+S_n}{\udim P_{\ell,0}+S_n}=2-\Sc{S_n}{P_{\ell,0}}=2-\dim\Ext(S_n,P_{\ell,0}).$$
  If $\dim\Ext(S_n,P_{\ell,0})\geq 2$, it follows that $\udim E$ is an imaginary Schur root and thus $E$ is not preinjective.

  Thus assume that $\dim\Ext(S_n,P_{\ell,0})=1$.
  This implies $\ell$ is the only vertex in the support of $P_{\ell,0}$ which is a neighbor of $n$ and $|Q_1(n,\ell)|=1$, otherwise we could again conclude that $\dim\Ext(S_n,P_{\ell,0})\geq 2$.
  Let $Q'\subset Q$ denote the support quiver of $E$ and label $Q'_0=\{i_1,\ldots,i_r\}$ along a sink adapted sequence for $Q'$.
  Then, since $\ell$ is a source in the support of $P_{\ell,0}$, we must have $i_r=n$ and $i_{r-1}=\ell$.
  It follows that 
  $$P_{\ell,0}=\Sigma_{i_1}\cdots\Sigma_{i_{r-2}} S_\ell \qquad \text{and} \qquad E=\Sigma_{i_1}\cdots\Sigma_{i_{r-2}}\Sigma_n S_\ell.$$

% in particular $E$ is projective as a representation of $Q'$.
 % It follows that $E$ is not injective as a representation of $Q$.
 % Indeed, this would imply that $E$ is injective as a representation of $Q'$, but as $\ell$ is not a limb vertex, $Q'$ is not of type A and thus $Q'$ does not admit a projective-injective representation.
  %\sayD{I am not sure why $E$ having a one-dimensional space at vertex $n$ implies it is not strictly preinjective.}
	%It follows that $\Sigma_n\Sigma_{i_{r-1}}\cdots\Sigma_{i_1} E\cong S_\ell$ as representations of $\mu_n\mu_{i_{r-1}}\cdots\mu_{i_1} Q$.
  As $\ell$ is not a limb vertex with respect to $n$ and as $\ell$ is a sink of $Q'':=\mu_{i_{r-1}}\ldots\mu_{i_1}Q$,   Proposition \ref{pro:simpleregular} shows that $S_\ell$ and thus $E=\Sigma_{i_1}\cdots\Sigma_{i_{r-2}}\Sigma_n S_\ell$ is not  preinjective. Indeed, we either have $|Q_1(-,\ell)|\geq 2$ or the unique vertex $\ell'\in Q_0(-,\ell)$ is not a limb vertex as $\ell$ is not a limb vertex.
	
	
	%there exist at least two arrows $\alpha:q\to \ell$ and $\alpha':q'\to \ell$ with $q,q'\neq n$ in $Q''$, i.e. we have $Q_1(-,\ell)\geq 2$.
  \sayD{I don't see why this claim should be true.  For example we could have $|Q_1(\ell,-)\cup Q_1(-,\ell)|=2$ with one of the arrows connected to $n$.}\sayT{Actually, the claim follows from the refined version of Prop. 4.3, changed this}

	
  %As $n$ was a source of $Q$, it follows that $n$ is a sink of $\mu_n\mu_{\ell-1}\cdots\mu_{1} Q$.
  %In turn $\ell$ is neither sink nor source and thus $S_{\ell}$ is not preinjective because we assumed that $Q$ is not of Dynkin type.
  %\sayT{(why) is this true? Make this precise}
  %\sayD{Here is an argument, is there a shorter one? The dimension vector of a preinjective representation is a positive root of the form $c^rs_n\cdots s_{i+1}\alpha_i$ where $c=s_n\cdots s_1$ is a Coxeter element.  If $Q$ is not of finite representation type, then a Coxeter element is not of finite order and thus $\alpha_\ell$ cannot be the dimension vector of a preinjective representation.}
  %Thus $E$ is also not preinjective.
\end{proof}

\begin{lemma}
  Suppose $\ell_1,\ell_2\in Q_0(n,-)$ are limb vertices with respect to the source vertex $n$.
  Then any indecomposable extension $E\in\Ext(S_n,P_{\ell_1,0}\oplus P_{\ell_2,0})$ is not preinjective.
\end{lemma}
\begin{proof}
Alternative proof (maybe I miss something): First note that $n$ is not a neighbor of any other vertex than $\ell_i$ in $\mathrm{supp}P_{\ell_i,0}$. As $\ell_i$ is a limb vertex, reflecting at all vertices in the type $A$ component except $\ell_i$, we can assume that $P_{\ell_i}=S_{\ell_i}$. This means that all arrows in the type $A$ component with leaf $\ell_i$ point towards $\ell_i$. As $\ell_i$ is a limb vertex and $n$ a source, $\ell_i$ is a sink.  

%Then we have $\Sc{\udim S_{l_i}}{\udim S_{l_j}}=0$ for $i\neq j$. Moreover, we have
%$$\Hom(S_{\ell_i},S_n)=0=\Hom(S_n,P_{\ell_i})=\Ext(P_{\ell_i}, S_n)=0.$$

We have $\dim\Ext(S_n,P_{\ell_i})=|Q_0(n,\ell_i)|=1$ as there is precisely one arrow from $n$ to $\ell_i$. Thus $n+\ell_1+\ell_2$ is an exceptional root. As $E$ is indecomposable, it already follows that 
$$S_n=\Sigma_{\ell_1}\Sigma_{\ell_2} E\in\rep(Q')$$
where $Q'=\mu_{\ell_1}\mu_{\ell_2} Q$. In $Q'$ we have $\{\ell_1,\ell_2\}\subset Q_1(-,n)$ and thus $|Q_1(-,n)|\geq 2$. In particular, $S_n$ is not a preinjective representation of $Q'$ by Proposition \ref{pro:simpleregular} which means that $E$ is not a preinjective representation of $Q$.

--------------------------------------------

  First note that $Q$ is not of finite representation type and so there exists $\ell\in Q_0(n,-)$ with $\ell\ne\ell_1,\ell_2$.
  If $\ell$ is a limb with respect to vertex $n$ and $Q_0(n,-)=\{\ell,\ell_1,\ell_2\}$, then an easy calculation shows that $E\cong\tau^{-r}P_{\ell',0}$ is preprojective, here $\ell'$ is the leaf vertex of $Q$ in the subquiver $Q(\ell)\subset Q$ and $r$ is the number of arrows in $Q(\ell)$ directed towards $\ell'$.
  Since $Q$ is not of finite representation type, this implies $E$ is not preinjective.
  Thus we may assume that $Q_0(n,-)$ contains a vertex which is not a leaf or that $|Q_0(n,-)|\ge4$.
  
  There exists a commutative diagram
  \[\xymatrix{
      & & 0 \ar[d] & 0 \ar[d] & \\
      & & P_{\ell_1,0} \ar@{=}[r]\ar[d] & P_{\ell_1,0} \ar[d] & \\
      0 \ar[r] & P_{\ell_2,0} \ar[r]\ar@{=}[d] & E \ar[r]\ar[d]\ar@{}[dr]|(.7){\lrcorner} & I_{\ell'_1,0} \ar[r]\ar[d] & 0\\
      0 \ar[r] & P_{\ell_2,0} \ar[r] & I_{\ell'_2,0} \ar[r]\ar[d] & S_n \ar[r]\ar[d] & 0\\
      & & 0 & 0 & 
    }\]
  Since the upper horizontal morphism (or the left hand vertical morphism) is an equality, the lower right square is a pullback and thus there exists an exact sequence
  \[\ses{E}{I_{\ell'_1,0}\oplus I_{\ell'_2,0}}{S_n}.\]
  This sequence is not split since $E\not\cong I_{\ell'_1,0},I_{\ell'_2,0}$ and so $E$ is not injective.

  Why is $E$ not strictly preinjective?  Does applying $\tau^{-1}$ to $E$ help?  
  This would produce surjective maps $P_{\ell_i,1}\onto \tau^{-1} E$, but I don't know how that helps.
\end{proof}


\begin{lemma}
  \label{le:nonpreinjective extensions}
  If $M\in\rep Q$ is indecomposable and not preinjective, then any extension $E\in\Ext(M,P_{i,m})$ of $M$ by a preprojective representation $P_{i,m}$ contains no preinjective summand.
\end{lemma}
\begin{proof}
  For sake of contradiction, assume $E$ contains a preinjective summand.

  By assumption, the quiver $Q$ is not of finite representation type so that $P_{i,m}$ is not preinjective for any $i\in Q_0$ and $m\ge0$.
  Therefore, after an APR-tilt, we may assume $E$ actually contains a simple injective summand, say $E\cong E'\oplus S_k$ for a source $k\in Q_0$.
  Since $M$ is indecomposable and not injective, this $S_k$ component must lie in the kernel of the projection $E\onto M$.
  That is, we have $S_k\subset P_{i,m}$ which implies $S_k$ is a summand of $P_{i,m}$, a contradiction.
\end{proof}

In the definitions that follow we will need to identify particularly nice subsets of $\Hom_Q(P_{i,m},P_{j,m})$ and of $\Hom_Q(P_{i,m},P_{j,m+1})$.
To do this efficiently, we introduce the universal covering quiver $\widetilde{Q}$ of $Q$.
\begin{definition}
  Let $W=W(Q)$ denote the free group generated by the arrows $\alpha\in Q_1$ of $Q$.
  The \emph{universal covering quiver} of $Q$ is the quiver $\widetilde{Q}$ with vertices $(i,w)$ for $i\in Q_0$ and $w\in W$, and arrows $(\alpha,w):\big(s(\alpha),w\big)\to \big(t(\alpha),w \alpha \big)$ for $\alpha\in Q_1$ and $w\in W$.
\end{definition}
There is a natural forgetful functor $F:\rep\widetilde{Q}\to\rep Q$ and we say $M\in\rep Q$ \emph{lifts to $\widetilde{Q}$} if there exists $\widetilde{M}\in\rep\widetilde{Q}$ with $F(\widetilde{M})=M$.
Below, when we assume that a representation $M$ lifts to $\widetilde{Q}$, we will also implicitly assume that such a lift $\widetilde{M}$ has been chosen.
This forgetful functor intertwines the action of the reflection functor $\Sigma_k:\rep Q\to\rep Q$ for $k$ a sink (resp.~source) vertex with the action of the composite reflection functor $\widetilde{\Sigma}_k:\rep\widetilde{Q}\to\rep\widetilde{Q}$ given by applying all reflections $\Sigma_{(k,w)}$, $w\in W$, associated to sinks (resp.~sources) of $\widetilde{Q}$ covering vertex $k$ of $Q$.
Analogously, the Auslander-Reiten translation $\tilde\tau:\rep\widetilde{Q}\to\rep\widetilde{Q}$ is naturally isomorphic to the product $\widetilde{\Sigma}_n\cdots\widetilde{\Sigma}_1$ and thus the actions of $\tau$ and $\tilde{\tau}$ are also intertwined by $F$. 

Each projective representation $P_i$, $i\in Q_0$, admits a canonical lift $\widetilde{P}_i$ which is symmetric with respect to permutations about the central vertex $(i,e)$; this can be obtained using the intertwined reflection functors of $\widetilde{Q}$ as described above.
By applying Auslander-Reiten translations on the universal covering quiver, we obtain the same claim for each preprojective representation $P_{i,m}$ for $i\in Q_0$ and $m\ge0$.

There is a natural action of $W$ on $\widetilde{Q}$ given on vertices by $w.(i,w')=(i,ww')$ and on arrows by $w.(\alpha,w')=(\alpha,ww')$.
This induces an action of $W$ on $\rep\widetilde{Q}$ which we denote by $\widetilde{M}\mapsto w.\widetilde{M}$ for a representation $\widetilde{M}\in\rep\widetilde{Q}$ and $w\in W$.
\begin{lemma}
  \label{le:hom lifts}
  Given representations $M,N\in\rep Q$ which lift to $\widetilde{Q}$, there is an isomorphism 
  \[\Hom_Q(M,N)\cong\bigoplus\limits_{w\in W} \Hom_{\widetilde{Q}}(w.\widetilde{M},\widetilde{N}).\]
\end{lemma}
\begin{lemma}
  For $i,j\in Q_0$ and $m\ge0$, the spaces 
  \[\Hom_{\widetilde{Q}}(\alpha.\widetilde{P}_{j,m},\widetilde{P}_{i,m})\qquad\text{and}\qquad\Hom_{\widetilde{Q}}(\alpha^{-1}.\widetilde{P}_{i,m},\widetilde{P}_{j,m+1})\]
  for $\alpha\in Q_1(i,j)$ are both one-dimensional.
\end{lemma}
\begin{proof}
  For $m=0$ and $j$ a sink in $Q$, both claims are immediate from the definitions via reflection functors.
  For $m>0$ and $j$ arbitrary, these follow from the first case inductively by applying reflection functors.
  \sayD{This requires the cokernels to not be preinjective, why does that have to be true?}\sayT{If the cokernel is injective, injective maps become surjective, see $\tilde D_4$ in subspace orientation}
	\end{proof}
\begin{definition}
  For $i,j\in Q_0$ and $m\ge0$, write $\Irr(P_{j,m},P_{i,m})\subset\Hom_Q(P_{j,m},P_{i,m})$ for the subspace corresponding to $\bigoplus_{\alpha\in Q_1(i,j)} \Hom_{\widetilde{Q}}(\alpha.\widetilde{P}_{j,m},\widetilde{P}_{i,m})$ in Lemma~\ref{le:hom lifts}.
  Similarly, let $\Irr(P_{i,m},P_{j,m+1})\subset\Hom_Q(P_{i,m},P_{j,m+1})$ denote the subspace corresponding to $\bigoplus_{\alpha\in Q_1(i,j)} \Hom_{\widetilde{Q}}(\alpha^{-1}.\widetilde{P}_{i,m},\widetilde{P}_{j,m+1})$ in Lemma~\ref{le:hom lifts}.
\end{definition}
\begin{corollary}
  For $i,j\in Q_0$ and $m\ge0$, the morphism spaces $\Irr(P_{j,m},P_{i,m})$ and $\Irr(P_{i,m},P_{j,m+1})$ are both $b_{ij}$-dimensional.
\end{corollary}
\begin{remark}
  The notation $\Irr(P_{j,m},P_{i,m})$ and $\Irr(P_{i,m},P_{j,m+1})$ is justified by the following observation: these spaces map isomorphically to the appropriate spaces $\Rad(M,N)/\Rad^2(M,N)$ of irreducible morphisms labeling arrows in the Auslander-Reiten quiver of $Q$.
  For the results that follow, it will be important to have honest morphisms and we fix these particular choices of lifts from such quotient spaces.
\end{remark}

For $k\in Q_0$, we take $P_{k,-1}=0$.
\begin{definition}
  For $i\in Q_0$ and $m\in\ZZ_{\ge-1}$, consider the tuple of representations 
  \[\cP_{i,m}=\big(P_{k,m}\big)_{k\in Q_0(-,i)} \smile\big(P_{j,m+1}\big)_{j\in Q_0(i,-)}\]
  and the tuple of vector spaces 
  \[\cI_{i,m}=\big(\Irr(P_{k,m},P_{i,m+1})\big)_{k\in Q_0(-,i)} \smile\big(\Irr(P_{j,m+1},P_{i,m+1})\big)_{j\in Q_0(i,-)}.\]
  We write $\cP_{i,m}\bullet\cI_{i,m}$ for the representation
  \[\bigoplus_{k\in Q_0(-,i)} P_{k,m}\otimes \Irr(P_{k,m},P_{i,m+1}) \oplus \bigoplus_{j\in Q_0(i,-)} P_{j,m+1}\otimes\Irr(P_{j,m+1},P_{i,m+1}).\]
  Call a tuple of subspaces $\cV\subsetneq\cI_{i,m}$ \emph{admissible} if $\cP_{i,m}\bullet(\cI_{i,m}/\cV)$ is not isomorphic to $P_{\ell,m}$ nor to $P_{\ell,m+1}$ for any vertex $\ell\in Q_0$ which is a limb with respect to $i$.
\end{definition}

\begin{lemma}
  \label{le:standard AR sequences}
  \mbox{}
  \begin{enumerate}
    \item For $i\in Q_0$, there exists an exact sequence
      \begin{equation}
        \label{eq:projective AR}
        \ses{\cP_{i,-1}\bullet\cI_{i,-1}}{P_{i,0}}{S_i}.
      \end{equation}
    \item For $i\in Q_0$ and $m\in\ZZ_{\ge0}$, there exists an Auslander-Reiten sequence
      \begin{equation}
        \label{eq:preprojective AR}
        \ses{P_{i,m}}{\cP_{i,m}\bullet\cI_{i,m}}{P_{i,m+1}}.
      \end{equation}
      %\begin{equation}
      %  \label{eq:preprojective AR}
      %  \ses{P_{i,m}}{\bigoplus\limits_{k\in Q_1(-,i)} P_{k,m}\otimes\Irr(P_{k,m},P_{i,m+1})\oplus\bigoplus\limits_{j\in Q_1(i,-)} P_{j,m+1}\otimes \Irr(P_{j,m+1},P_{i,m+1})}{P_{i,m+1}};
      %\end{equation}
  \end{enumerate}
\end{lemma}
\begin{proof}
  Part (2) can be reduced to the case $m=0$ by applying Auslander-Reiten translations and then to the case where vertex $i$ is a sink in $Q$ using reflection functors.
  Thus it suffices to establish the existence of an Auslander-Reiten sequence 
  \[\ses{P_{i,0}}{\bigoplus\limits_{k\in Q_0(-,i)} P_{k,0}\otimes\Irr(P_{k,0},P_{i,1})}{P_{i,1}}\]
  when $i\in Q_0$ is a sink.
  This sequence and the one from part (1) follow immediately from the definitions $P_{k,0}:=\Sigma_1\cdots\Sigma_{k-1} S_k$ and $P_{i,1}:=\Sigma_1\cdots\Sigma_n P_{i,0}$.
\end{proof}

The sequences \eqref{eq:projective AR} and \eqref{eq:preprojective AR} are essential for the results to follow.
In some cases, the left hand term $P_{i,m}$ in \eqref{eq:preprojective AR} surjects onto one of the summands of the middle term and the sequence \eqref{eq:preprojective AR} will not serve our purposes.
The next results describe situations when this will occur and provide the correct alternative sequences to consider.
\begin{corollary}
  \label{cor:thin sequences}
  Suppose $i\in Q_0$ is adjacent to a limb vertex $\ell$ with associated leaf $\ell'$.
  \begin{enumerate}
    \item If $\ell\in Q_0(i,-)$, then for any $m\ge h(\ell)$, we have an exact sequence
      \begin{equation}
        \label{eq:thin limb sequence 1}
        \ses{P_{\ell',m-h(\ell)}}{P_{i,m}}{P_{\ell,m+1}}.
      \end{equation}
    \item If $\ell\in Q_0(-,i)$, then for any $m\ge h(\ell)$, we have an exact sequence
      \begin{equation}
        \label{eq:thin source sequence}
        \ses{P_{\ell',m-h(\ell)}}{P_{i,m}}{P_{\ell,m}}.
      \end{equation}
  \end{enumerate}
\end{corollary}
\begin{proof}
  We work by simultaneous induction on $h(\ell)$ and $m$.
  When $h(\ell)=0$, the type A quiver $Q(\ell)$ is linearly oriented from vertex $i$ to the leaf $\ell'$, in particular $\ell\in Q_0(i,-)$ and $\ell'$ is a sink of $Q$.

  If $\ell'=\ell$, then Lemma~\ref{le:standard AR sequences} gives the desired exact sequence
  \[\ses{P_{\ell,m}}{P_{i,m}}{P_{\ell,m+1}}\]
  for $m\ge0$.
  When $\ell'\ne\ell$, there exists $\ell''\in Q_0(\ell,-)$ which is itself a limb vertex with respect to $\ell$ and by induction we obtain an exact sequence
  \[\ses{P_{\ell',m}}{P_{\ell,m}}{P_{\ell'',m+1}}\]
  for $m\ge0$.
  Lemma~\ref{le:standard AR sequences} gives us an exact sequence
  \[\ses{P_{\ell,m}}{P_{i,m}\oplus P_{\ell'',m+1}}{P_{\ell,m+1}}\]
  which reduces to the desired sequence using Lemma~\ref{le:sequence reducing}.

  Now assume $h(\ell)>0$.
  If $\ell'=\ell$, then $\ell\in Q_0(-,i)$ is a source of $Q$.
  In this case, Lemma~\ref{le:standard AR sequences} gives the desired exact sequence
  \[\ses{P_{\ell,m}}{P_{i,m+1}}{P_{\ell,m+1}}\]
  for $m\ge0$.
  
  Assume $\ell'\ne\ell$.
  Write $\ell''$ for the limb vertex with respect to $\ell$.
  Then there are several cases to consider.
  \begin{itemize}
    \item If $\ell\in Q_0(i,-)$ and $\ell''\in Q_0(\ell,-)$, then $h(\ell'')=h(\ell)$ and by induction we obtain an exact sequence
      \[\ses{P_{\ell',m-h(\ell)}}{P_{\ell,m}}{P_{\ell'',m+1}}\]
      for $m\ge h(\ell)$.
      Lemma~\ref{le:standard AR sequences} gives an exact sequence
      \[\ses{P_{\ell,m}}{P_{i,m}\oplus P_{\ell'',m+1}}{P_{\ell,m+1}}\]
      which reduces to the desired sequence using Lemma~\ref{le:sequence reducing}.
    \item If $\ell\in Q_0(i,-)$ and $\ell''\in Q_0(-,\ell)$, then $h(\ell'')=h(\ell)$ and by induction we obtain an exact sequence
      \[\ses{P_{\ell',m-h(\ell)}}{P_{\ell,m}}{P_{\ell'',m}}\]
      for $m\ge h(\ell)$.
      Lemma~\ref{le:standard AR sequences} gives an exact sequence
      \[\ses{P_{\ell,m}}{P_{i,m}\oplus P_{\ell'',m}}{P_{\ell,m+1}}\]
      which reduces to the desired sequence using Lemma~\ref{le:sequence reducing}.
    \item If $\ell\in Q_0(-,i)$ and $\ell''\in Q_0(\ell,-)$, then $h(\ell'')=h(\ell)-1$ and by induction we obtain an exact sequence
      \[\ses{P_{\ell',m-h(\ell)}}{P_{\ell,m-1}}{P_{\ell'',m}}\]
      for $m-1\ge h(\ell)-1$.
      Lemma~\ref{le:standard AR sequences} gives an exact sequence
      \[\ses{P_{\ell,m-1}}{P_{i,m}\oplus P_{\ell'',m}}{P_{\ell,m}}\]
      which reduces to the desired sequence using Lemma~\ref{le:sequence reducing}.
    \item If $\ell\in Q_0(-,i)$ and $\ell''\in Q_0(-,\ell)$, then $h(\ell'')=h(\ell)-1$ and by induction we obtain an exact sequence
      \[\ses{P_{\ell',m-h(\ell)}}{P_{\ell,m-1}}{P_{\ell'',m-1}}\]
      for $m-1\ge h(\ell)-1$.
      Lemma~\ref{le:standard AR sequences} gives an exact sequence
      \[\ses{P_{\ell,m-1}}{P_{i,m}\oplus P_{\ell'',m-1}}{P_{\ell,m}}\]
      which reduces to the desired sequence using Lemma~\ref{le:sequence reducing}.
  \end{itemize}
\end{proof}

For $i\in Q_0$ and $m\ge0$, given $\ell\in Q_0(-,i)\cup Q_0(i,-)$ we write $\cI_{i,m}^{\{\ell\}}\subset\cI_{i,m}$ for the tuple of subspaces with the only proper subspace at position $\ell$ and the zero subspace in that component.
\begin{corollary}
  \sayT{We have $|\cI_{i,m}^{\{\ell\}}|=1$ right?}
  \sayD{We have $\codim_{\cI_{i,m}}(\cI_{i,m}^{\{\ell\}})=1$ in this case.}
  \label{cor:reduced sequences}
  Suppose $i\in Q_0$ is adjacent to a limb vertex $\ell$ with associated leaf $\ell'$.
  Then there is an exact sequence
  \begin{equation}
    \label{eq:reduced AR}
    \ses{P_{\ell',m-h(\ell)}}{\cP_{i,m}\bullet\cI_{i,m}^{\{\ell\}}}{P_{\ell,m+1}}.
  \end{equation}
\end{corollary}
\begin{proof}
  This is immediate by applying Lemma~\ref{le:sequence reducing} to the sequences from Corollary~\ref{cor:thin sequences} and Lemma~\ref{le:standard AR sequences}.
\end{proof}

The following result is crucial to the constructions to follow.
\begin{lemma}
  Consider $i\in Q_0$ and $m\ge0$.
  \begin{enumerate}
    \item For any tuple of subspaces $\cV\subsetneq\cI_{i,-1}$, the natural evaluation map $\cP_{i,-1}\bullet\cV\to P_{i,0}$ is injective.
      \sayT{$\cV$ admissible in (1)?}
      \sayD{I believe this is true as stated with no assumptions on $\cV$.}
    \item For any admissible tuple $\cV\subsetneq\cI_{i,m}$, the natural evaluation map $\cP_{i,m}\bullet\cV\to P_{i,m+1}$ is injective.
    \item The sequence \eqref{eq:preprojective AR} induces isomorphisms 
      \[\Irr(P_{k,m},P_{i,m+1})^*\cong \Irr(P_{i,m},P_{k,m}) \qquad \text{and} \qquad \Irr(P_{j,m+1},P_{i,m+1})^*\cong \Irr(P_{i,m},P_{j,m+1})\]
      for each $k\in Q_0(-,i)$ and each $j\in Q_0(i,-)$.
  \end{enumerate}
\end{lemma}
\begin{proof}
  Part (1) immediately follows from the exactness of the sequence \eqref{eq:projective AR}.
  Thus for an admissible tuple~$\cV\subsetneq\cI_{i,-1}$, say with $V_j\subset\Irr(P_{j,0},P_{i,0})$ in the $j$-th component, we have an exact commutative diagram:
  \[\xymatrix{
      & 0 \ar[d] & 0 \ar[d] & & \\
      0 \ar[r] & \bigoplus\limits_{j\in Q_0(i,-)} P_{j,0}\otimes V_j \ar@{=}[r] \ar[d] & \bigoplus\limits_{j\in Q_0(i,-)} P_{j,0}\otimes V_j \ar[r]\ar[d] & 0 \ar[d] & \\
      0 \ar[r] & \bigoplus\limits_{j\in Q_0(i,-)} P_{j,0}\otimes\Irr(P_{j,0},P_{i,0}) \ar[r]\ar[d] & P_{i,0} \ar[r]\ar[d] & S_i \ar[r]\ar@{=}[d] & 0 \\
      0 \ar[r] & \bigoplus\limits_{j\in Q_0(i,-)} P_{j,0}\otimes\big(\Irr(P_{j,0},P_{i,0})/V_j\big) \ar[r]\ar[d] & C \ar[r]\ar[d] & S_i \ar[r]\ar[d] & 0 \\
      & 0 & 0 & 0 & 
    }
  \]
  
  Part (2) is trivial if $\cV$ has all components zero, so we assume this is not the case.
  First observe that after an APR-tilt (applying a sequence of reflection functors) it suffices to show in the case $m=-1$ with vertex $i=n$ the unique source of $Q$ that the representation $C$ in the diagram above contains no preinjective summands.

  There are two cases to consider.
  \begin{itemize}
    \item Assume $\bigoplus\limits_{j\in Q_0(i,-)} P_{j,0}\otimes\big(\Irr(P_{j,0},P_{i,0})/V_j\big)$ contains a summand $P_{\ell,0}$ for $\ell\in Q_0(n,-)$ which is not a limb with respect to the source vertex $n$.
      Then the result follows by induction using Lemma~\ref{le:nonlimb extensions} and Lemma~\ref{le:nonpreinjective extensions}.
    \item Assume all summands of $\bigoplus\limits_{j\in Q_0(i,-)} P_{j,0}\otimes\big(\Irr(P_{j,0},P_{i,0})/V_j\big)$ correspond to vertices which are limbs with respect to $n$, say $V_j=0$ exactly for $j\in\{i_1,\ldots,i_q\}$.
      Since $\cV$ is admissible, we must have $q\ge2$.
  \end{itemize}

  A simple homological calculation shows that $\End(C)\cong\End(P_{i,0})$ is one-dimensional.
  Therefore $C$ must be indecomposable and hence preinjective.
  Thus there exists $k\in Q_0$ with $C\cong \tau^r\Sigma_n\cdots \Sigma_{k+1} S_k$ for some $r\ge0$.

  Note that the lower horizontal sequence in the diagram above shows that $C$ contains a one-dimensional space at vertex $n$.
  As $C$ has a one-dimensional space at vertex $n$, this implies there is a unique path in $Q$ from vertex $k$ to vertex $n$.
  Since $\cV\subsetneq\cI_{n,-1}$ is a proper subspace, $C$ contains a projective subrepresentation and so vertex $k$ must be a sink in $Q$.
  
  
  are two cases to consider.
  We need to verify that $\cV$ being admissible implies that $P_{i,0}/\cP_{i,-1}\bullet\cV$ contains no preinjective summands.
  This then implies the statement of part (2) by applying reflection functors.

  By Lemma~\ref{le:preprojective ext groups}, we have $\Ext(\cP_{i,m}\bullet\cV,P_{i,m})=0$.
  Thus, by taking pullbacks, we obtain a commutative diagram 
  \[\xymatrix{0 \ar[r] & P_{i,m}\ar@{=}[d]\ar[r] & P_{i,m}\oplus \cP_{i,m}\bullet\cV \ar@{}[dr]|(.7){\lrcorner} \ar[r]\ar[d]_{(\iota_{i,m},\mathrm{id}_{\cP_{i,m}}\bullet\iota_\cV)} & \cP_{i,m}\bullet\cV \ar[r]\ar_{ev_\cV}[d] & 0\\
  0 \ar[r] & P_{i,m}\ar[r]^{\iota_{i,m}} & \cP_{i,m}\bullet \cI_{i,m} \ar[r]^{ev} & P_{i,m+1} \ar[r] & 0}\]
  
\end{proof}

\begin{definition}
  For $i\in Q_0$, $m\in\ZZ_{\ge0}$, and $\cV\subsetneq\cI_{i,m}$, define the \emph{truncated preprojective} representation $P_{i,m+1}^\cV=P_{i,m+1}/\cP_{i,m}\bullet\cV$.
\end{definition}

\begin{theorem}
  The quiver Grassmannians of the truncated preprojective $P_{i,m+1}^\cV$ only depend on the dimensions of the subspaces $V_j$ of irreducible morphisms.
\end{theorem}
\begin{proof}
  Idea: use $GL_{b_{ij}}(\CC)$-actions on the spaces of irreducible morphisms as in the Kronecker cases. 
\end{proof}

\begin{proposition}
  Consider $\cV\subsetneq\cI_{i,m}$ with $\codim_{\cI_{i,m}}(\cV)=1$, say $(\cP_{i,m}\bullet\cI_{i,m})/(\cP_{i,m}\bullet\cV)\cong P_{\ell,n}$ with $n=m$ or $n=m+1$.
  Then $P_{i,m+1}^\cV\cong P_{\ell,n}^{\overline{\cV}}$, where $\overline{\cV}=..$.
\end{proposition}
\begin{proof}
  There is a commutative diagram of the following form:
  \[\xymatrix{
      & & 0 \ar[d] & 0 \ar[d] \\
      & & \cP_{i,m}\bullet\cV \ar[d]\ar@{=}[r] & \cP_{i,m}\bullet\cV \ar[d] \\
      0 \ar[r] & P_{i,m} \ar[r]\ar@{=}[d] & \cP_{i,m}\bullet\cI_{i,m}  \ar[d]\ar[r] & P_{i,m+1} \ar[r]\ar[d] & 0\\
      0 \ar[r] & P_{i,m} \ar[r] & P_{\ell,n} \ar[r]\ar[d] & P_{i,m+1}^\cV \ar[r]\ar[d] & 0 \\
      & & 0 & 0}\]
  It follows that we may identify $P_{i,m+1}^\cV$ with a truncated preprojective $P_{\ell,n}^{\overline{\cV}}$.
\end{proof}

\begin{lemma}
  For $i\in Q_0$, write $L_-\subset Q_0(-,i)$ and $L_+\subset Q_0(i,-)$ for the subsets of limbs.
  Then given any subset $L=\{\ell,\ell''\}\subset L_-\sqcup L_+$ and $m\ge h(\ell)$, the truncated preprojective representation
  \[
    P_{i,m+1}^{\hat L}:=P_{i,m+1}/\left( \bigoplus_{k\in Q_0(-,i)\setminus L} P_{k,m}\otimes \Irr(P_{k,m},P_{i,m+1}) \oplus \bigoplus_{j\in Q_0(i,-)\setminus L} P_{j,m+1}\otimes \Irr(P_{j,m+1},P_{i,m+1}) \right)
  \]
  is isomorphic to 
  \[
    \begin{cases}
      P_{\ell'',m}/P_{\ell',m-h(\ell)} & \text{if $\ell''\in Q_0(-,i)$;}\\
      P_{\ell'',m+1}/P_{\ell',m-h(\ell)} & \text{if $\ell''\in Q_0(i,-)$;}\\
    \end{cases}
  \]
  where $\ell'$ is the leaf associated to the limb $\ell\in L$.
\end{lemma}
\begin{proof}
  Using an analogue of Lemma~\ref{le:quotients} with the sequence \eqref{eq:preprojective AR}, we get an isomorphism of the truncated preprojective $P_{i,m+1}^{\hat L}$ with the representation
  \[
    \Big( \bigoplus_{\ell\in L_-\cap L} P_{\ell,m}\otimes \Irr(P_{\ell,m},P_{i,m+1}) \oplus \bigoplus_{\ell\in L_+\cap L} P_{\ell,m+1}\otimes \Irr(P_{\ell,m+1},P_{i,m+1}) \Big)/P_{i,m}.
  \]
  Note that these multiplicity spaces are each 1-dimensional.
  The result then follows by applying Lemma~\ref{le:sequence reducing} using an appropriate sequence from Corollary~\ref{cor:thin sequences}.
\end{proof}

%%%%%%%%%%%%%%%%%%%%%%%%%%%%%%%%%%%%%%%%%%%%%%%%%%%%%%%%%%%%%%%%%
\section{Truncated preprojective representations of tree quivers}

  \noindent In this section we assume that the underlying graph of $Q$ is connected tree. 
  Recall that, for two indecomposable representations $M$ and $N$, a morphism $f:M\to N$ is called \emph{sectional} if it can be decomposed as $f=f_1\circ\ldots\circ f_r$, where each morphism $f_i:M_{i-1}\to M_i$ is an irreducible morphism between indecomposable representations and, furthermore, $M_i\ncong \tau^{-1}M_{i-2}$.
  In other words, $f$ is given as an oriented path in the Auslander-Reiten quiver of $Q$ which avoids traveling along meshes.
  \sayD{Does this have a useful interpretation as a representation of the Auslander-Reiten quiver?}
  If $M$ and $N$ are preprojective, it follows by the Happel-Ringel Lemma that a sectional morphism is either injective or surjective.
  In particular, irreducible morphisms are always injective or surjective.
  Note that each sectional morphism corresponds to a (non-oriented) path in $Q$.
  
  As before, for $i\in Q_0$, we write $P_{i,m}=\tau^{-m}P_i$, where $P_i$ is the projective cover of the vertex simple $S_i$.  
  \begin{lemma}
   % \sayT{ statement should be (close to) correct now, proof must be adjusted}
    Consider the unique filtration $P_{i_0,m_0}\subset P_{i_1,m_1}\subset\ldots\subset P_{i_k,m_k}$, where $m_0=0$ and $m_j=m_{j-1}+1$ if $i_j\in Q_0(i_{j-1},-)$ and $m_j=m_{j-1}$ otherwise.
    Then $P_{i_k,m_k}/P_{i_0,m_0}$ is preinjective if and only if $i_1$ is a limb with respect to $i_0$ or if there exists at most one vertex $i_s$ with $s=1,\ldots, k-1$  which is not a limb with $N_{i_s}=\{i_{s-1},i_{s+1}, q\}$ and, additionally, one of the following conditions holds: 
		\begin{enumerate}
		
		\item $s=1$ and, furthermore, $i_k$ is a leaf and $q$ is a limb;
		\item $s=2$ and, furthermore, $i_k$ and $q$ are both leaves.
		\item $Q$ is a flag quiver and has one $3$-valent vertex $i_s$ where $s=3$, $k=5$ and, furthermore, $i_k$ and $q$ are both leaves. 
		\item $s=k-1$ and, furthermore, $i_k$ and $q$ are both leaves.
		%\item If $s=k$, then $i_1$ is a limb, i.e. $q=i_{k+1}$.
		\end{enumerate}
\end{lemma}
\begin{proof}
  We may without loss of generality assume that that all arrows are oriented towards $i_0$. Then we have $i_j\in Q_0(-,i_{j-1})$ for $j=1,\ldots,k$, $m_k=0$ and
  $$\alpha:=\udim P_{i_k}/P_{i_0}=\sum_{j=1}^{k}i_j.$$
	  If $i_k$ is not a limb with respect to $i_{k-1}$,  Proposition \ref{pro:simpleregular} shows that $S_{i_k}$ is a regular quotient of $P_{i_k}/P_{i_0}$. In turn, $P_{i_k}/P_{i_0}$ is not preinjective. Thus in the following assume that $i_k$ is limb.


	For any root $\delta$, we have 
$$\Sc{\delta}{\alpha}=\sum_{j=1}^k\delta_j-\sum_{j=1}^k\sum_{l\in Q(-,j)}\delta_l.$$
Recall that if $\delta$ is a regular root with $\Sc{\delta}{\alpha}<0$, we have $\Ext(M_\delta,M_\alpha)\neq 0$ for all representations of dimension $\delta$ and $\alpha$, and thus $\alpha$ cannot be a preinjective root.

	Assume that there are two vertices $i_s$ and $i_t$ with $1\leq s,t\leq k$ and
  $$c=|Q_0(-,i_s)\backslash\{i_{s+1}\}\cup Q_0(-,i_t)\backslash\{i_{t+1}\}|\geq 2.$$
	Here $i_{k+1}$ is any neighbor of $i_k$ except $i_{k-1}$. Let $\{q_1,q_2\}\in Q_0(-,i_s)\backslash\{i_{s+1}\}\cup Q_0(-,i_t)\backslash\{i_{t+1}\}$.
  If $s,t\neq k$, then  $\delta=\sum_{j=0}^ki_j+\sum_{j=s}^ti_j+q_1+q_2$ is a regular root such that $\Sc{\delta}{\alpha}<0$. Indeed, $\delta$ is the unique imaginary Schur of a subquiver of type $\tilde D$ which is a regular root of $Q$ by Lemma \ref{lem:regular support}.	
	
Thus assume that $t=k$ and $s\neq k$. 
Then we have
$$\Sc{\delta}{\alpha}\leq \delta_{i_1}-\delta_{q_1}-\delta_{q_2}.$$


As $Q$ is wild or of extended Dynkin type such that $i_k$ is not a leaf and $i_s$ (and thus $i_1$) is not a limb with respect to $i_0$, there exists a regular root with $\delta_{i_1}<\delta_{q_1}+\delta_{q_2}$. Indeed, if $i_s$ is a $3$-valent vertex and, moreover, the only $n$-valent vertex with $n\geq 3$, then $Q$ has a subquiver of extended Dynkin type $\tilde E_n$ which includes all vertices under consideration and thus the unique imaginary Schur root $\delta$ satisfies the condition, see \cite[Section 4]{CB} for a list. If $Q$ has more than one $n$-valent vertex with $n\geq 3$, then it is rather straightforward to construct a sufficient regular root $\delta$.\sayT{details here or below?}
\begin{comment}
The unique exceptional representation of dimension 
$\alpha:=\sum_{j=s}^ki_j$ is a factor of $P_{i_k}/P_{i_0}$. Consider the full subquiver $Q'$ with vertices $Q_0\backslash\{i_{k+1},\ldots,i_n\}$. When reflecting successively at the sources $i_k,\ldots,i_{s-1}$ the representation $P_{i_k}/P_{i_0}$ transforms into the simple representation $S_s$ of the corresponding transformation of $Q'$. By Proposition \ref{pro:simpleregular}, the representation $S_s$ is a preinjective representation of $Q'$ if and only if $q_1$ is a limb with respect to $s$. By Lemma \ref{lem:regular support} and the argument from above, this shows that $P_{i_k}/P_{i_0}$ is regular if $q_1$ is not limb.
\end{comment}

 

Thus we have shown so far that $\alpha$ can only be preinjective if $c=1$.  A similar argument as the last one shows that $q_1$ is forced to be a limb. Note that if $q$ were no limb, then either $i_k$ is no limb or there exists more than one $n$-valent vertex with $n\geq 3$.

Thus we assume that $c=1$ and that $q:=q_1$ is a limb. We first collect conditions which ensure $\alpha$ is preinjective.

If $i_1$ is a limb with respect to $i_0$ (whence $i_k$ is a limb, but not necessarily a leaf), it can be checked straightforwardly that $P_{i_k}/P_{i_0}$ is preinjective as it the Auslander-Reiten translate of the indecomposable injective representation corresponding to one of the vertices of the type $A$ subquiver with leaf $i_1$.

If $s=1$, $i_k$ is a leaf and $q$ is a limb with respect to $i_1$, applying reflection functors successively to the vertices $i_k,\ldots, i_2$, the representation $P_{i_k}/P_{i_0}$ can be transformed into the simple representation $S_s$ which is a preinjective representation of the transformed quiver following Proposition \ref{pro:simpleregular}. 

If $s=2$ and, moreover, $i_k$ and $q$ are leaves, we have $\tau^{-2}\alpha=i_3$ which is preinjective by Proposition \ref{pro:simpleregular}. Thus $\alpha$ is preinjective.

If $i_s$ is the only $3$-valent vertex, $s=3$, $k=5$ and, moreover, $q$ and $i_k$ are leaves, then $\tau^{-3}\alpha$ is the injective root corresponding to the unique leaf other than $q$ and $i_k$ which means that $\alpha$ is preinjective.

If $q$ and $i_k$ are leaves and $s=k-1$, we have $\tau^{-s}\alpha=i_k$ if $s$ is odd and $\tau^{-s}\alpha=q$ if $s$ is even. Thus $P_{i_k}/P_{i_0}$ is preinjective as it is the translate of an injective representation.

%Thus if
%$$|\bigcup_{j=1}^kQ_0(-,i_j)\backslash\{i_{j+1}\}|=T=:c\leq 1,$$
%the representation $P_{i_k}/P_{i_0}$ is preinjective if and only if the single neighbor is a limb.

In particular, $\alpha$ can only be preinjective if $i_k$ is a leaf or $i_1$ is a limb with respect to $i_0$.  As the preceding observations included the last case, we have to deal with the case that $c=1$, $k\geq s\geq 2$, $q=q_1$ is a limb with respect to $i_s$ and $i_k$ is a leaf. Moreover, we can assume that $q$ is not a leaf if $s=k-1$ or $s=2$. If $s=3$ and $k=5$, then $Q$ can be assumed to have more than one $n$-valent vertex with $n\geq 3$. 
%This means that there exists no subquiver $Q'$ of extended Dynkin type including the vertices $i_k$ and $q$ which is of type $D$.\sayT{do we need this last remark?} 

%It remains to show that $\alpha$ is regular in the remaining cases where we may assume that $c\leq 2$, $2 \leq s\neq k$, $t=k$ and, furthermore, that $q_1$ is a limb with respect to $i_s$ and $q_2$ is a limb with respect to $i_k$. Then we have
%$$\Sc{\delta}{\alpha}=\delta_{i_1}-\delta_{q_1}-\delta_{q_2}.$$

Then we have 
$$\Sc{\delta}{\alpha}=\delta_{i_1}-\delta_q.$$

If $Q$ has a $n$-valent vertex other than $i_s$ with $n\geq 3$, in all these cases it is again straightforward to check that there exists a regular root $\delta$ which satisfies the condition $\Sc{\delta}{\alpha}<0$. \sayT{double-check from here on}

Thus assume that $i_s$ is the only $n$-valent vertex of $Q$ with $n\geq 3$. Then $Q$ has a subquiver which is of extended Dynkin type $\tilde E$. If we have $\Sc{\delta}{\alpha}= 0$ for the imaginary Schur root $\delta$ of this subquiver, \cite[Section 7, Lemma 2]{CB} shows that $\alpha$ is regular and in turn a regular root of $Q$.

If $k\geq s+2$ and if $q$ is not a leaf, then having in mind the regular roots of \cite[Section 4]{CB}, this argument can be used to show that $\alpha$ is a regular root.

Thus two cases remain to be considered: $k=s+1,\,s\geq 2$, and $q$ is not a leaf, and $3\leq s\leq k-2$ and $q$ is a leaf (the case $s=2$ and $q$ a leaf was treated before). 
As $Q$ has a subquiver of extended Dynkin type $\tilde E_7$ or $\tilde E_8$ and as $q$ is not a leaf, in the first case, the same argument shows that $\alpha$ is a regular root.

Thus assume that $q$ is a leaf and $3\leq s\leq k-2$ . If $k\geq s+3$, then $\alpha$ can be checked to be regular as before. Thus assume that $k=s+2$. As the case $s=3$ was treated before, we also assume that $s\geq 4$. But as $Q$ contains a subquiver of type $\tilde E_8$, as before $\alpha$ can be shown to be a regular root of this subquiver and thus it is a regular root of $Q$.

%If $Q$ is of type $\tilde E_8$, then we have  $\Sc{\delta}{\alpha}=1$ if $s=3$ for the imaginary Schur root. Thus $\alpha$ is preinjective in this case, see \cite[Section 7, Lemma 2]{CB} for the argument and \cite[Section 4]{CB} for a list of the imaginary Schur roots of extended Dynkin quivers. Moreover, is regular if $s\geq 4$.

%Thus let $s=3$ and assume that $Q$ contains a proper subquiver of type $\tilde E_8$. If $k=5$, then it can be checked with help of Proposition \ref{pro:simpleregular} that $\alpha$ is preinjective. If $k\geq 6$, then $\alpha$ is regular.

\begin{comment}
We first assume that $Q$ itself is of extended Dynkin type.\sayT{continue here, the rest are notes which are maybe not entirely correct}

If $ Q=\tilde E_6$, the root $\alpha$ is regular in any case as $\Sc{\delta}{\alpha}=0$, see \cite[Section 7, Lemma 2]{CB} for the argument and \cite[Section 4]{CB} for a list of the imaginary Schur roots of extended Dynkin quivers.
If $Q=\tilde E_7$, the same argument shows that the root $\alpha$ is regular if $k=3$ and $s=2$ and preinjective if $k=5$, $s=2$ and $q$ is a leaf (which are the only possibilities and where the last case was considered before).
If $Q=\tilde E_8$, the same argument shows that the root $\alpha$ is regular if $s=2$ and $k=3$ and preinjective if $s=2$ and $k=7$, $s=3$ and $k=5$ or $s=5$ and $k=7$ or $k=8$ (which are the only possibilities).\sayT{double-check}
This shows that $\alpha$ is preinjective if $q_1$ is a leaf of $i_s$ in the case $Q$ is of extended Dynkin type $\tilde D_n$, $\tilde E_6$ or $\tilde E_7$. If $Q$ is of type $\tilde E_8$, it is a limb whose neighbor is a leaf or it is a leaf itself.

Assume that $Q$ is wild with $s\geq 3$ and that either $q_1$ is not a leaf or that $s<k-1$. In this case $\alpha$ is regular. If $q_1$ is not a leaf, we already proved that $\alpha$ is regular ($Q$ has a subquiver of extended Dynkin type). Thus assume that $q_1$ is a leaf and $3\leq s\leq k-2$. Moreover, we have $k\geq 6$ if $Q$ has only one $3$-valent vertex. First assume that $k\geq 6$. Then it has a subquiver of type $\tilde E_7$ and the imaginary Schur root does the job.
 \end{comment}


%To see this construct a regular root $\delta$ with $\Sc{\delta}{\alpha}<0$.

 %it has a subquiver of extended Dynkin type. 

%If it is of type $\tilde E_6$, we are done as $\alpha$ is already regular as a root of this quiver.
%Let us assume that this quiver is of type $\tilde E_8$. Then $q_1$ 

\end{proof}
\begin{lemma}
  \label{lem:preinjective quotients}
  \mbox{}
\begin{enumerate}
\item Let $j\in Q_0(-,i)$. Then $P_j/P_i$ is preinjective if and only if $j$ is a limb with respect to $i$. Furthermore, we have $\tau^{-h(j)}(P_j/P_i)=0$ in this case.
\item Let $j\in Q_0(i,-)$. Then $P_{j,1}/P_i$ is preinjective if and only if $j$ is a limb with respect to $i$. Furthermore, we have $\tau^{-h(j)-1}(P_{j,1}/P_i)=0$ in this case.
\end{enumerate}
\end{lemma}
\begin{proof} As $Q$ is a tree, deleting the arrow $j\to i$, yields two connected components $Q(i)$ and $Q(j)$ containing $i$ and $j$ respectively. As we did in Proposition \ref{pro:simpleregular}, we may successively apply reflection functors to sinks of $Q(j)$ in such a way that we can without loss of generality assume that $Q_0(j,-)=\{i\}$. This yields $P_j/P_i\cong S_j$. By Proposition \ref{pro:simpleregular}, $S_j$ is preinjective if and only if $j$ is a limb with respect to $i$. 
Now $\tau^{-h(j)}(P_j/P_i)=0$ can be checked straightforwardly.\sayT{we should do this somewhere as we use it frequently}This shows the first part.

Similar to the proof of the first part, in a first step we may assume that $Q_0(j,-)=\emptyset$. But this means that $j$ is a sink. After reflecting at $j$, the result follows from the first part. 


\end{proof}


\begin{lemma}\label{lem:quotients projectives}
Let $j\in Q_0(-,i)$ and $P_j/P_i$ not injective\sayT{condition in terms of limbs}. Then there exists a long exact sequence
$$0\to\bigoplus_{k\in Q_0(i,-)}P_k\to P_i\to\tau(P_j/P_i)\to\bigoplus_{\substack{k\in Q_0(-,i)\\k\neq j}}I_k \to 0$$
inducing two short exact sequences
$$0\to\bigoplus_{k\in Q_0(i,-)}P_k\to P_i\to S_i\to 0,\qquad 0\to S_i\to \tau(P_j/P_i)\to\bigoplus_{\substack{k\in Q_0(-,i)\\k\neq i}}I_k\to 0.$$
\end{lemma}
\begin{proof}
Consider the (unique) short exact sequence
$$\ses{P_i}{P_j}{P_j/P_i}$$
and the induced non-zero map $f:P_i\to\tau(P_j/P_i)$. Since $P_j$ and $P_i$ are projective, we have 
$$\udim \tau(P_j/P_i)=\udim I_i-\udim I_j.$$
In particular, we have $\supp\tau(P_j/P_i)\cap P_i=\{i\}$ and it follows that $\im(f)=S_i$. When considering the epimorphism $P_i\to S_i$ and the monomorphism $S_i\to\tau(P_j/P_i)$, the claim is straightforward.

\end{proof}
\begin{lemma}\label{lem:quotients projectives2}
Let $j\in Q_0(i,-)$ and $P_{j,1}/P_i$ not injective\sayT{condition in terms of limbs}. Then there exists a long exact sequence
$$0\to\bigoplus_{\substack{k\in Q_0(i,-)\\k\neq j}}P_k\to P_i\to \tau(P_{j,1}/ P_i)\to\bigoplus_{k\in Q_0(-,i)}I_k\ \to 0$$
inducing two short exact sequences
$$0\to\bigoplus_{\substack{k\in Q_0(i,-)\\k\neq j}}P_k \to P_i\to C_{ij}\to 0,\quad 0\to C_{ij}\to \tau(P_{j,1}/ P_i)\to\bigoplus_{k\in Q_0(-,i)}I_k\to 0$$
where $C_{ij}$ is the middle term of the unique exact sequence $\ses{P_j}{C_{ij}}{S_i}$.
\end{lemma}
\begin{proof}
Consider the (unique) short exact sequence
$$\ses{P_i}{P_{j,1}}{P_{j,1}/P_i}$$
and the induced map $f:P_i\to\tau(\tau^{-1}P_j/P_i)$. We have 
$$\udim \tau(P_{j,1}/P_i)=\udim I_i+\udim P_j.$$
In particular, we have $\supp \tau(P_{j,1}/P_i)\cap P_i=\supp(C_{ij})$ and it follows that $\im(f)=C_{ij}$. When considering the epimorphism $P_i\to C_{ij}$ and the monomorphism $C_{ij}\to\tau(P_{j,1}/P_i)$, the claim is straightforward.

\end{proof}





\begin{lemma}
  \label{lem:irreducible morphisms}
  \mbox{}
  \begin{enumerate}
    \item For $j\in Q_0(-,i)$, an irreducible morphism $(i,j)_m:P_{i,m}\to P_{j,m}$ is surjective if and only if $j$ is a limb with respect to $i$ and $m\geq h(j)$. 
    %either $j$ is a source or if $|Q_0(-,j)|=1$ and $\ell\in Q_0(-,j)$ is a limb vertex. 
    \item For $j\in Q_0(i,-)$, an irreducible morphism $(i,j)^m:P_{i,m}\to P_{j,m+1}$ is surjective if and only if $j$ is a limb with respect to $i$ and $m\geq h(j)$.
    %if $j$ is not a limb. If $j$ is a limb vertex, $(i,j)^m$ is injective if and only if $i$ is a limb vertex with respect to $j$.
  \end{enumerate}
\end{lemma}
\begin{proof}

We consider the case $m=0$ and $i\in Q_0(j,-)$. Then there exists an injective irreducible morphism $(i,j)_0:P_{i}\to P_j$.  Now the claim follows when combining Lemma \ref{lem:indecomposable2} and Lemma \ref{lem:preinjective quotients}.


%If $|Q_0(j,-)|\geq 2$, we can proceed as we did in Proposition \ref{pro:simpleregular}, i.e. we apply a series of reflections to ''transform $C$ into $S_j$'', in order to show that $C$ is preinjective if and only if we have $|Q_0(-,j)|=0$, $|Q_0(j,-)|=2$ and $i\neq \ell\in Q_0(j,-)$ is a limb vertex. In any other case, we have $S_j\cong C$.

%By Proposition \ref{pro:simpleregular}, $S_j$ is preinjective if and only if $|Q_0(-,j)|\leq 1$ and if $\ell\in Q_0(-,j)$, it is a limb vertex with respect to $j$. If $|Q_0(-,j)|=0$, it follows that $P_{i,m}\to P_{j,m}$ is surjective for each $m\geq 1$. In the second case, we have $\tau^{-h(\ell)-1}S_j=0$ and it follows that $P_{i,h(\ell)+1}\to P_{j,h(\ell)+1}$   is surjective. In every other case, $S_j$ is either preprojective or regular and thus we have $\tau^{-m}S_j\neq 0$ for every $m\geq 0$ which means that $P_{i,m}\to P_{j,m}$ is injective for each $m\geq 0$. This shows the first statement.

The second part follows in the same way when considering the irreducible morphism $(i,j)^1:P_{i,0}\to P_{j,1}$.

\end{proof}

\begin{remark}
  Let $i\in Q_0(j,-)$ (resp. $i\in Q_0(-,j)$).
  Note that if $i$ and $j$ are both limbs on the type $A$ subquiver $P_{i,m}\to P_{j,m}$ (resp. $P_{i,m-1}\to P_{j,m}$) becomes surjective for big enough $m$ if and only if $j$ is closer to the leaf of the corresponding subquiver of type $A$.
  The maps $P_{j,m}\to P_{i,m+1}$ (resp. $P_{j,m}\to P_{i,m}$) are always injective if $j$ is a limb with respect to $i$.

If $j$ is not a limb, then all irreducible maps ending in $P_{j,m}$ are injective. If $i$ is a leaf, then all maps starting in $P_{i,m}$ are injective.
\end{remark}

For a fixed pair of vertices $i,j\in Q_0$, there exists a unique (non-oriented) path from $i$ to $j$ in $Q$ which we denote by $p_{i,j}$. The number of arrows passed through in non-opposite direction is denoted by $n_{i,j}$. This path induces sectional morphisms
$$f_{i,j}^m:P_{i,m}\to P_{j,m+n_{i,j}}$$
where we write $P_{i,j}^m:=P_{j,m+n_{i,j}}$ in the following.
If $|p_{i,j}|=n$, we set $i_0=i$ and $i_n=j$ and recursively define $m_0=m$ and 
$$m_l=\begin{cases}m_{l-1}\text{ if $i_{l}\to i_{l-1}$}\\m_{l-1}+1\text{ if $i_{l-1}\to i_l$}\end{cases}$$
for $l=1,\ldots,n-1$. Then we have $$f_{i,j}^m=\prod_{l=0}^{n-1} f_{i_l,i_{l+1}}^{m_l}.$$
As $\Ext(P_{j,m+n_{i,j}},P_{i,m})=0$, the Happel-Ringel Lemma yields that  $f_{i,j}^m$ is either surjective or injective\sayT{more details on vanishing $\Ext$}. 


\begin{corollary}\label{cor:sectional morphisms}
The morphism $f_{i,j}^m$ is surjective if  $i_1$ is a limb with respect to $i$. If $j$ is not a limb, then $f_{i,j}^m$ is injective.\sayT{this is the general case, conditions on $m$?!}
\end{corollary}
\begin{proof} The statement follows from Lemma \ref{lem:irreducible morphisms}. 

If $i_1$ is a limb (then $i$ is not a limb), $f_{i,j}^m$ is forced to be surjective as $f_{i_0,i_1}^m$ is surjective. If $j$ is not a limb, then $f_{i_{n-1},i_n}^{m_{n-1}}$ is injective which means that $f_{i,j}^m$ is forced to be injective. 
\end{proof}

For 
$$\cP_{i,m}=\big(P_{i,k}^m\big)_{k\in N_i},$$
there exists an Auslander-Reiten sequence
$$\Ses{P_{i,m}}{\cP_{i,m}}{P_{i,m+1}}{f}{g}.$$
with $f=(f_{i,k}^m)_{k\in N_i}$ and $g=(g_{k,i}^{m+1})_{k\in N_i}$ where $g^{m+1}_{k,i}=f_{k,i}^m$ if $k\in Q_0(-,i)$ and $g^{m+1}_{k,i}=f_{k,i}^{m+1}$ if $k\in Q_0(i,-)$.

Note that,  we analogously obtain a sectional morphism $g_{i,j}^m:P_{i,m-n_{i,j}}\to P_{j,m}$.




\subsection{Generalized Auslander-Reiten sequences}
Let $Q$ be a quiver whose underlying graph is a tree. The set of vertices decomposes as 
$Q_0=L\coprod V$ where $L$ denotes the set of vertices which are limbs. For a limb $l$ we denote by $b_l$ the corresponding leaf. For a vertex $i\in Q_0$, write 
$N_i$ for the set of neighbors which decomposes into $L_i\coprod V_i$ where $L_i=L\cap N_i$ and $V_i=V\cap N_i$. 

For a path $p$ denoted by $i_0i_1\ldots i_n$, we define $V_p=\{i_0,\ldots,i_n\}$ and 
$$N_p=\left(\bigcup_{i\in V_p}N_i\right)\backslash V_p,$$
i.e. all neighbors of the vertices on the path which do not lie on the path itself.

\begin{lemma}
For every pair of vertices $(i,j)$ of $Q$ and every $m\geq 0$, there exists a short exact sequence
$$E_{i,j}^m:\ses{P_{i,m}}{\bigoplus_{k\in N_{p_{i,j}}}P_{i,k}^m}{P_{i,j}^{m+1}}.$$

\end{lemma}
\begin{proof}We proceed by induction on the length of the path. For $i=j$, we have that $E_{i,i}^m$ is the Auslander-Reiten sequence considered above. Let $p_{i,j}=i_0\ldots i_n$. In the present notation, the Auslander-Reiten sequences can be written as
$$\ses{P_{i,j}^m}{\left(\bigoplus_{k\in N_j\backslash \{i_{n-1}\}}P_{i,k}^m\right)\oplus P_{i,i_{n-1}}^{m+1}}{P_{i,j}^{m+1}}.$$
As $j\in N_{i_{n-1}}$, $P_{i,j}^m$ appears as a direct summand of the middle term of the exact sequence
$$E_{i,i_{n-1}}^m:\ses{P_{i,m}}{\bigoplus_{k\in N_{p_{i,i_{n-1}}}}P_{i,k}^m}{P_{i,i_{n-1}}^{m+1}},$$
which exists by induction hypothesis, the second part of Corollary \ref{cor:sequence reducing} gives the claim.
\end{proof}



\begin{corollary}\sayT{we should try to make one statement out of the first two}\label{cor:injsur} \begin{enumerate}\item Let $j\in Q_0(-,i)$. For $m\geq 1$, the map
$$\bigoplus_{k\in Q_0(i,-)}P_{k,m}\oplus \bigoplus_{\substack{k\in Q_0(-,i)\\k\neq j}}P_{k,m-1}\to P_{i,m}$$ is injective if $j$ is not a limb with respect to $i$.
For $m\geq h(j)$, the map is surjective if $j$ is a limb with respect to $i$.
\item Let $j\in Q_0(i,-)$. For $m\geq 1$, the map
$$\bigoplus_{k\in Q_0(i,-)}P_{k,m}\oplus \bigoplus_{\substack{k\in Q_0(-,i)\\k\neq j}}P_{k,m-1}\to P_{i,m}$$ is injective if $j$ is not a limb with respect to $i$.
For $m\geq h(j)+1$, the map is surjective if $j$ is a limb with respect to $i$.
\item Let  $i\in Q_0$ and $I\subset N_i$ with $|I|\leq |N_i|-2$. Then the map
$$\bigoplus_{k\in I}P_{k,m_k}\to P_{i,m}$$
is injective where $m_k=m$ if $k\in Q_0(i,-)$ and $m_k=m-1$ if $k\in Q_0(-,i)$.
\end{enumerate}
\end{corollary}
\begin{proof}
The first two parts of the statement follow directly from Lemmas \ref{lem:preinjective quotients}, \ref{lem:quotients projectives} and \ref{lem:quotients projectives2}.

We may assume that $I=N_{i}\backslash\{j_1,j_2\}$.
\end{proof}


\begin{lemma}Let $i\in Q_0$ be not a limb. The pair $(E_{i,j}^m,I)$ with $I\neq\emptyset$ is reduced if $|I|\leq |N_{p_{i,j}}|-2$ or if $I=|N_{p_{i,j}}|-1$ and $q\in N_{p_{i,j}}\backslash I$ is no limb with respect to $i$. 

\end{lemma}      
\begin{proof}We proceed by induction on the length of $p_{ij}$. If $i=j$,

The pair $(E_{i,j}^m,I)$ is reduced if $\sum_{k\in I}\udim P_{i,k}^m<\udim P_{i,j}^{m+1}$ which is if and only if 
$$\udim P_{i,m}< \sum_{k\in N_{p_{i,j}}\backslash I}\udim P_{i,k}^m.$$
If $|I|=|N_{p_{i,j}}|-1|$, this means that the morphism $P_{i,m}\to P_{q,m}$ is surjective which is the case if and only if $q$ is a limb by Corollary \ref{cor:sectional morphisms}.
\sayT{to be continued}
\end{proof}
\begin{comment}
\begin{lemma}\begin{enumerate}
\item If $j$ is not a leaf, there exists a vertex $i$ such that the exact sequence $E_{i,j}^m$ is reduced.
\item If $j$ is a leaf, every representation $P_{j}^m$ appears as an appropriate quotient.
\end{enumerate}
\end{lemma}
\end{comment}
Assume that $(E_{i,j}^m,I)$ is reduced with $I=\{k_1,\ldots,k_{r+1}\}$. This induces a series of short exact sequences and a filtration of $P_{ij}^{m+1}$:
\begin{equation}\label{ses}\ses{P_{i,k_l}^m}{P_{i,j}^{m,I_{l-1}}}{P_{i,j}^{m,I_l}}\end{equation}
where $I_l=\{k_1,\ldots,k_l\}$. By Lemma \ref{le:quotients}, we have
$$P_{i,j}^{m,I_{r}}\cong P_{i,k_{r+1}}^m/P_{i,j}^m.$$
If we choose $k_{r+1}$ in such a way that $j\in N_{p_{i,k_{r+1}}}$, this shows that $P_{i,j}^{m,I}$ appears as a quotient of induced by a filtration of $P_{i,k_{r+1}}^m\cong P_{k_{r+1},m'}$ for an appropriate $m'$. Here we can take any path $p$ ending in $k_{r+1}$ with $j\in N_p$ which clearly exists as $j\neq k_{r+1}$.


Every short exact sequence (\ref{ses}) induces a morphism $P_{i,k_l}^m\to \tau  P_{i,j}^{m,I_l}$ and its $\tau^{-1}$-translate which we are interested in. In particular, we need to know its kernel and its image.


\begin{proposition}\label{pro:cell decompositions}
Let $(E_{i,j}^m,I)$ be a reduced with $I=\{k_1,\ldots,k_{r+1}\}$ and let $k_l=k_l^1,\ldots,k_l^{t_l}=j$ be the vertices on the path $p_{k_l,j}$. Then $P_{ij}^{m+1}$ has property (C) if $P_{i,k_l}^m$, $P^{m,I_{r}}_{i,j}$, $P_{i,k^{s}_l}^m/P_{i,k^{s-1}_l}^m$ and its $\tau$-translates have property (C) for $l=1,\ldots,r$ and $s=1,\ldots,t_l$ where we set $P_{i,k^{0}_l}^m=0$.\sayT{the vertices on the path from $k_l$ to $j$ are certain vertices on the path from $i\to j$}
\end{proposition}
\begin{proof}\sayT{this is the general strategy and roughly the induction step, there are assumptions missing, e.g. we need $\tau$-stability (induction basis)}
For each $l$, there exists a chain of irreducible monomorphisms \say{to have monos later here we need assumptions on the reps}
$$P^m_{i,k^0_l}\subset P_{i,k^1_l}^m\subset\ldots\subset P_{i,k_l^{t_l}}^m\subset P_{i,j}^{m+1}.$$
If we write $V_u=\bigoplus_{l=u}^rP_{i,k_l}^m$ for $u=2,\ldots, r$, we may consider the exact sequences\say{make a choice such that the cokernel is not zero for $l=1$ and $s=t_l$?!}\sayT{include modified version of 8.7}
$$\ses{P_{i,k_l^s}/P_{i,k_l^{s-1}}}{P_{i,j}^{m+1}/(V_{l}\oplus P_{i,k_l^{s-1}}^m)}{P_{i,j}^{m+1}/(V_l\oplus P_{i,k_l^{s}}^m)}$$
for $l=1,\ldots r$ and $s=t_l,\ldots,1$. Note that we have $P_{i,j}^{m+1}/V_1=P^{m,I_{r}}_{i,j}.$

Consider $\psi_{s,l}:P_{i,k_l^s}/P_{i,k_l^{s-1}}\to \tau (P_{i,j}^{m+1}/(V_l\oplus P_{i,k_l^{s}}^m))$. By Lemma \ref{lem:mochain2}, we have 
$$\ker(\psi_{s,l})=\tau(N(P_{i,k_l^s})),\quad\im(\psi_{s,l})=\tau(P_{i,k_l^{s+1}}/P_{i,k_l^{s}})$$
where $N(P_{i,k_l^s})$ are direct summands of the middle terms of the Auslander-Reiten sequence 
$$\ses{P_{i,k_l^s}}{\tau^{-1}P_{i,k_l^{s-1}}\oplus N(P_{i,k_l^s})\oplus P_{i,k_l^{s+1}}}{\tau^{-1}P_{i,k_l^s}}$$
and where $P_{i,k_l^{s+1}}=P_{i,j}^{m+1}$.

By induction, we see that $(P_{i,k_l^{s+1}}/P_{i,k_l^{s}},P_{i,j}^{m+1}/(V_l\oplus P_{i,k_l^{s}}^m))$ is a compatible pair.
Finally, $(N(P_{i,k_l^s}),P_{i,k_l^s})$ is also compatible by assumption.\sayT{why can we assume this?}.


\end{proof}
\section{Results on bipartite quivers}
  \noindent In this section, we consider truncated preprojective representations of those quivers whose underlying graph is a tree.
  We first consider bipartite quivers, i.e. every vertex is either sink or source.
  Actually, it will turn out that the general situation can be reduced to this case.

  Let $Q$ be a bipartite tree quiver with sources $\cT$ and sinks $\cH$.
  For a source $t\in\cT$, the corresponding projective representation $P_t$ (resp. injective representation $I_t$) is the exceptional representation with dimension vector $t+\sum_{h\in Q_0(t,-)}h$ (resp. $t$).
  Dually, the projective (resp. injective) representation of a sink $h\in\cH$ has dimension vector $h$ (resp. $h+\sum_{t\in Q_0(-,h)} t$).

For every $h\in Q_0(t,-)$, there exists a unique irreducible morphism $f_{h,t}^m:P_{h,m}\to P_{t,m}$ and the corresponding Auslander-Reiten sequence
$$\ses{P_{t,m-1}}{\bigoplus_{h\in Q_0(t,-)}P_{h,m}}{P_{t,m}}.$$ 

Dually, for every $t\in Q_0(-,h)$, there exists a unique irreducible morphism $f_{t,h}^m:P_{t,m}\to P_{h,m+1}$ and the corresponding Auslander-Reiten sequence
$$\ses{P_{h,m}}{\bigoplus_{t\in Q_0(-,h)}P_{t,m}}{P_{h,m+1}}.$$

In turn, for every $H\subset Q_0(t,-)$ we obtain a morphism
$f_{t,H}^m:\bigoplus_{h\in H}P_{h,m}\to P_{t,m}$ and for every $T\subset Q_0(-,h)$ we obtain a morphism
$f_{h,T}^m:\bigoplus_{t\in T}P_{h,m}\to P_{t,m}.$ Up to scalars, these morphisms only depend on $t$ and $H$ (resp. $h$ and $T$). 

If $f_{t,H}^m$ (resp. $f_{h,T}^m$) is injective, we define
$$P_{t,H}^m=\mathrm{coker}f_{t,H}^m,\quad P_{h,T}^m=\mathrm{coker}f_{h,T}^m.$$  
If $f_{t,H}^m$ (resp. $f_{h,T}^m$) is surjective, we define
$$P_{t,H}^m=\mathrm{ker}f_{t,H}^m,\quad P_{h,T}^m=\mathrm{ker}f_{h,T}^m.$$  

\begin{lemma}\label{lem:truncated1}        
Let $Q_0(t,-)=\{h_1,\ldots,h_n\}$ and let $H_i=\{h_1,\ldots,h_i\}$. 
\begin{enumerate} 
\item If both $P^m_{t,H_i}$ and $P^m_{t,H_{i+1}}$ are cokernels, we obtain a unique map
$$P_{t,H_i}^m\twoheadrightarrow P_{t,H_{i+1}}^m$$
and $P^m_{t,H_{i}}\in (P^m_{t,H_{i+1}})^{\perp}$.
\item If $P^m_{t,H_i}$ is a cokernel and $P^m_{t,H_{i+1}}$ is a kernel, we have $\Ext(P^m_{t,H_{i+1}},P^m_{t,H_i})=\kk$ and $P^m_{t,H_{i+1}}\in (P^m_{t,H_i})^{\perp}$.

\item If both $P^m_{t,H_i}$ and $P^m_{t,H_{i+1}}$ are kernels, we have a unique map $$P_{t,H_i}^m\hookrightarrow P_{t,H_{i+1}}^m$$
 and $P^m_{t,H_{i}}\in (P^m_{t,H_{i+1}})^{\perp}$.
\end{enumerate}

The dual version also holds.
\end{lemma}


We call the representation $P_{t,H}^m$ and $P_{T,h}^m$ truncated preprojective in the following.\sayT{we should discuss whether we want certain preinjectives and preprojective to be t.proj. as well. Maybe this makes the definition easier?!}

Lemma \ref{lem:indecomposable2} immediately gives the following.
\begin{lemma}\label{lem:truncated2}\begin{enumerate}
\item
If $P_{t,H}^m$ is neither preprojective nor injective, $f_{t,H}^m$ is injective and we obtain a short exact sequence
$$\ses{\bigoplus_{h\in H}P_{h,m+1}}{P_{t,m+1}}{P_{t,H}^{m+1}}$$
with $P_{t,H}^{m+1}=\tau^{-1}P_{t,H}^m$.
\item
If $P_{t,H}^m$ is a preprojective representation, $\tau^{-1}$ preserves injectivity and surjectivity of $f_{t,H}^m$ and $P_{t,H}^{m+1}=\tau^{-1}P_{t,H}^m$.
\item
If $P_{t,H}^m$ is an injective representation, say $I_q$ for $q\in Q_0$, the map $f_{t,H}^m$ is injective, and we obtain a short exact sequence 
$$\ses{P_{t,H}^{m+1}}{\bigoplus_{h\in H}P_{h,m+1}}{P_{t,m+1}}$$
with $P_{t,H}^{m+1}=P_q$.
\end{enumerate}

The dual statement also holds.\sayT{make it explicit}
\end{lemma}

The next step is to analyze the truncated preprojective representations and to give criteria which help to decide whether they are prepeprojective, regular or preinjective.

\begin{lemma}\label{lem:truncated3}Let $H\subset Q_0(t,-)$ and $T\subset Q_0(-,h)$ for fixed $t\in\cT$ and $h\in\cH $.
\begin{enumerate} 
\item $P_{t,H}^0$ is preinjective if $H=Q_0(t,-)$ or if $|H|=|Q_0(t,-)|-1\neq 0$ and, moreover, the unique vertex $h\in Q_0(t,-)\backslash\{H\}$ is a limb vertex.
\item $P_{t,H}^0$ is preprojective if $H=\emptyset$ or if $|H|=1\neq |Q_0(t,-)| $ and, moreover, the unique vertex $h\in H$ is a limb vertex.
\item  $P_{h,T}^0$ is preinjective if and only if $0\neq |T|=|Q_0(h,-)|-1$ and, moreover, the unique vertex $t\in Q_0(h,-)\backslash T$ is a limb vertex.
\item $P_{h,T}^0$ is preprojective if $|T|=0$ or $|T|=1\neq |Q_0(-,h)|$ and, moreover, the unique vertex $t\in T$ is a limb vertex.
\end{enumerate}

\end{lemma}
\begin{proof}
The results follow by Proposition \ref{pro:simpleregular} and Corollary \ref{cor:simpleregular}. Note that $P_{h,T}^0$ is an exceptional representation such that
$$\udim\tau\left(P_{h,T}^0\right)=h+\sum_{t\in T}t.$$

Moreover, for $T\subsetneq Q_0(-,h)$, we have 
$$\udim P_{h,T}^0=\sum_{\substack{t\in Q_0(-,h)\\t\notin T}}\left(t+\sum_{h'\in Q_0(t,-)}h'\right)-h.$$






\end{proof}


\begin{proposition}\label{pro:truncated}Let $t\in\cT$ and $H\subsetneq  Q_0(t,-)$.
Assume that 
$$\bigoplus_{h\in H} P_{h,m}\to P_{t,m}$$ is surjective. Then $|H|=|Q_0(t,-)|-1$, the unique vertex $h'\in Q_0(t,-)\backslash\{H\}$ is a limb vertex and we have $P_{t,H}^m=P_{\ell',m-h(h')}$ where $\ell'\in Q_0(-,h')$ with $\ell'\neq t$. Finally, we obtain a short exact sequence
$$\ses{P_{\ell',m-h(h')}}{P^m_{t,m}}{P_{h',m}}$$
with preprojective kernel.
\end{proposition}
\begin{proof}
For $m=0$, the maps $f_{t,H}$ are all injective and the cokernel is preinjective if and only if $H=Q_0(t,-)$ or if $H=Q_0(t,-)\backslash\{h'\}$ for some $h'\in Q_0(t,-)$ and $h'$ is a limb vertex. Thus Lemma \ref{lem:truncated2} gives the first part of the Proposition. As  $\Ext(P_{h,m},P_{t,m})=0$ for $h\in Q_0(t,-)$, applying $\Hom(\bigoplus_{h\in H}P_{h,m},-)$ to the Auslander-Reiten sequence between $P_{t,m}$ and $P_{t,m+1}$ we obtain the commutative diagram

  \[\xymatrix{
      & & 0 \ar[d] & 0 \ar[d] \\
      & & P^m_{t,H} \ar[d]\ar@{=}[r] & P_{t,H}^m \ar[d] \\
      0 \ar[r] & P_{t,m} \ar[r]\ar@{=}[d] & \bigoplus_{h\in H}P_{h,m}\oplus P_{t,m} \ar[d]^\vartheta\ar[r] &  \bigoplus_{h\in H}P_{h,m} \ar[r]\ar[d] & 0\\
      0 \ar[r] & P_{t,m} \ar[r] & \bigoplus_{h\in Q_0(t,-)}P_{h,m}\ar[r]\ar[d] &  P_{t,m+1}\ar[r]\ar[d] & 0 \\
      & & 0 & 0}\]
			
			As the representations $P_{h,m}$ with $h\in Q_0(t,-)$ are pairwise orthogonal and as $\vartheta$ is surjective, it restricts to an isomorphism on $\bigoplus_{h\in H}P_{h,m}$. It follows that, we obtain a short exact sequence
			$$\ses{P_{t,H}^m}{P_{t,m}}{P_{h',m}}.$$
			As we have $\Hom(P_{t,m},P_{h',m})=\kk$, we can combine Corollary \ref{cor:thin sequences} with Lemma \ref{lem:indecomposable} and obtain $P_{t,H}^m=P_{\ell,m-h(h')}$.\sayT{here we need that $m\geq h(h')$; for $m<h(h')$, the horizontal maps are not surjective}
\end{proof}


\begin{remark}
Note that the only possibility that $\bigoplus_{h\in H}P_{h,m}\to P_{t,m}$ is surjective, is that $h'\in Q_0(t,-)$ is a limb vertex and $|H|=|Q_0(t,-)-1|$. In this case, there exists an $m'<m$ such that there exists a short exact sequence
$$\ses{\bigoplus_{h\in H}P_{h,m'}}{P_{t,m'}}{P_{t,H}^{m'}}$$
with injective cokernel.
\end{remark}

\begin{theorem}
Let $t\in \cT$. Let $Q_0(t,-)=\{h_1,\ldots,h_n\}$ and write $H_i:=\{h_1,\ldots,h_i\}$
\begin{enumerate}\item If each $h\in Q_0(t,-)$ is not a limb vertex, for every $m\geq 1$, there exists chains of  non-preinjective exceptional representations
$$P_{t,H_1}^m\twoheadrightarrow\ldots\twoheadrightarrow P_{t,H_{n-1}}^m.$$
Furthermore, we have $$P_{t,H_{n-1}}^m\cong P_{h_n,\{t\}}^{m-1}.$$  \sayT{the same applies when $h_n$ is not a limb vertex. But maybe it is better to organize all t.projectives?!}


\item Assume that $h_{i_1},\ldots,h_{i_r}$ are limb vertices. For every $m\geq 1$, there exists chains of regular representations
$$P_{t,H_1}^m\twoheadrightarrow\ldots\twoheadrightarrow P_{t,H_{n-2}}^m.$$

Furthermore, we have $$P_{t,H_{n-1}}^m\cong P_{h_n,\{t\}}^{m-1}.$$
 
\end{enumerate}

\end{theorem}
\begin{proof}
The statements of this section show that for all $H\subsetneq Q_0(t,-)$, the maps $\oplus_{h\in H}P_{h,m}\to P_{t,m}$ are injective. We obtain a commutative diagram
  \[\xymatrix{
      & & 0 \ar[d] & 0 \ar[d] \\
      & & \bigoplus_{h\in H}P_{h,m} \ar[d]\ar@{=}[r] &\bigoplus_{h\in H}P_{h,m} \ar[d] \\
      0 \ar[r] & P_{t,m-1} \ar[r]\ar@{=}[d] & \bigoplus_{h\in Q_0(t,-)}P_{h,m} \ar[d]\ar[r] & P_{t,m} \ar[r]\ar[d] & 0\\
      0 \ar[r] & P_{t,m-1} \ar[r] &\bigoplus_{h\in Q_0(t,-)\backslash H}P_{h,m}\ar[r]\ar[d] &  P_{t,H}^m\ar[r]\ar[d] & 0 \\
      & & 0 & 0}\]
If $r<n$ the second part holds when choosing $h_n$ to be not a limb. Thus assume that $r=n$, i.e. all neighbors of $t$ are limb vertices.
			
\end{proof}


\begin{comment}
\begin{definition}
\end{definition}

Write $P_{q,m}:=\tau^{-m} P_q$ for $q\in Q_0$. Then we have 

$$\udim P_{h,1}=|Q_0(h,-)-1|h+\sum_{t\in Q_0(h,-)}\left(t+\sum_{h'\in Q_0(-,t)}h'\right).$$
\end{comment}




\section{Appendix: Homological Lemmas}

If $A\oplus B$ is a direct sum, we write $\iota_A:A\into A\oplus B$ and $\pi_A:A\oplus B\onto A$ for the natural embedding and the natural projection. 

\begin{lemma}
  \label{le:sequence reducing}
  Suppose there is an exact sequence 
  \[\ses{A}{X\oplus B}{C}\]
  for which the induced map $f:A\to B$ given by composing the left hand morphism with the natural projection $\pi_B$ is surjective.
  Then there is an induced short exact sequence 
  \[\ses{K}{X}{C},\]
  where $K=\ker(f)$.
\end{lemma}
\begin{proof}
  Consider the following exact commutative diagram:
  \[\xymatrix{
      & 0 \ar[d] & 0 \ar[d] & \\
      0 \ar[r] & K \ar[d]\ar[r]\ar@{}[dr]|(.3){\ulcorner} & X \ar[d] & & \\
      0 \ar[r] & A \ar[d]\ar[r] & X\oplus B \ar[d]\ar[r] & C \ar[r] & 0\\
      & B \ar@{=}[r]\ar[d] & B \ar[d] & \\
      & 0 & 0 & }\]
  The middle vertical sequence is naturally split which induces the upper horizontal morphism and shows that it is injective.
  Since the lower horizontal morphism is an equality, the upper left square is a pushout.
  It follows that the cokernel of the upper horizontal morphism coincides with the cokernel in the middle horizontal sequence, giving the claim.
\end{proof}

\begin{corollary}
  \label{cor:sequence reducing}
  Suppose there is an exact sequence
  \[\sesM{A}{B\oplus X}{C}{f}\].
  \begin{enumerate}
    \item If there is an exact sequence $\sesm{K}{A\oplus D}{X}{g}$ with $\pi_X\circ f=g\circ\iota_A$, then there is an induced short exact sequence $\ses{K}{B\oplus D}{C}.$
    \item If there is an exact sequence $\ses{X}{C\oplus D}{K}$, then there is an induced short exact sequence $\ses{A}{B\oplus D}{K}.$
  \end{enumerate}
\end{corollary}
\begin{proof}
  Due to duality, we only prove the first part.
  Consider the following exact commutative diagram with split vertical sequences induced by the given exact sequences:
  \[\xymatrix{
      & 0 \ar[d] & 0 \ar[d] & \\
      0 \ar[r] & A \ar^{\iota_A}[d]\ar^f[r]\ar@{}[dr]|(.2){\ulcorner}& X\oplus B \ar^{\iota_{X\oplus B}}[d]\ar[r] &C\ar@{=}[d]\ar[r] &0 \\
      0 \ar[r] & A\oplus D \ar[d]\ar^{\scriptsize{\begin{pmatrix}g\\0\\\pi_D\end{pmatrix}}}[r] & X\oplus B\oplus D \ar[d]\ar[r] & C \ar[r] & 0\\
      & D \ar@{=}[r]\ar[d] & D \ar[d] & \\
      & 0 & 0 & }\]
 As $g$ is surjective, we can apply Lemma \ref{le:sequence reducing}.
\end{proof}
\begin{definition}
  A pair $(e,I)$ consisting of 
  \begin{itemize}
    \item a short exact sequence
      $$e:\sesm{X}{\bigoplus\limits_{i=1}^rB_i}{Y}{g}$$
      with indecomposable representations $X$, $Y$, and $B_i$ for $i=1,\ldots,r$;
    \item a subset $\emptyset\neq I\subset\{1,\ldots,r\}$ with $|I|\leq r-1$;
  \end{itemize}
  is called \emph{reduced} if the induced morphism $g\circ\iota_{B_I}:B_I\to Y$ is injective where we write $B_I:=\oplus_{i\in I}B_i$.

  If $(e,I)$ is reduced for every $\emptyset\neq I\subset \{1,\ldots,r\}$ with $|I|\le r-1$, we call the sequence $e$ \emph{reduced}.
\end{definition}

\begin{lemma}\label{le:quotients}
  Let $(\ses{X}{\bigoplus_{i=1}^rB_i}{Y},I)$ be reduced with $\{1,\ldots,r\}\setminus I=J$.
  Then we have $Y/B_I\cong B_J/X$.
\end{lemma}
\begin{proof}
  It is verified straightforwardly that we obtain the following commutative diagram with the natural vertical morphisms:
  \[\xymatrix{
      & & 0 \ar[d] & 0 \ar[d] \\
      & & B_I \ar[d]\ar@{=}[r] & B_I\ar[d] \\
      0 \ar[r] & X \ar[r]\ar@{=}[d] & \bigoplus\limits_{i=1}^rB_i \ar[d]\ar[r] & Y \ar[r]\ar[d] & 0\\
      0 \ar[r] & X \ar[r] & B_J\ar[r]\ar[d] & Y/B_I\ar[r]\ar[d] & 0 \\
      & & 0 & 0}\]
\end{proof}
\begin{comment}
\begin{comment}
<<<<<<< HEAD

 


 
\begin{lemma}\label{lem:mochain}
Let $0=M_0\subset M_1\subset M_2\subset \ldots \subset M_r$ be a $\tau$-stable chain of representations such that for $j=1,\ldots,r-1$ the induced map $\phi_{0,j,j+1}:M_j\to \tau M_{j+1}/M_j$ is surjective and such that $\ker(\phi_{0,j,j+1})=M_{j-1}\oplus K_j$ for some representation $K_j$. For $0\leq i<j<k\leq r$, we have $\Im(\phi_{i,j,k})=\tau (M_{j+1}/M_j)$, $\ker(\phi_{i,j,k})= M_{j-1}/M_{i}\oplus K_{j}$ and $\mathrm{coker}(\phi_{i,j,k})=\tau M_k/M_{j+1}$.
\end{lemma}
\begin{proof}
As $\ker(\phi_{0,j,j+1})=M_{j-1}\oplus K_{j}$ for some representation $K_{j}$, there exists a short exact sequence
$$\ses{K_{j}}{M_j/M_{j-1}}{\tau (M_{j+1}/M_j)}.$$
Taking into account that we have a short exact sequence
$$\ses{M_{j-1}/M_i}{M_j/M_i}{M_j/M_{j-1}},$$ we obtain a short exact sequence $$\ses{M_{j-1}/M_i\oplus K_j}{ M_j/M_i}{\tau( M_{j+1}/M_j)}.$$ As we have an exact sequence
$$\ses{\tau (M_{j+1}/M_j)}{\tau(M_k/M_j)}{\tau(M_k/M_{j+1})},$$
 we end up with an exact sequence
$$0\rightarrow M_{j-1}/M_{i}\oplus K_{j}\to M_{j}/M_i\xrightarrow{\phi_{i,j,k}}\tau( M_k/M_j)\to\tau (M_k/M_{j+1})\to 0.$$\sayT{check why the induced morphism is $\phi_{i,j,k}$. Uniqueness?}
In particular, we have $\Im(\phi_{i,j,k})=\tau (M_{j+1}/M_j)$, $\ker(\phi_{i,j,k})= M_{j-1}/M_{i}\oplus K_{j}$ and $\mathrm{coker}(\phi_{i,j,k})=\tau (M_k/M_{j+1})$.
=======
\end{comment}

\begin{definition}A \emph{filtration} $\cF(M)$ of a representation  $M$ is given by a chain of representations $$0=M_0\subset M_1\subset M_2\subset \ldots \subset M_r=M.$$
It is called \emph{irreducible} if the corresponding monomorphisms are irreducible. 
It is called \emph{orthogonal} if all representation $M_i$ are exceptional and if we have $\Hom(M_i,M_j)=\kk$, $\Ext(M_i,M_j)=0$ and $M_i\in M_j^{\perp}$  for $1\le i<j\le r$.
It is called $\tau$-\emph{stable} (resp. $\tau^{-1}$-stable) if the $\tau$-translates ( resp. $\tau^{-1}$-translates) of $M_i$ define a filtration of $\tau M$ (resp. $\tau^{-1}M$).
\say{better names}
 
\end{definition}
Let $\cF(M)$ be a $\tau$-stable filtration of $M$. For $0\leq i< j< k\leq r$, we have $\tau$-stable short exact sequences
$$\ses{M_j/M_i}{M_k/M_i}{M_k/M_j}$$ and thus Auslander-Reiten theory induces morphisms $\phi_{i,j,k}:M_j/M_i\to\tau (M_k/M_j)$ and $\tau^{-1}\phi_{i,j,k}:\tau^{-1}M_j/M_i\to M_k/M_j$.




 
\begin{lemma}
  \label{lem:mochain}
  Let $0=M_0\subset M_1\subset M_2\subset \ldots \subset M_r=M$ be an orthogonal $\tau$-stable filtration of $M$ such that the induced map $\phi_{0,j,j+1}:M_j\to \tau (M_{j+1}/M_j)$ is surjective and such that $\ker(\phi_{0,j,j+1})=M_{j-1}\oplus K_j$ for some representations $K_j$.
  Then for $0\le i<j<k\le r$, we have 
  \[\im(\phi_{i,j,k})=\tau M_{j+1}/M_j,\qquad \ker(\phi_{i,j,k})= M_{j-1}/M_{i}\oplus K_{j},\qquad \mathrm{coker}(\phi_{i,j,k})=\tau M_k/M_{j+1}.\]
\end{lemma}
\begin{proof}
As the filtration is orthogonal, we have $\Hom(M_j/M_i,M_j)=0$ for $i\le j$. Applying $\Hom(-,M_j)$ to the exact sequence induced by $M_i\subset M_j$, we thus see that $\Ext(M_j/M_i,M_j)=0$. Now applying $\Hom(M_j/M_i,-)$ to the same sequence, we obtain $\kk=\Hom(M_j/M_i,M_j/M_i)\cong \Ext(M_j/M_i,M_i)$ and $\Ext(M_j/M_i,M_j/M_i)=0$. Note that applying $\Hom(M_j,-)$ and then $\Hom(-,M_j/M_i)$ to the same sequence we immediately see that
$$\kk=\Hom(M_j,M_j)\cong\Hom(M_j,M_j/M_i)\cong\Hom(M_j/M_i,M_j/M_i).$$
In particular, it follows that $\Hom(M_i,\tau(M_j/M_i))=\kk$. 


Furthermore, applying $\Hom(M_k/M_j,-)$ to the induced sequence 
$$\ses{M_j/M_i}{M_k/M_i}{M_k/M_j},$$ keeping in mind that all quotients are exceptional and have trivial endomorphism ring, we obtain an exact sequence
$$0\to\kk=\Hom(M_k/M_j,M_k/M_j)\to\Ext(M_k/M_j,M_j/M_i)\to \Ext(M_k/M_j,M_j/M_i)\to 0.$$
 It is straightforward that we have $\Hom(M_j/M_i,M_k/M_i)=\kk$. Thus applying $\Hom(-,M_k/M_i)$ to the same sequence yields $\Ext(M_k/M_j,M_k/M_i)=0$ and thus $\Ext(M_k/M_j,M_j/M_i)=\kk$. In particular, it follows that $\Hom(M_j/M_i,\tau(M_k/M_j))=\kk$.
\sayT{easier proof?}


  There is a commutative diagram
  \[\xymatrix{
      & 0 \ar[d] & 0 \ar[d] & \\
      & M_i \ar[d]\ar@{=}[r] & M_i\ar[d] & \\
      0 \ar[r] & M_{j-1}\oplus K_j \ar[r]\ar[d] & M_j \ar[d]\ar[r] & \tau(M_{j+1}/M_j) \ar[r]\ar[d] & 0\\
      0 \ar[r] & (M_{j-1}/M_i)\oplus K_j \ar[r]\ar[d] & M_j/M_i \ar[r]\ar[d] & X \ar[r] & 0 \\
      & 0 & 0 & }\]
  in which an easy diagram chase shows that the right hand morphism is an isomorphism.
  As the exact sequence
	$$\ses{M_{j+1}/M_j}{M_k/M_j}{M_k/M_{j+1}}$$
  is $\tau$-stable and as $\Hom(M_j/M_i,\tau(M_k/M_j))=\kk=\kk$, we end up with an exact sequence
  $$0\rightarrow M_{j-1}/M_{i}\oplus K_{j}\to M_{j}/M_i\xrightarrow{\phi_{i,j,k}}\tau(M_k/M_j)\to\tau(M_k/M_{j+1})\to 0.$$
  \sayD{I see that there is an induced map $M_j/M_i\to\tau(M_k/M_i)$ but why does this have to be the same as $\phi_{i,j,k}$?}
	\sayT{I missed out some conditions on the chain, nevertheless I think this lemma holds in a more general setting}
  In particular, we have
  \[\im(\phi_{i,j,k})=\tau M_{j+1}/M_j,\qquad \ker(\phi_{i,j,k})= M_{j-1}/M_{i}\oplus K_{j},\qquad \mathrm{coker}(\phi_{i,j,k})=\tau M_k/M_{j+1}.\]

\end{proof}

We mostly deal with the case when $M_1\xrightarrow{f_1} M_2\xrightarrow{f_2}\ldots\xrightarrow{f_{r-1}} M_r$ is a chain of irreducible monomorphisms between exceptional representations $M_i$ of a tree quiver which means that the chain is already orthogonal.\sayT{double-check if this is always this true? In the preprojective component it is}. Then the maps $\phi_{i,j,k}$ are unique up to scalars. 

Furthermore, we have Auslander-Reiten sequences
$$\ses{M_i}{ \tau^{-1}M_{i-1}\oplus\bigoplus_{j=1}^{l_i}N_{i,j}\oplus M_{i+1}}{\tau^{-1}M_i}$$ with $M_{r+1}:=0$ and indecomposables $N_{i,j}$. Write $N_i=\bigoplus_{j=1}^{l_i}N_{i,j}$. 
\begin{lemma}\label{lem:mochain2}
Let $\cF(M)$ be an irreducible orthogonal $\tau$-stable filtration. We have:
\begin{enumerate}
\item The induced map $\phi_i:M_i\to \tau(M_{i+1}/M_i)$ is surjective with $\ker(\phi_i)=M_{i-1}\oplus \tau N_i$.
\item Let $V\subset M_r$ be an exceptional\sayT{is probably not necessary} subrepresentation with $\Hom(V,M_r)=\kk$ and assume that there exists $j\in\{1,\ldots,r-1\}$ such that $V\cap M_{j}=\{0\}$ and such that $(V,M_i)$ is orthogonal for each $1\le i\le j$. Then for each $1\leq i\leq j$ there exist unique (up to scalars) non-zero induced maps $\psi_i:M_i/M_{i-1}\to\tau (M_r/(V\oplus M_{i}))$, for which we have $\im(\psi_i)=\tau(M_{i+1}/M_i)$ and $\ker(\psi_i)=\tau(N_i)$ if $i<j$. If $i=j\leq r-2$, the map $\psi_j$ is surjective with $\ker(\psi_j)=\tau(V\cap M_{j+1})\oplus\tau(N_{j})$.  
%Furthermore, $\psi_r$ is surjective with $\ker(\psi_r)= \tau N_r$.
\end{enumerate}


\end{lemma}
\begin{proof}

  There is a commutative diagram where the middle exact sequence is an Auslander-Reiten sequence
  \[\xymatrix{
      &  & 0 \ar[d] &0 \ar[d] \\
      &  0\ar[d]\ar[r]& K\ar[d]\ar@{=}[r]  &K\ar[d]\ar[r]&0 \\
      0 \ar[r] & \tau M_{i}\ar[r]\ar@{=}[d] & M_{i-1}\oplus\bigoplus_{j=1}^{l_j}\tau N_{i,j}\oplus \tau M_{i+1} \ar[d]\ar[r] & M_i \ar[r]\ar[d] & 0\\
      0 \ar[r] & \tau M_i \ar[r]\ar[d] & \tau M_{i+1} \ar[r] & \tau(M_{i+1}/M_i) \ar[r] & 0 \\
      & 0 &  & }\]
As $\tau M_{i+1}$ is a direct summand of the middle term of the Auslander-Reiten sequence, the middle vertical morphism is surjective and thus the vertical morphism on the right hand side is also surjective. As we obviously have
$K=M_{i-1}\oplus \tau N_i$, the first part follows by Lemma \ref{lem:mochain}. 


 As $V\cap M_{j}=\{0\}$, for $1\leq i\leq j$ there is an exact sequence
$$e_i:\ses{M_i}{M_r/V}{M_r/(V\oplus M_i)}.$$ 
Applying $\Hom(-,M_i/M_{i-1})$, the assumptions yield $\Hom(M_r/V,M_i/M_{i-1})=0$ and $\Ext(M_r/V,M_i/M_{i-1})=0$. Thus we obtain
$$\kk=\Hom(M_i,M_i/M_{i-1})\cong \Ext(M_r/(V\oplus M_i),M_i/M_{i-1}).$$

Thus there are unique maps $\psi_i:M_i/M_{i-1}\to\tau (M_r/(V\oplus M_{i}))$ and $\tau^{-1}\psi_i$ which are induced by the short exact sequence
$$\ses{M_{i}/M_{i-1}}{M_r/(V\oplus M_{i-1})}{M_r/(V\oplus M_{i})}.$$
Together with the natural short exact sequence
$$\sesm{N_{i}\oplus \tau^{-1}M_{i-1}}{\tau^{-1}M_{i}}{M_{i+1}/M_{i}}{g_i}$$
obtained with the first part and the assumption that $V\cap M_{j}=\{0\}$, for $i\le j-1$ we obtain a  diagram
  \[\xymatrix{
      & & \tau^{-1}(M_{i}/M_{i-1}) \ar^{\tau^{-1}\psi_i}[r] & M_r/(V\oplus M_{i}) \\
			&&\tau^{-1}M_{i}\ar^{\pi_i}[u]\ar^{g_i}[r]&M_{i+1}/M_{i}\ar^{f_i}[u]
		}\]
		with the natural vertical morphisms. Now we have $\Hom(\tau^{-1}M_i,M_r/(V\oplus M_{i}))=\Ext(M_r/(V\oplus M_{i}),M_i)=\kk$, which can be seen when applying $\Hom(-,M_i)$ to the exact sequence $e_i$ from above. Indeed, we have $$\kk=\Hom(M_{i},M_i)\cong\Ext(M_r/(V\oplus M_{i}),M_i)$$ as $\Hom(M_r/V,M_i)=0$ and $\Ext(M_r/V,M_i)=0$. Thus the diagram is forced to commute (up to scalars). As the right hand vertical map is injective and the left hand vertical and the bottom map are surjective, it follows that $\im(\tau^{-1}\psi_i)=M_{i+1}/M_{i}$ as claimed. Moreover, it follows that 
		
$$\ker(\tau^{-1}\psi_i)\cong \ker(g_i)/\ker(\pi_i)= (N_i\oplus \tau^{-1}M_{i-1})/ \tau^{-1}M_{i-1}\cong N_i.$$

If $i=j$, consider the exact sequence
$$\ses{M_j}{M_{j+1}}{M_{j+1}/M_j}$$
and the corresponding Caldero-Chapoton map. As $M_j\cap V=\{0\}$, we see that $M_{j+1}\cap V\cong (M_{j+1}\cap V+M_j)/M_j\subset M_{j+1}/M_j$ which yields $M_{j+1}\cap V \subset (M_{j+1}/M_j)\cap V$. For dimension reasons, we obtain $M_{j+1}\cap V \cong (M_{j+1}/M_j)\cap V$. 

\sayT{maybe this part is slightly inaccurate as we should deal with images of morphisms (or leave the check to the reader...)}
Finally, consider 
$$\sesm{V}{M_r/M_{j}}{M_r/(V\oplus M_{j})}{\pi}$$
and the corresponding Caldero-Chapoton map together with the natural embedding $\iota: M_{j+1}/M_j\to M_r/M_{j+1}$. Then we have $\ker(\pi\circ\iota)=\iota(M_{j+1}/M_j)\cap V\cong M_{j+1}\cap V$. 

In this case, all morphism in the commutative diagram are surjective and together with the first part we have 
$$\ker(f_j\circ g_j)\cong\ker(f_j)\oplus\ker(g_j)=(\tau^{-1}M_{j-1}\oplus N_j)\oplus (M_{j+1}\cap V).$$
As $\ker(\pi_j)=\tau^{-1}M_{j-1}$, we end up with $\ker(\tau^{-1}\psi_j)=N_j\oplus M_{j+1}\cap V$. 
		

\end{proof}


%%%%%%%%%%%%%%%%%%%%%%%%%%%%%%%%%%%%%%%%%%%%%%%%%%%
\section{$\CC^*$-equivariant Caldero-Chapoton maps}
The ideas presented here assume that we have already constructed a cell decomposition by combining the Bia\l{}ynicki-Birula cells coming from $\CC^*$-actions with the inductive construction of cell decompositions for lifts to the universal covering quiver.


%%%%%%%%%%%%%%%%%%%%%%
\section{$2$-quivers}



%%%%%%%%%%%%%%%%%%%%%%%%%%%

\begin{thebibliography}{10}

\bibitem{ass} 
  I.~Assem, D.~Simson, A.~Skowronski: Elements of the Representation Theory of Associative Algebras. Cambridge University Press, Cambridge 2007.

\bibitem{ars} 
  M.~Auslander, I.~Reiten, S.~O.~Smalo: Representation theory of Artin algebras {\bf 36}. Cambridge University Press, Cambridge 1997.

\bibitem{bb} 
  A.~Bia\l{}ynicki-Birula: Some theorems on actions of algebraic groups. Annals of Mathematics \textbf{98}, 480-497 (1973).

\bibitem{bgp} 
  I.~N.~Bernstein, I.~M.~Gelfand, V.~A.~Ponomarev: Coxeter functors, and Gabriel's theorem. Russian Mathematical Surveys \textbf{28}(2), 17-32 (1973).

\bibitem{brenner-butler} 
  S.~Brenner, M.~C.~R.~Butler: The equivalence of certain functors occurring in the representation theory of artin algebras and species. Journal of the London Mathematical Society \textbf{2}(1), 183-187 (1976).

\bibitem{cc} 
  P.~Caldero, F.~Chapoton: Cluster algebras as {H}all algebras of quiver representations. Commentarii Mathematici Helvetici \textbf{81}(3), 595-616 (2006).

\bibitem{ck} 
  P.~Caldero, B.~Keller: From triangulated categories to cluster algebras II.  Annales Scientifique de l'Ecole Normale Superi\'{e}ure (4) \textbf{39} (6), 983-1009 (2006).

\bibitem{ca}	J.B.~Carrell: Torus actions and cohomology. Algebraic quotients. Torus actions and cohomology. The adjoint representation and the adjoint action, 83-158. Encyclopaedia Mathematical Science \textbf{131}, Invariant Theory and Algebraic Transformation Groups, Springer, Berlin, 2002.

\bibitem{ce} 
  G.~Cerulli Irelli, F.~Esposito: Geometry of quiver Grassmannians of Kronecker type and applications to cluster algebras. Algebra \&  Number Theory \textbf{5}(6), 777-801 (2011).

\bibitem{cefr} 
  G.~Cerulli Irelli, F.~Esposito, H.~Franzen, M.~Reineke: Topology of Quiver Grassmannians. Preprint 2018.

\bibitem{cr} 
  P.~Caldero, M.~Reineke: On the quiver Grassmannian in the acyclic case. Journal of Pure and Applied Algebra \textbf{212}(11), 2369-2380 (2008).

\bibitem{CB}
William Crawley-Boevey. Lectures on representations of quivers. Unpublished lecture notes, online available at \url{http://www1.maths.leeds.ac.uk/~pmtwc/quivlecs.pdf}, 1992.
	

\bibitem{gab} 
  P.~Gabriel: The universal cover of a finite-dimensional algebra. Representations of algebras. Lecture Notes in Mathematics {\bf 903}, 68-105 (1981).

\bibitem{llz} 
  K.~Lee, L.~Li, A.~Zelevinsky: Greedy elements in rank 2 cluster algebras. Selecta Mathematica \textbf{20}(1), 57-82 (2014).

\bibitem{lw} 
  O.~Lorscheid, T.~Weist: Representation type via Euler characteristics and singularities of quiver Grassmannians. Preprint 2017, arXiv:1706.00860.

\bibitem{rin1} 
  C.~M.~Ringel. Exceptional modules are tree modules. Linear algebra and its Applications \textbf{275/276}, 471-493 (1998).

\bibitem{rupel}          
  D.~Rupel: Rank Two Non-Commutative Laurent Phenomenon and Pseudo-Positivity. Preprint 2017, arXiv:1707.02696.
\bibitem{rw}
D.~Rupel, T.~Weist: Cell decomposition for Rank Two Quiver Grassmannians. Preprint 2018, arXiv:1803.06590.

\bibitem{sch} 
  A.~Schofield: General representations of quivers. Proceedings of the London Mathematical Society (3) \textbf{65}(1), 46-64 (1992).

\bibitem{wei} 
  T.~Weist: Localization of quiver moduli spaces. Representation Theory \textbf{17}(13), 382-425 (2013).

\end{thebibliography}

\end{document}
