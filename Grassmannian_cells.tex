\documentclass{amsart}

% Commands for marginal notes
\usepackage[draft]{say}
\newcommand{\sayD}[1]{\say[D]{#1}}
\newcommand{\sayT}[1]{\say[T]{#1}}

\usepackage{amsmath,amssymb,latexsym,color}
\usepackage[bookmarks=true,colorlinks=true, linkcolor=blue, citecolor=cyan]{hyperref}
\usepackage{tikz}
\usetikzlibrary{arrows}
\usetikzlibrary{fit}
\usepackage[margin=1in,marginparwidth=0.8in, marginparsep=0.1in]{geometry}
\usepackage{extarrows}
\usepackage{verbatim}
\input xy
\xyoption{all}

%\usepackage[inline]{showlabels}
%\usepackage[marginal]{showlabels}

\newtheorem{theorem}{Theorem}[section]
\newtheorem{claim}[theorem]{Claim}
\newtheorem{conjecture}[theorem]{Conjecture}
\newtheorem{corollary}[theorem]{Corollary}
\newtheorem{definition}[theorem]{Definition}
\newtheorem{lemma}[theorem]{Lemma}
\newtheorem{question}[theorem]{Question}
\newtheorem{proposition}[theorem]{Proposition}
\newtheorem{remark}[theorem]{Remark}
\newtheorem{example}[theorem]{Example}
\newtheorem{thm}{Theorem}
\numberwithin{equation}{section}

\renewcommand{\AA}{\mathbb{A}}
\newcommand{\C}{\mathbb{C}}
\newcommand{\CC}{\mathbb{C}}
\newcommand{\kk}{\Bbbk}
\newcommand{\NN}{\mathbb{N}}
\newcommand{\ZZ}{\mathbb{Z}}

\newcommand{\bfe}{\mathbf{e}}
\newcommand{\bff}{\mathbf{f}}
\newcommand{\bfg}{\mathbf{g}}
\newcommand{\bfh}{\mathbf{h}}
\newcommand{\bfi}{\mathbf{i}}
\newcommand{\bfI}{\mathbf{I}}
\newcommand{\bfJ}{\mathbf{J}}
\newcommand{\bfs}{\mathbf{s}}
\newcommand{\bft}{\mathbf{t}}

\newcommand{\tbfe}{{\tilde\bfe}}
\newcommand{\tbff}{{\tilde\bff}}
\newcommand{\tbfg}{{\tilde\bfg}}
\newcommand{\tbfh}{{\tilde\bfh}}
\newcommand{\tbfs}{{\tilde\bfs}}
\newcommand{\tbft}{{\tilde\bft}}

\newcommand{\cA}{\mathcal{A}}
\newcommand{\cB}{\mathcal{B}}
\newcommand{\cC}{\mathcal{C}}
\newcommand{\cG}{\mathcal{G}}
\newcommand{\cH}{\mathcal{H}}
\newcommand{\cP}{\mathcal{P}}
\newcommand{\cQ}{\mathcal{Q}}

\newcommand{\ui}{{\underline i}}
\newcommand{\uj}{{\underline j}}
\newcommand\udim{{\underline{\dim}\, }}

\newcommand{\into}{\hookrightarrow}
\newcommand{\onto}{\to\!\!\!\!\!\to}
\newcommand{\xonto}[1]{\xrightarrow{#1}\!\!\!\!\!\to}

\newcommand{\Att}{\operatorname{Att}}
\newcommand{\codim}{\operatorname{codim}}
\newcommand{\Ext}{\operatorname{Ext}}
\newcommand{\Gr}{\mathrm{Gr}}
\newcommand{\GL}{\mathrm{GL}}
\newcommand{\Hom}{\operatorname{Hom}}
\renewcommand{\Im}{\operatorname{Im}}
\newcommand{\Ind}{\mathrm{Ind}}
\newcommand{\Irr}{\operatorname{Irr}}
\newcommand{\pt}{\mathrm{pt}}
\newcommand{\Rad}{\operatorname{Rad}}
\newcommand{\Rep}{\mathrm{Rep}}
\newcommand{\rep}{\operatorname{rep}}
\newcommand{\rk}{\mathrm{rk}}
\newcommand{\Sc}[2]{\langle #1,#2\rangle}
\newcommand{\ses}[3]{\xymatrix@C15pt{0\ar[r] & #1\ar[r] & #2\ar[r] & #3 \ar[r] & 0}}
\newcommand{\sesm}[4]{\xymatrix{0\ar[r] & #1\ar[r] & #2\ar^{#4}[r] & #3 \ar[r] & 0}}
\newcommand{\spn}{\operatorname{span}}
\newcommand{\supp}{\operatorname{supp}}
\newcommand{\vs}{\vspace{0.2cm}}

\hyphenation{endo-functors}
\setcounter{MaxMatrixCols}{20}


\title{Cell Decompositions for Acyclic Quiver Grassmannians}

\author{Dylan Rupel}
\address[Dylan Rupel]{Michigan State University, Department of Mathematics, 619 Red Cedar Road, C212 Wells Hall, East Lansing, MI 48824}
\email{drupel@nd.edu}
\author{Thorsten Weist}
\address[Thorsten Weist]{Bergische Universit\"at Wuppertal, Gau\ss str.\ 20, 42097 Wuppertal, Germany}
\email{weist@uni-wuppertal.de}

\begin{document}
\begin{abstract}
  We prove that all quiver Grassmannians for exceptional representations of an acyclic quiver admit a cell decomposition.  
  In the process, we introduce a class of regular representations which arise as quotients of consecutive exceptional representations in an exceptional sequence.
  Cell decompositions for quiver Grassmannians of these ``truncated exceptionals'' are also established. 
  The cells admit a natural combinatorial labeling in terms of 2-quivers.
\end{abstract}

\setcounter{tocdepth}{2}

\maketitle

\tableofcontents
%%%%%%%%%%%%%%%%%%%%%%
\section{Introduction}

\subsubsection*{Acknowledgements}
We would like to thank Giovanni Cerulli Irelli, Hans Franzen, Oliver Lorscheid and Markus Reineke for very fruitful discussions related to this project.


%%%%%%%%%%%%%%%%%%%%%%%%%%%%%%%%

\section{Exceptional sequences}

A representation $X$ is \emph{exceptional} if $X$ is indecomposable and rigid, i.e.\ $\Ext(X,X)=0$.
A sequence of exceptional representations $(X_1,\ldots,X_k)$ is \emph{exceptional} itself if $\Hom(X_i,X_j)=0=\Ext(X_i,X_j)$ for $i<j$.
An immediate consequence of the definition is that an exceptional sequence must consist of non-isomorphic representations. 
An exceptional sequence $(X_1,\ldots,X_k)$ is \emph{$\Ext$-orthogonal} (resp.\ \emph{$\Hom$-orthogonal}) if $\Ext(X_i,X_j)=0$ (resp. $\Hom(X_i,X_j)=0$) for all $i,j$.

An exceptional sequence $(X_1,\ldots,X_n)$ consisting of $n=|Q_0|$ exceptional representations is called \emph{complete}.
A (non-)complete $\Ext$-orthogonal exceptional sequence $(X_1,\ldots,X_k)$ determines a (partial) tilting representation $X_1\oplus\cdots\oplus X_k$. 
As any exceptional representation $X$ can be completed to tilting representation \cite{???}, every exceptional representation can be found in a complete exceptional sequence.
We will see below that this may be done with $X$ in any position of the sequence. 

Without loss of generality we will assume $Q_0=\{1,\ldots,n\}$ is a \emph{sink adapted sequence}, i.e.\ vertex 1 is a sink in $Q$ and $\{2,\ldots,n\}$ is sink adapted for the quiver obtained from $Q$ by removing vertex 1 and all arrows incident to this vertex.
Then we have a canonical $\Hom$-orthogonal exceptional sequence $(S_1,\ldots,S_n)$.
A remarkable result of Crawley-Boevey shows that all complete exceptional sequences may be obtained from this standard sequence by an action of the braid group.

Given an exceptional sequence $(X,Y)$, write $\cC(X,Y)$ for the full subcategory of $\rep Q$ containing $X$ and $Y$ which is closed under taking direct sums, kernels, and cokernels.
It is well-known \cite{???} that the category $\cC(X,Y)$ is isomorphic to the category of representations for a generalized Kronecker quiver.
In this subcategory the exceptional sequences are easy to describe, namely they are either the canonical $\Hom$-orthogonal exceptional sequence or a slice of either the preprojetive of preinjective component of the Auslander-Reiten quiver of $\cC(X,Y)$.
The braid group action is then given by either replacing an odd slice by an even slice or by replacing the projective or injective slice with the $\Hom$-orthogonal exceptional sequence.


\subsection{Quiver Grassmannian Fibrations from Exceptional Sequences}

Fix a complete exceptional sequence $(X_1,\ldots,X_n)$.
Write $S_1^{(i)}$ and $S_2^{(i)}$ for the simple projective and simple injective repectively in the category $\cC(X_i,X_{i+1})$.
Given a representation $X\in\cC(X_i,X_{i+1})$, write $\Gr^{(i)}_{e_1,e_2}(X)$ for the projective variety of subrepresentations $E\subset X$ with $E\in\cC(X_i,X_{i+1})$.
Then there is a map
\[\Gr_\bfe(X)\to\bigsqcup_{e_1,e_2} \Gr^{(i)}_{e_1,e_2}(X)\]
given by forgetting the summands of a subrepresentation $E\subset X$ which do not lie in $\cC(X_i,X_{i+1})$.

What can we say about the fibers of this map?

\section{Truncated Preprojectives}

Let $Q=(Q_0,Q_1,s,t)$ be an acyclic quiver.
For $i,j\in Q_0$, we introduce the following notation:
\begin{itemize}
  \item $Q_1(i,-)$ is the set of arrows $\alpha\in Q_1$ with $s(\alpha)=i$;
  \item $Q_1(-,j)$ is the set of arrows $\alpha\in Q_1$ with $t(\alpha)=j$;
  \item $Q_1(i,j)=Q_1(i,-)\cap Q_1(-,j)$ is the set of arrows $\alpha\in Q_1$ with $s(\alpha)=i$ and $t(\alpha)=j$;
  \item write $q_{ij}=|Q_1(i,j)|$.
\end{itemize}

For $i\in Q_0$, write $P_i$ for the projective cover of the simple $S_i$.
For $i,j\in Q_0$, write $\Irr(P_i,P_j)\subset\Hom(P_i,P_j)$ for the subspace of morphisms which have no nontrivial factorization through a projective $P_k$ with $k\ne i,j$. 
\begin{lemma}
  For $i,j\in Q_0$, we have $\dim\Irr(P_j,P_i)=q_{ij}$.
\end{lemma}

The preprojective representations $P_{i,m}$ for $i\in Q_0$ and $m\in\ZZ_{\ge0}$ are defined inductively by
\[P_{i,0}=P_i,\qquad P_{i,m+1}=\tau^{-1} P_{i,m}.\]
For $i,j\in Q_0$ and $m\in\ZZ_{\ge0}$, set $\Irr(P_{i,m},P_{j,m})=\tau^{-m}\Irr(P_i,P_j)$ and write $\Irr(P_{i,m},P_{j,m+1})\subset\Hom(P_{i,m},P_{j,m+1})$ for the subspace of morphisms which do not admit a nontrivial factorization through a preprojective $P_{k,m}$ nor through $P_{k,m+1}$ with $k\ne i,j$.
\begin{lemma}
  For $i,j\in Q_0$ and $m\in\ZZ_{\ge0}$, we have $\dim\Irr(P_{i,m},P_{j,m+1})=q_{ij}$.
\end{lemma}

\begin{proposition}
  For $i\in Q_0$ and $m\in\ZZ_{\ge0}$, the following hold:
  \begin{enumerate}
    \item there exists an Auslander-Reiten sequence
      \begin{equation}
        \label{eq:preprojective ar}
        \ses{P_{i,m}}{\bigoplus\limits_{k\in Q_1(-,i)} P_{k,m}\otimes\Irr(P_{k,m},P_{i,m+1})\oplus\bigoplus\limits_{j\in Q_1(i,-)} P_{j,m+1}\otimes \Irr(P_{j,m+1},P_{i,m+1})}{P_{i,m+1}};
      \end{equation}
    \item if there exists an Auslander-Reiten sequence
      \[\ses{P_{\ell,m-1}}{P_{i,m}}{P_{\ell,m}}\] 
      with $\ell\in Q_1(-,i)$ and $\dim\Irr(P_{\ell,m},P_{i,m+1})=1$, then the sequence \eqref{eq:preprojective ar} induces an Auslander-Reiten sequence
      \[\ses{P_{\ell,m-1}}{\bigoplus\limits_{\substack{k\in Q_1(-,i)\\ k\ne\ell}} P_{k,m}\otimes\Irr(P_{k,m},P_{i,m+1})\oplus\bigoplus\limits_{j\in Q_1(i,-)} P_{j,m+1}\otimes \Irr(P_{j,m+1},P_{i,m+1})}{P_{i,m+1}};\] 
    \item if there exists an Auslander-Reiten sequence
      \[\ses{P_{\ell,m}}{P_{i,m}}{P_{\ell,m+1}}\] 
      with $\ell\in Q_1(i,-)$ and $\dim\Irr(P_{\ell,m+1},P_{i,m+1})=1$, then the sequence \eqref{eq:preprojective ar} induces an Auslander-Reiten sequence
      \[\ses{P_{\ell,m}}{\bigoplus\limits_{k\in Q_1(-,i)} P_{k,m}\otimes\Irr(P_{k,m},P_{i,m+1})\oplus\bigoplus\limits_{\substack{j\in Q_1(i,-)\\ j\ne\ell}} P_{j,m+1}\otimes \Irr(P_{j,m+1},P_{i,m+1})}{P_{i,m+1}}.\] 
  \end{enumerate}
\end{proposition}
\begin{proof}
  To see part (2), we consider the commutative diagram:
  \[\xymatrix{
      0 \ar[d] & 0 \ar[d] \\
      P_{\ell,m-1} \ar[d]\ar[r]\ar@{}[dr]|(.7){\lrcorner} & \bigoplus\limits_{\substack{k\in Q_1(-,i)\\ k\ne\ell}} P_{k,m}\otimes\Irr(P_{k,m},P_{i,m+1})\oplus\bigoplus\limits_{j\in Q_1(i,-)} P_{j,m+1}\otimes \Irr(P_{j,m+1},P_{i,m+1}) \ar[d] \\
      P_{i,m} \ar[d]\ar[r] & \bigoplus\limits_{k\in Q_1(-,i)} P_{k,m}\otimes\Irr(P_{k,m},P_{i,m+1})\oplus\bigoplus\limits_{j\in Q_1(i,-)} P_{j,m+1}\otimes \Irr(P_{j,m+1},P_{i,m+1}) \ar[d] \\
      P_{\ell,m} \ar@{=}[r]\ar[d] & P_{\ell,m} \ar[d] \\
      0 & 0}\]
  Since the lower morphism is an equality, the top square is a pushout.
  It follows that the cokernels of the two upper horizontal morphisms are equal, giving the claim.
\end{proof}


%%%%%%%%%%%%%%%%%%%%%%%%%%%
\begin{thebibliography}{10}

\bibitem{ass} 
  I.~Assem, D.~Simson, A.~Skowronski: Elements of the Representation Theory of Associative Algebras. Cambridge University Press, Cambridge 2007.

\bibitem{ars} 
  M.~Auslander, I.~Reiten, S.~O.~Smalo: Representation theory of Artin algebras {\bf 36}. Cambridge University Press, Cambridge 1997.

\bibitem{bb} 
  A.~Bia\l{}ynicki-Birula: Some theorems on actions of algebraic groups. Annals of Mathematics \textbf{98}, 480-497 (1973).

\bibitem{bgp} 
  I.~N.~Bernstein, I.~M.~Gelfand, V.~A.~Ponomarev: Coxeter functors, and Gabriel's theorem. Russian Mathematical Surveys \textbf{28}(2), 17-32 (1973).

\bibitem{brenner-butler} 
  S.~Brenner, M.~C.~R.~Butler: The equivalence of certain functors occurring in the representation theory of artin algebras and species. Journal of the London Mathematical Society \textbf{2}(1), 183-187 (1976).

\bibitem{cc} 
  P.~Caldero, F.~Chapoton: Cluster algebras as {H}all algebras of quiver representations. Commentarii Mathematici Helvetici \textbf{81}(3), 595-616 (2006).

\bibitem{ck} 
  P.~Caldero, B.~Keller: From triangulated categories to cluster algebras II.  Annales Scientifique de l'Ecole Normale Superi\'{e}ure (4) \textbf{39} (6), 983-1009 (2006).

\bibitem{ca}	J.B.~Carrell: Torus actions and cohomology. Algebraic quotients. Torus actions and cohomology. The adjoint representation and the adjoint action, 83-158. Encyclopaedia Mathematical Science \textbf{131}, Invariant Theory and Algebraic Transformation Groups, Springer, Berlin, 2002.

\bibitem{ce} 
  G.~Cerulli Irelli, F.~Esposito: Geometry of quiver Grassmannians of Kronecker type and applications to cluster algebras. Algebra \&  Number Theory \textbf{5}(6), 777-801 (2011).

\bibitem{cefr} 
  G.~Cerulli Irelli, F.~Esposito, H.~Franzen, M.~Reineke: Topology of Quiver Grassmannians. Preprint 2018.

\bibitem{cr} 
  P.~Caldero, M.~Reineke: On the quiver Grassmannian in the acyclic case. Journal of Pure and Applied Algebra \textbf{212}(11), 2369-2380 (2008).

\bibitem{gab} 
  P.~Gabriel: The universal cover of a finite-dimensional algebra. Representations of algebras. Lecture Notes in Mathematics {\bf 903}, 68-105 (1981).

\bibitem{llz} 
  K.~Lee, L.~Li, A.~Zelevinsky: Greedy elements in rank 2 cluster algebras. Selecta Mathematica \textbf{20}(1), 57-82 (2014).

\bibitem{lw} 
  O.~Lorscheid, T.~Weist: Representation type via Euler characteristics and singularities of quiver Grassmannians. Preprint 2017, arXiv:1706.00860.

\bibitem{rin1} 
  C.~M.~Ringel. Exceptional modules are tree modules. Linear algebra and its Applications \textbf{275/276}, 471-493 (1998).

\bibitem{rupel} 
  D.~Rupel: Rank Two Non-Commutative Laurent Phenomenon and Pseudo-Positivity. Preprint 2017, arXiv:1707.02696.

\bibitem{sch} 
  A.~Schofield: General representations of quivers. Proceedings of the London Mathematical Society (3) \textbf{65}(1), 46-64 (1992).

\bibitem{wei} 
  T.~Weist: Localization of quiver moduli spaces. Representation Theory \textbf{17}(13), 382-425 (2013).

\end{thebibliography}

\end{document}
