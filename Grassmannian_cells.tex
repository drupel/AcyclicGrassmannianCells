\documentclass{amsart}

% Commands for marginal notes
\usepackage[draft]{say}
\newcommand{\sayD}[1]{\say[D]{#1}}
\newcommand{\sayT}[1]{\say[T]{#1}}

\usepackage{amsmath,amssymb,latexsym,color}
\usepackage[bookmarks=true,colorlinks=true, linkcolor=blue, citecolor=cyan]{hyperref}
\usepackage{tikz}
\usetikzlibrary{arrows}
\usetikzlibrary{fit}
\usepackage[margin=1in,marginparwidth=0.8in, marginparsep=0.1in]{geometry}
\usepackage{extarrows}
\usepackage{verbatim}
\input xy
\xyoption{all}

\usepackage[inline]{showlabels}
%\usepackage[marginal]{showlabels}

\newtheorem{theorem}{Theorem}[section]
\newtheorem{claim}[theorem]{Claim}
\newtheorem{conjecture}[theorem]{Conjecture}
\newtheorem{corollary}[theorem]{Corollary}
\newtheorem{definition}[theorem]{Definition}
\newtheorem{lemma}[theorem]{Lemma}
\newtheorem{question}[theorem]{Question}
\newtheorem{proposition}[theorem]{Proposition}
\newtheorem{remark}[theorem]{Remark}
\newtheorem{example}[theorem]{Example}
\newtheorem{thm}{Theorem}
\numberwithin{equation}{section}

\renewcommand{\AA}{\mathbb{A}}
\newcommand{\C}{\mathbb{C}}
\newcommand{\CC}{\mathbb{C}}
\newcommand{\kk}{\Bbbk}
\newcommand{\NN}{\mathbb{N}}
\newcommand{\ZZ}{\mathbb{Z}}

\newcommand{\bfe}{\mathbf{e}}
\newcommand{\bff}{\mathbf{f}}
\newcommand{\bfg}{\mathbf{g}}
\newcommand{\bfh}{\mathbf{h}}
\newcommand{\bfi}{\mathbf{i}}
\newcommand{\bfI}{\mathbf{I}}
\newcommand{\bfJ}{\mathbf{J}}
\newcommand{\bfs}{\mathbf{s}}
\newcommand{\bft}{\mathbf{t}}

\newcommand{\tbfe}{{\tilde\bfe}}
\newcommand{\tbff}{{\tilde\bff}}
\newcommand{\tbfg}{{\tilde\bfg}}
\newcommand{\tbfh}{{\tilde\bfh}}
\newcommand{\tbfs}{{\tilde\bfs}}
\newcommand{\tbft}{{\tilde\bft}}

\newcommand{\cA}{\mathcal{A}}
\newcommand{\cB}{\mathcal{B}}
\newcommand{\cC}{\mathcal{C}}
\newcommand{\cG}{\mathcal{G}}
\newcommand{\cH}{\mathcal{H}}
\newcommand{\cI}{\mathcal{I}}
\newcommand{\cP}{\mathcal{P}}
\newcommand{\cQ}{\mathcal{Q}}
\newcommand{\cU}{\mathcal{U}}
\newcommand{\cV}{\mathcal{V}}
\newcommand{\cW}{\mathcal{W}}

\newcommand{\ui}{{\underline i}}
\newcommand{\uj}{{\underline j}}
\newcommand\udim{{\underline{\dim}\, }}

\newcommand{\into}{\hookrightarrow}
\newcommand{\onto}{\to\!\!\!\!\!\to}
\newcommand{\xonto}[1]{\xrightarrow{#1}\!\!\!\!\!\to}

\newcommand{\Att}{\operatorname{Att}}
\newcommand{\codim}{\operatorname{codim}}
\newcommand{\End}{\operatorname{End}}
\newcommand{\Ext}{\operatorname{Ext}}
\newcommand{\Gr}{\mathrm{Gr}}
\newcommand{\GL}{\mathrm{GL}}
\newcommand{\Hom}{\operatorname{Hom}}
\renewcommand{\Im}{\operatorname{Im}}
\newcommand{\Ind}{\mathrm{Ind}}
\newcommand{\Irr}{\operatorname{Irr}}
\newcommand{\pt}{\mathrm{pt}}
\newcommand{\Rad}{\operatorname{Rad}}
\newcommand{\Rep}{\mathrm{Rep}}
\newcommand{\rep}{\operatorname{rep}}
\newcommand{\rk}{\mathrm{rk}}
\newcommand{\Sc}[2]{\langle #1,#2\rangle}
\newcommand{\ses}[3]{\xymatrix@C15pt{0\ar[r] & #1\ar[r] & #2\ar[r] & #3 \ar[r] & 0}}
\newcommand{\sesm}[4]{\xymatrix{0\ar[r] & #1\ar[r] & #2\ar^{#4}[r] & #3 \ar[r] & 0}}
\newcommand{\spn}{\operatorname{span}}
\newcommand{\supp}{\operatorname{supp}}
\newcommand{\vs}{\vspace{0.2cm}}

\hyphenation{endo-functors}
\setcounter{MaxMatrixCols}{20}


\title{Cell Decompositions for Acyclic Quiver Grassmannians}

\author{Dylan Rupel}
\address[Dylan Rupel]{Michigan State University, Department of Mathematics, 619 Red Cedar Road, C212 Wells Hall, East Lansing, MI 48824}
\email{drupel@nd.edu}
\author{Thorsten Weist}
\address[Thorsten Weist]{Bergische Universit\"at Wuppertal, Gau\ss str.\ 20, 42097 Wuppertal, Germany}
\email{weist@uni-wuppertal.de}

\begin{document}
\begin{abstract}
  We prove that all quiver Grassmannians for exceptional representations of an acyclic quiver admit a cell decomposition.  
  In the process, we introduce a class of regular representations which arise as quotients of consecutive exceptional representations in an exceptional sequence.
  Cell decompositions for quiver Grassmannians of these ``truncated exceptionals'' are also established. 
  The cells admit a natural combinatorial labeling in terms of 2-quivers.
\end{abstract}

\setcounter{tocdepth}{2}

\maketitle
\tableofcontents

%%%%%%%%%%%%%%%%%%%%%%

\section{Introduction}

\subsubsection*{Acknowledgements}
We would like to thank Giovanni Cerulli Irelli, Hans Franzen, Oliver Lorscheid and Markus Reineke for very fruitful discussions related to this project.


\subsection{Notation}
\begin{itemize}
  \item Given two tuples $x=(x_1,\ldots,x_n)$ and $y=(y_1,\ldots,y_m)$, we write $x\smile y$ for their concatenation $(x_1,\ldots,x_n,y_1,\ldots,y_m)$. (better notation?)
  \item Given two tuples of vector spaces $\cU=(U_1,\ldots,U_n)$ and $\cW=(W_1,\ldots,W_n)$, we write $\cU\subset\cW$ if $U_i$ is a subspace of $W_i$ for each $i$ and $\cU\subsetneq\cW$ if at least one of these subspaces is proper.
    In this case, we write $\cW/\cU=(W_1/U_1,\ldots,W_n/U_n)$ and set $\codim_\cW(\cU)=\sum\limits_{i=1}^n \dim(W_i/U_i)$.
  \item Given two tuples of vector spaces $\cU=(U_1,\ldots,U_n)$ and $\cW=(W_1,\ldots,W_n)$, we write $\cU\bullet\cW$ for $\bigoplus_{i=1}^n U_i\otimes W_i$. (better notation?)
\end{itemize}


%%%%%%%%%%%%%%%%%%%%%%%%%%%%%%%

\section{Exceptional sequences}

A representation $X$ is \emph{exceptional} if $X$ is indecomposable and rigid, i.e.\ $\Ext(X,X)=0$.
A sequence of exceptional representations $(X_1,\ldots,X_k)$ is \emph{exceptional} itself if $\Hom(X_i,X_j)=0=\Ext(X_i,X_j)$ for $i<j$.
An immediate consequence of the definition is that an exceptional sequence must consist of non-isomorphic representations. 
An exceptional sequence $(X_1,\ldots,X_k)$ is \emph{$\Ext$-orthogonal} (resp.\ \emph{$\Hom$-orthogonal}) if $\Ext(X_i,X_j)=0$ (resp. $\Hom(X_i,X_j)=0$) for all $i,j$.

An exceptional sequence $(X_1,\ldots,X_n)$ consisting of $n=|Q_0|$ exceptional representations is called \emph{complete}.
A (non-)complete $\Ext$-orthogonal exceptional sequence $(X_1,\ldots,X_k)$ determines a (partial) tilting representation $X_1\oplus\cdots\oplus X_k$. 
As any exceptional representation $X$ can be completed to tilting representation \cite{???}, every exceptional representation can be found in a complete exceptional sequence.
We will see below that this may be done with $X$ in any position of the sequence. 

Without loss of generality we will assume $Q_0=\{1,\ldots,n\}$ is labeled along a \emph{sink adapted sequence}, i.e.\ vertex 1 is a sink in $Q$ and $(2,\ldots,n)$ is sink adapted for the quiver obtained from $Q$ by removing vertex 1 and all arrows incident to this vertex.
Then we have a canonical $\Hom$-orthogonal exceptional sequence $(S_1,\ldots,S_n)$.
A remarkable result of Crawley-Boevey shows that all complete exceptional sequences may be obtained from this standard sequence by an action of the braid group.

Given an exceptional sequence $(X,Y)$, write $\cC(X,Y)$ for the full subcategory of $\rep Q$ containing $X$ and $Y$ which is closed under taking direct sums, kernels, and cokernels.
It is well-known \cite{???} that the category $\cC(X,Y)$ is isomorphic to the category of representations for a generalized Kronecker quiver.
In this subcategory the exceptional sequences are easy to describe, namely they are either the canonical $\Hom$-orthogonal exceptional sequence or a slice of either the preprojetive of preinjective component of the Auslander-Reiten quiver of $\cC(X,Y)$.
The braid group action is then given by either replacing an odd slice by an even slice or by replacing the projective or injective slice with the $\Hom$-orthogonal exceptional sequence.


\subsection{Quiver Grassmannian Fibrations from Exceptional Sequences}

Fix a complete exceptional sequence $(X_1,\ldots,X_n)$.
Write $S_1^{(i)}$ and $S_2^{(i)}$ for the simple projective and simple injective repectively in the category $\cC(X_i,X_{i+1})$.
Given a representation $X\in\cC(X_i,X_{i+1})$, write $\Gr^{(i)}_{e_1,e_2}(X)$ for the projective variety of subrepresentations $E\subset X$ with $E\in\cC(X_i,X_{i+1})$.
Then there is a map
\[\Gr_\bfe(X)\to\bigsqcup_{e_1,e_2} \Gr^{(i)}_{e_1,e_2}(X)\]
given by forgetting the summands of a subrepresentation $E\subset X$ which do not lie in $\cC(X_i,X_{i+1})$.

What can we say about the fibers of this map?


%%%%%%%%%%%%%%%%%%%%%%%%%%%%%%%%%%

\section{Truncated Preprojectives}

Let $Q=(Q_0,Q_1,s,t)$ be a connected acyclic quiver, to avoid special cases we assume $Q$ is not of finite representation type.
Given a sink or source $k\in Q_0$, write $\mu_k Q$ for the quiver obtained from $Q$ by reversing all arrows incident to vertex $k$.
Without loss of generality, we will assume the vertex set $Q_0=\{1,\ldots,n\}$ is labeled along a \emph{sink adapted sequence}, i.e.~vertex 1 is a sink in $Q$, vertex $2$ is a sink in $\mu_1 Q$, vertex $3$ is a sink in $\mu_2\mu_1 Q$, etc.
Observe that this implies vertex $n$ is a source in $Q$, vertex $n-1$ is a source in $\mu_n Q$, vertex~$n-2$ is a source in $\mu_{n-1}\mu_n Q$, etc.

For $i,j\in Q_0$, we introduce the following notation:
\begin{itemize}
  \item $Q_1(i,-)$ is the set of arrows $\alpha\in Q_1$ with source $s(\alpha)=i$ and $Q_0(i,-)$ is the set of targets $t(\alpha)$ for all such arrows;
  \item $Q_1(-,j)$ is the set of arrows $\alpha\in Q_1$ with target $t(\alpha)=j$ and $Q_0(-,j)$ is the set of sources $s(\alpha)$ for all such arrows;
  \item $Q_1(i,j)=Q_1(i,-)\cap Q_1(-,j)$ is the set of arrows $\alpha\in Q_1$ with $s(\alpha)=i$ and $t(\alpha)=j$;
  \item write $b_{ij}=|Q_1(i,j)|$ for the number of arrows from vertex $i$ to vertex $j$.
\end{itemize}

A \emph{leaf} of $Q$ which is a vertex $\ell\in Q_0$ with $|Q_1(\ell,-)\cup Q_1(-,\ell)|=1$.
For vertices $i,\ell\in Q_0$, an arrow $\alpha\in Q_1(i,\ell)\cup Q_1(\ell,i)$ is called a \emph{limb arrow} and $\ell$ is a \emph{limb vertex} with respect to vertex $i$ if removing $\alpha$ from $Q$ creates a disconnected quiver in which the component containing vertex $\ell$ is an orientation of a Dynkin diagram of type A with $\ell$ a leaf in this component.

Note that each limb vertex $\ell\in Q_0$ has a uniquely associated leaf vertex $\ell'$ of~$Q$ lying at the opposite end of its type A component, if $\ell$ itself is a leaf of $Q$ then $\ell'=\ell$.
Once a limb vertex $\ell$ is specified, the type A component associated to $\ell$ is uniquely determined (we assume $Q$ itself is not of type A).
In the type A subquiver $Q(\ell)\subset Q$ obtained by adjoining the limb arrow $\alpha$ to this component, each arrow is directed either toward vertex $i$ or toward vertex $\ell'$.
We write $h(\ell)$ for the total number of arrows in the quiver $Q(\ell)$ directed toward vertex $i$.

For a sink or source vertex $k\in Q_0$, write $\Sigma_k:\rep Q\to\rep \mu_k Q$ for the BGP-reflection functor \cite{BGP??} (see also \cite{APR??} for an equivalent formulation known as APR-tilting).
Note that we use the same symbol $\Sigma$ to denote the reflection functor associated to a sink or a source and also for the reflection functors associated to any reflection of $Q$.
In particular, the functor $\Sigma_k^2$ is naturally isomorphic to the identity functor when restricted to the full subcategory $\rep^{\langle k\rangle} Q\subset\rep Q$ consisting of representations with no summand isomorphic to the vertex simple $S_k$.
Moreover, the reflection functor $\Sigma_k:\rep^{\langle k\rangle} Q\to\rep^{\langle k\rangle} \mu_k Q$ is an exact equivalence of categories \cite{DR76}.
Recall that the \emph{Auslander-Reiten translation} $\tau:\rep Q\to\rep Q$ is naturally isomorphic to the product of reflection functors $\Sigma_n\cdots\Sigma_1$ \cite{BB??}.

For $i\in Q_0$, write $P_i$ for the projective cover of the simple $S_i$ and write $I_i$ for its injective hull.
These representations are constructed inductively via $P_i=\Sigma_1\cdots\Sigma_{i-1}S_i$ (resp.~via $I_i=\Sigma_n\cdots\Sigma_{i+1}S_i$), where $S_i$ here denotes the vertex simple of the quiver $\mu_{i-1}\cdots\mu_1 Q$ (resp.~of the quiver $\mu_{i+1}\cdots\mu_n Q$).
%For $i,j\in Q_0$, write $\Irr(P_i,P_j)\subset\Hom(P_i,P_j)$ for the subspace of morphisms which have no nontrivial factorization through a projective $P_k$ with $k\ne i,j$. 
%\begin{lemma}
%  For $i,j\in Q_0$, we have $\dim\Irr(P_j,P_i)=b_{ij}$.
%\end{lemma}
The \emph{preprojective} (resp.~\emph{preinjective}) representations $P_{i,m}$ (resp.~$I_{i,m}$) for $i\in Q_0$ and $m\in\ZZ_{\ge0}$ are defined inductively by
\[P_{i,0}=P_i,\qquad P_{i,m+1}=\tau^{-1} P_{i,m},\qquad I_{i,0}=I_i,\qquad I_{i,m+1}=\tau I_{i,m}.\]
More generally, we say a representation $M\in\rep Q$ is \emph{preprojective} (resp.~\emph{preinjective}) if $\tau^r M=0$ (resp. if~$\tau^{-r} M=0$) for some $r>0$.

\begin{lemma}
  \label{le:preprojective ext groups}
  For $i,j,k\in Q_0$ with $|Q_1(i,j)|\ne0$ and $m\ge0$, we have 
  \[\Ext(P_{k,m},P_{i,m})=0 \qquad \text{and} \qquad \Ext(P_{j,m+1},P_{i,m})=0.\]
\end{lemma}
%\begin{proof}
%  The first claim for $m=0$ is immediate since the representations $P_{i,0}$ are projective, for $m>0$ this follows by applying Auslander-Reiten translations.
%  The second claim may be reduced to the case $m=0$ and $j$ a sink in $Q$ using Auslander-Reiten translations and reflection functors.
%\end{proof}
For two dimension vectors $\alpha$ and $\beta$ we denote by $\hom(\alpha,\beta)$ the minimal and thus general value if $\dim\Hom:R_\alpha(Q)\times R_\beta(Q)\to \NN$, see \cite{sch} for more details.
\begin{lemma}
  \label{lem: non-preinjective}
  If $M,N\in\rep(Q)$ are indecomposable such that $\udim M$ is an imaginary root and such that $\Hom(N,M)\neq 0$, then $N$ is not preinjective.
  In particular, if $\hom(\alpha,\beta)\neq 0$ and $\beta$ is an imaginary Schur root, $\alpha$ is not the dimension vector of a preinjective representation.
\end{lemma}
\begin{proof}
  As $\udim M$ is imaginary, $M$ is a regular representation.
  Indeed all preinjectives and preprojective indecomposable representations are exceptional.
  But if $N$ were preinjective, we had $\Hom(N,M)\neq 0$ only for preinjectives $M$.

  Finally, if $\beta$ is a Schur root, then a general representation is indecomposable which gives the second part.
\end{proof}

For a dimension vector $\alpha\in\NN^{Q_0}$, we write $\alpha=\sum_{q\in Q_0}\alpha_q q$.
\begin{proposition}
  \label{pro:simpleregular}
  Let $Q$ be any acyclic quiver which is not of Dynkin type and let $i\in Q_0$. % be neither a sink nor a source. 
  %Then $S_\ell$ is preprojective if and only if there exists a unique limb vertex $n\in Q_1(\ell,-)$ and 
  Then $S_i$ is preinjective if and only if one of the following cases holds: 
  \begin{enumerate}
    \item $|Q_1(-,i)|=0$;
    \item $|Q_1(-,i)|=1$ and $\ell\in Q_0(-,i)$ is a limb vertex such that the associated leaf is not equal to $i$.
      \sayT{If the quiver is of extended Dynkin type $A$, the simples corresponding to vertices which are neither sink nor source are regular, but are limb vertices. Probably we want to exclude this cases by definition?!}
  \end{enumerate}
In particular, $S_i$ is not preinjective if $|Q_1(-,i)|\geq 2$.
			
\end{proposition}
\begin{proof}
  If $Q_1(-,i)=0$, then $i$ is a source of $Q$ and so $S_i$ is injective.

  Assume $|Q_1(-,i)|=1$, say $\ell\in Q_0(-,i)$, and that $\ell$ is a limb vertex with respect to $i$ with associated leaf $\ell'$ in $Q(\ell)$.
  Then $S_i=\tau^{-h(\ell)}I_{\ell',0}$ is preinjective.
  \sayT{reference?} 
	

For the remaining cases, the aim is to apply Lemma~\ref{lem: non-preinjective}. Let $Q'$ be the connected component containing $i$ after removing all arrows from $Q_1(i,-)$.
	
If $Q'$ is wild or of extended Dynkin type, let $M$ be any indecomposable representation of $Q'$ such that $\beta=\udim M$ is an imaginary root of $Q'$ and thus of $Q$. We can naturally identify $M$ with a representation of $Q$ when setting $M_\alpha=0$ for all $\alpha\in Q_1(i,-)$. This yields $\Hom_Q(S_i,M)\neq 0$. Now Lemma \ref{lem: non-preinjective} gives the claim. 

Thus in the following we assume that $Q'$ is of Dynkin type $D$ or $E$ (with leaf $i$) or that $|Q_1(-,i)|\geq 2$ holds if it is of type $A$.
  In particular, we can assume that $|Q_0(-,i)|=|Q_1(-,i)|$, i.e.~there are no parallel arrows with target $i$. Moreover, since $Q$ is not of Dynkin type, there exists a vertex $j$ such that $|Q_1(i,j)|\geq 1$.  
	
	%If $|Q_1(i,-)|=0$, then $S_i$ is projective and hence not preinjective as $Q$ is not of Dynkin type.
 % If there exists $q\in Q_0(-,i)$ with $|Q_1(q,i)|\ge2$, then the root $q+i$ is imaginary with $\hom(i,q+i)=1$ and thus $S_i$ is not preinjective by Lemma~\ref{lem: non-preinjective}.

	First assume that $|Q_0(-,i)|\geq 2$. Thus there exist distinct vertices $q,q'\in Q_0(-,i)$.  Note that $q,q'\neq j$ as $Q$ is acyclic. 
  If $|Q_1(i,j)|\geq 2$, we may consider the imaginary root $\beta=q+q'+3i+j$ for which we have $\hom(i,\beta)\geq 1$.
  This shows that $S_i$ is not preinjective by Lemma~\ref{lem: non-preinjective}.
  Thus we may additionally assume that $|Q_0(i,-)|=|Q_1(i,-)|$, i.e.~there are no parallel arrows with source $i$.

  If there were a vertex $j'\notin \{j,q,q'\}$ such that $|Q_1(j',i)\cup Q_1(i,j')|\geq 1$, there exists a subquiver $Q''$ of $Q$ with vertices $\{i,j,j',q,q'\}$ of type $\tilde D_4$ with middle vertex $i$ which is neither a source nor a sink.
  Then it is straightforward that $S_i$ is regular as a representation of $Q''$.
  In particular, there exists an indecomposable imaginary root representation $X$ in the same tube which is of dimension $\delta=j+j'+q+q'+2i$ and such that $\Hom(S_i,X)\neq 0$.
  As we clearly have $\dim\Ext_Q(X,X)\geq \dim\Ext_{Q''}(X,X)\neq 0$, we can again apply Lemma~\ref{lem: non-preinjective} and conclude that $S_i$ is not preinjective.

  Thus we may assume that there does not exist such a vertex $j'$ and so $i$ has precisely three neighbors $q$, $q'$, and $j$. Note that, as $Q'$ is of Dynkin type, $q$ is not connected to $q'$.
	
	  If $j$ is connected to $q$ (or analogously to $q'$) by a sequence $j=q_1,q_2,\ldots,q_r=q$ with $q_i\neq q'$, the root $\beta=i+\sum_{p=1}^rq_p$ is imaginary. Denoting the arrow from $i$ to $j$ by $\rho$, there exists a Schurian representation of dimension $\beta$ with $X_\rho=0$ which implies $\Hom(S_i,X_\rho)\neq 0$. 
  Thus Lemma \ref{lem: non-preinjective} yields the claim in this case.  


  Thus we may additionally assume that there exists precisely one path from each of the vertices $q,q'$ to $j$. If there exists vertex $i'\in Q_0$ different from $i$ such that $|Q_0(i,-)\cup Q_0(-,i)|\geq 3$, it is straightforward to check that $Q$ has a subquiver of extended Dynkin type $D$. In this subquiver precisely two vertices out of $\{q,q',j\}$ are leafs. With this observation, it is straightforward to reduce to the case $\tilde D_4$ treated before.

  Thus it remains to consider the case when $Q$ is a flag quiver with central vertex $i$.
  If $Q$ is not a tree, i.e. there exist vertices $s,s'$ with $|Q_1(s,s')|\geq 2$, the root 
  $$\beta=2i+\sum_{q\in Q_0\backslash\{i\}}q$$
  is imaginary with $\hom(i,\beta)\neq 0$ and Lemma \ref{lem: non-preinjective} applies.
  
	Thus assume that $Q$ is a tree. As it is not of Dynkin type, it contains an extended Dynkin quiver of type $E$ as a full subquiver. But then it is straightforward that $\hom(i,\delta)\neq 0$ for the respective imaginary Schur roots $\delta$, see \cite[Section 4]{CB} for a list. Thus again Lemma \ref{lem: non-preinjective} applies.
	
	It remains to consider the case when $|Q_0(-,i)|=1$ and $\ell\in Q_0(-,i)$ is a limb vertex such that corresponding leaf is not equal to $i$. Let $\gamma:\ell\to i$ denote the unique arrow with sink $i$. 
The consideration from above give that we can assume that $Q'$ is of Dynkin type $D$ or $E$ (with leaf $i$). Moreover, we can assume that removing $\gamma$ yields precisely two connected components, the one of $\ell$ and the one of $i$. Applying reflection functors we can assume that all arrows in $Q'$ are oriented towards $\ell$, i.e. $\ell$ is a sink of $Q'$. By assumption, we either have $|Q_0(-,\ell)|=2$ or there is a vertex $\ell'\in Q'_0$ such that $|Q_0(-,\ell')|=2$ and $|Q_0(\ell',-)|=1$. Moreover, there is a unique path from $\ell'$ to $i$ where we denote the vertices in between by $q_1,\ldots,q_r$. Now let $Q(\ell',i)$ be the quiver obtained from $Q$ when identifying $\ell'$ and $i$ and removing all arrows and vertices along the path from $\ell'$ to $i$. Denote the vertex obtained via this identification by $(\ell',i)$. It is straightforward to check that roots $\beta$ of $Q$ with $\beta_i=\beta_{q_l}=\beta_\ell'$ for $l=1,\ldots,r$ are imaginary if and only if the corresponding root of $Q(\ell',i)$ with $\beta_{(\ell',i)}=\beta_i$ is imaginary. Thus if $Q(\ell',i)$ is not of Dynkin type, we can make use of the observations in the case $|Q(-,i)|=2$. Note that $|Q(\ell',i)_1(-,(\ell',i))|\geq 2$.

 Thus we may finally assume that $Q(\ell',i)$ is of Dynkin type and $Q$ is not. This is only possible if $Q$ is of type $E$. But then for the imaginary Schur root $\delta$ we have $\delta_i<\delta_\ell$. But this yields $\hom(i,\delta)\neq 0$ and thus the claim.


%Let $M$ be any indecomposable representation of $Q'$ such that $\beta=\udim M$ is an imaginary root of $Q'$ and thus of $Q$. We can naturally identify $M$ with a representation of $Q$ when setting $M_\alpha=0$ for all $\alpha\in Q_1(i,-)$. This yields $\Hom_Q(S_i,M)\neq 0$. Thus Lemma \ref{lem: non-preinjective} yields the claim.
  
	 

\end{proof}

\begin{corollary}
  \label{pro:simpleregular2}
  Let $Q$ be any acyclic quiver which is not of Dynkin type and let $i\in Q_0$. Then $S_i$ is preprojective if and only if one of the following cases holds: 
  \begin{enumerate}
    \item $|Q_1(i,-)|=0$;
    \item $|Q_1(i,-)|=1$ and $\ell\in Q_0(i,-)$ is a limb vertex such that the associated leaf is not equal to $i$.
		\end{enumerate}
In particular, $S_i$ is not preprojective if $|Q_1(i,-)|\geq 2$.
			

\end{corollary}

\begin{lemma}
  \label{le:nonlimb extensions}
  Suppose $\ell\in Q_0(n,-)$ is not a limb vertex with respect to the source vertex $n$.
  Then any nontrivial extension $E\in\Ext(S_n,P_{\ell,0})$ is indecomposable and not preinjective.
\end{lemma}
\begin{proof}
  Recall that $\dim (P_{\ell,0})_n$ is the number of paths from $\ell$ to $n$.
  As there is an arrow from $n$ to $\ell$ and as $Q$ is acyclic, we have $\dim (P_{\ell,0})_n=0$.
  By assumption, we have a non-split exact sequence
  \[\ses{P_{\ell,0}}{E}{S_n}.\]
  Since $\Hom(P_{\ell,0},S_n)=0$, applying $\Hom(P_{\ell,0},-)$ to this sequence shows $\Hom(P_{\ell,0},E)\cong\Hom(P_{\ell,0},P_{\ell,0})$ is one-dimensional.
  Since $S_n$ is injective, we have $\Hom(S_n,E)=0$ and thus applying $\Hom(-,E)$ to the sequence above shows that $\End(E)$ is also one-dimensional, in particular $E$ must be indecomposable.

  Using that $P_{\ell,0}$ and $S_n$ are exceptional and $\Sc{P_{\ell,0}}{S_n}=0$, we obtain
  $$\Sc{\udim E}{\udim E}=\Sc{\udim P_{\ell,0}+S_n}{\udim P_{\ell,0}+S_n}=2-\Sc{S_n}{P_{\ell,0}}=2-\dim\Ext(S_n,P_{\ell,0}).$$
  If $\dim\Ext(S_n,P_{\ell,0})\geq 2$, it follows that $\udim E$ is an imaginary Schur root and thus $E$ is not preinjective.

  Thus assume that $\dim\Ext(S_n,P_{\ell,0})=1$.
  This implies $\ell$ is the only vertex in the support of $P_{\ell,0}$ which is a neighbor of $n$ and $|Q_1(n,\ell)|=1$, otherwise we could again conclude that $\dim\Ext(S_n,P_{\ell,0})\geq 2$.
  Let $Q'\subset Q$ denote the support quiver of $E$ and label $Q'_0=\{i_1,\ldots,i_r\}$ along a sink adapted sequence for $Q'$.
  Then by definition of $P_{\ell,0}$ we must have $i_r=n$ and $i_{r-1}=\ell$.
  It follows that 
  $$P_{\ell,0}=\Sigma_{i_1}\cdots\Sigma_{i_{r-2}} S_\ell \qquad \text{and} \qquad E=\Sigma_{i_1}\cdots\Sigma_{i_{r-2}}\Sigma_n S_\ell.$$
 

% in particular $E$ is projective as a representation of $Q'$.
 % It follows that $E$ is not injective as a representation of $Q$.
 % Indeed, this would imply that $E$ is injective as a representation of $Q'$, but as $\ell$ is not a limb vertex, $Q'$ is not of type A and thus $Q'$ does not admit a projective-injective representation.
  %\sayD{I am not sure why $E$ having a one-dimensional space at vertex $n$ implies it is not strictly preinjective.}
	%It follows that $\Sigma_n\Sigma_{i_{r-1}}\cdots\Sigma_{i_1} E\cong S_\ell$ as representations of $\mu_n\mu_{i_{r-1}}\cdots\mu_{i_1} Q$.
  As $\ell$ is not a limb vertex with respect to $n$ and as $\ell$ is a sink of $Q'':=\mu_{i_{1}}\ldots\mu_{i_{r-1}}Q$, there exist at least two arrows $\alpha:q\to \ell$ and $\alpha':q'\to \ell$ with $q,q'\neq n$ in $Q''$, i.e. we have $Q_1(-,\ell)\geq 2$. But now Proposition \ref{pro:simpleregular} shows that $S_\ell$ and thus $E=\Sigma_{i_1}\cdots\Sigma_{i_{r-2}}\Sigma_n S_\ell$ is not  preinjective.	
	
  %As $n$ was a source of $Q$, it follows that $n$ is a sink of $\mu_n\mu_{\ell-1}\cdots\mu_{1} Q$.
  %In turn $\ell$ is neither sink nor source and thus $S_{\ell}$ is not preinjective because we assumed that $Q$ is not of Dynkin type.
  %\sayT{(why) is this true? Make this precise}
  %\sayD{Here is an argument, is there a shorter one? The dimension vector of a preinjective representation is a positive root of the form $c^rs_n\cdots s_{i+1}\alpha_i$ where $c=s_n\cdots s_1$ is a Coxeter element.  If $Q$ is not of finite representation type, then a Coxeter element is not of finite order and thus $\alpha_\ell$ cannot be the dimension vector of a preinjective representation.}
  %Thus $E$ is also not preinjective.
%  For sake of contradiction, suppose $E$ is preinjective, say $E\cong \tau^r\Sigma_n\cdots\Sigma_{k+1} S_k$ for some $k\in Q_0$ and some~$r\ge0$.
\end{proof}

\begin{lemma}
  Suppose $\ell_1,\ell_2\in Q_0(n,-)$ are limb vertices with respect to the source vertex $n$.
  Then any indecomposable extension $E\in\Ext(S_n,P_{\ell_1,0}\oplus P_{\ell_2,0})$ is not preinjective.
\end{lemma}
\begin{proof}
Alternative proof (maybe I miss something): First note that $n$ is not a neighbor of any other vertex than $\ell_i$ in $\mathrm{supp}P_{\ell_i,0}$. As $\ell_i$ is a limb vertex, reflecting at all vertices in the type $A$ component except $\ell_i$, we can assume that $P_{\ell_i}=S_{\ell_i}$. This means that all arrows in the type $A$ component with leaf $\ell_i$ point towards $\ell_i$. As $\ell_i$ is a limb vertex and $n$ a source, $\ell_i$ is a sink.  

%Then we have $\Sc{\udim S_{l_i}}{\udim S_{l_j}}=0$ for $i\neq j$. Moreover, we have
%$$\Hom(S_{\ell_i},S_n)=0=\Hom(S_n,P_{\ell_i})=\Ext(P_{\ell_i}, S_n)=0.$$

We have $\dim\Ext(S_n,P_{\ell_i})=|Q_0(n,\ell_i)|=1$ as there is precisely one arrow from $n$ to $\ell_i$. Thus $n+\ell_1+\ell_2$ is an exceptional root. As $E$ is indecomposable, it already follows that 
$$S_n=\Sigma_{\ell_1}\Sigma_{\ell_2} E\in\rep(Q')$$
where $Q'=\mu_{\ell_1}\mu_{\ell_2} Q$. In $Q'$ we have $\{\ell_1,\ell_2\}\subset Q_1(-,n)$ and thus $|Q_1(-,n)|\geq 2$. In particular, $S_n$ is not a preinjective representation of $Q'$ by Proposition \ref{pro:simpleregular} which means that $E$ is not a preinjective representation of $Q$.

--------------------------------------------

  First note that $Q$ is not of finite representation type and so there exists $\ell\in Q_0(n,-)$ with $\ell\ne\ell_1,\ell_2$.
  If $\ell$ is a limb with respect to vertex $n$ and $Q_0(n,-)=\{\ell,\ell_1,\ell_2\}$, then an easy calculation shows that $E\cong\tau^{-r}P_{\ell',0}$ is preprojective, here $\ell'$ is the leaf vertex of $Q$ in the subquiver $Q(\ell)\subset Q$ and $r$ is the number of arrows in $Q(\ell)$ directed towards $\ell'$.
  Since $Q$ is not of finite representation type, this implies $E$ is not preinjective.
  Thus we may assume that $Q_0(n,-)$ contains a vertex which is not a leaf or that $|Q_0(n,-)|\ge4$.
  
  There exists a commutative diagram
  \[\xymatrix{
      & & 0 \ar[d] & 0 \ar[d] & \\
      & & P_{\ell_1,0} \ar@{=}[r]\ar[d] & P_{\ell_1,0} \ar[d] & \\
      0 \ar[r] & P_{\ell_2,0} \ar[r]\ar@{=}[d] & E \ar[r]\ar[d]\ar@{}[dr]|(.7){\lrcorner} & I_{\ell'_1,0} \ar[r]\ar[d] & 0\\
      0 \ar[r] & P_{\ell_2,0} \ar[r] & I_{\ell'_2,0} \ar[r]\ar[d] & S_n \ar[r]\ar[d] & 0\\
      & & 0 & 0 & 
    }\]
  Since the upper horizontal morphism (or the left hand vertical morphism) is an equality, the lower right square is a pullback and thus there exists an exact sequence
  \[\ses{E}{I_{\ell'_1,0}\oplus I_{\ell'_2,0}}{S_n}.\]
  This sequence is not split since $E\not\cong I_{\ell'_1,0},I_{\ell'_2,0}$ and so $E$ is not injective.

  Why is $E$ not strictly preinjective?  Does applying $\tau^{-1}$ to $E$ help?  
  This would produce surjective maps $P_{\ell_i,1}\onto \tau^{-1} E$, but I don't know how that helps.
\end{proof}


\begin{lemma}
  \label{le:nonpreinjective extensions}
  If $M\in\rep Q$ is indecomposable and not preinjective, then any extension $E\in\Ext(M,P_{i,m})$ of $M$ by a preprojective representation $P_{i,m}$ contains no preinjective summand.
\end{lemma}
\begin{proof}
  For sake of contradiction, assume $E$ contains a preinjective summand.

  By assumption, the quiver $Q$ is not of finite representation type so that $P_{i,m}$ is not preinjective for any $i\in Q_0$ and $m\ge0$.
  Therefore, after an APR-tilt, we may assume $E$ actually contains a simple injective summand, say $E\cong E'\oplus S_k$ for a source $k\in Q_0$.
  Since $M$ is indecomposable and not injective, this $S_k$ component must lie in the kernel of the projection $E\onto M$.
  That is, we have $S_k\subset P_{i,m}$ which implies $S_k$ is a summand of $P_{i,m}$, a contradiction.
\end{proof}

In the definitions that follow we will need to identify particularly nice subsets of $\Hom_Q(P_{i,m},P_{j,m})$ and of $\Hom_Q(P_{i,m},P_{j,m+1})$.
To do this efficiently, we introduce the universal covering quiver $\widetilde{Q}$ of $Q$.
\begin{definition}
  Let $W=W(Q)$ denote the free group generated by the arrows $\alpha\in Q_1$ of $Q$.
  The \emph{universal covering quiver} of $Q$ is the quiver $\widetilde{Q}$ with vertices $(i,w)$ for $i\in Q_0$ and $w\in W$, and arrows $(\alpha,w):\big(s(\alpha),w\big)\to \big(t(\alpha),w \alpha \big)$ for $\alpha\in Q_1$ and $w\in W$.
\end{definition}
There is a natural forgetful functor $F:\rep\widetilde{Q}\to\rep Q$ and we say $M\in\rep Q$ \emph{lifts to $\widetilde{Q}$} if there exists $\widetilde{M}\in\rep\widetilde{Q}$ with $F(\widetilde{M})=M$.
Below, when we assume that a representation $M$ lifts to $\widetilde{Q}$, we will also implicitly assume that such a lift $\widetilde{M}$ has been chosen.
This forgetful functor intertwines the action of the reflection functor $\Sigma_k:\rep Q\to\rep Q$ for $k$ a sink (resp.~source) vertex with the action of the composite reflection functor $\widetilde{\Sigma}_k:\rep\widetilde{Q}\to\rep\widetilde{Q}$ given by applying all reflections $\Sigma_{(k,w)}$, $w\in W$, associated to sinks (resp.~sources) of $\widetilde{Q}$ covering vertex $k$ of $Q$.
Analogously, the Auslander-Reiten translation $\tilde\tau:\rep\widetilde{Q}\to\rep\widetilde{Q}$ is naturally isomorphic to the product $\widetilde{\Sigma}_n\cdots\widetilde{\Sigma}_1$ and thus the actions of $\tau$ and $\tilde{\tau}$ are also intertwined by $F$. 

Each projective representation $P_i$, $i\in Q_0$, admits a canonical lift $\widetilde{P}_i$ which is symmetric with respect to permutations about the central vertex $(i,e)$; this can be obtained using the intertwined reflection functors of $\widetilde{Q}$ as described above.
By applying Auslander-Reiten translations on the universal covering quiver, we obtain the same claim for each preprojective representation $P_{i,m}$ for $i\in Q_0$ and $m\ge0$.

There is a natural action of $W$ on $\widetilde{Q}$ given on vertices by $w.(i,w')=(i,ww')$ and on arrows by $w.(\alpha,w')=(\alpha,ww')$.
This induces an action of $W$ on $\rep\widetilde{Q}$ which we denote by $\widetilde{M}\mapsto w.\widetilde{M}$ for a representation $\widetilde{M}\in\rep\widetilde{Q}$ and $w\in W$.
\begin{lemma}
  \label{le:hom lifts}
  Given representations $M,N\in\rep Q$ which lift to $\widetilde{Q}$, there is an isomorphism 
  \[\Hom_Q(M,N)\cong\bigoplus\limits_{w\in W} \Hom_{\widetilde{Q}}(w.\widetilde{M},\widetilde{N}).\]
\end{lemma}
\begin{lemma}
  For $i,j\in Q_0$ and $m\ge0$, the spaces 
  \[\Hom_{\widetilde{Q}}(\alpha.\widetilde{P}_{j,m},\widetilde{P}_{i,m})\qquad\text{and}\qquad\Hom_{\widetilde{Q}}(\alpha^{-1}.\widetilde{P}_{i,m},\widetilde{P}_{j,m+1})\]
  for $\alpha\in Q_1(i,j)$ are both one-dimensional.
\end{lemma}
\begin{proof}
  For $m=0$ and $j$ a sink in $Q$, both claims are immediate from the definitions via reflection functors.
  For $m>0$ and $j$ arbitrary, these follow from the first case inductively by applying reflection functors.
  \sayD{This requires the cokernels to not be preinjective, why does that have to be true?}\sayT{If the cokernel is injective, injective maps become surjective, see $\tilde D_4$ in subspace orientation}
	\end{proof}
\begin{definition}
  For $i,j\in Q_0$ and $m\ge0$, write $\Irr(P_{j,m},P_{i,m})\subset\Hom_Q(P_{j,m},P_{i,m})$ for the subspace corresponding to $\bigoplus_{\alpha\in Q_1(i,j)} \Hom_{\widetilde{Q}}(\alpha.\widetilde{P}_{j,m},\widetilde{P}_{i,m})$ in Lemma~\ref{le:hom lifts}.
  Similarly, let $\Irr(P_{i,m},P_{j,m+1})\subset\Hom_Q(P_{i,m},P_{j,m+1})$ denote the subspace corresponding to $\bigoplus_{\alpha\in Q_1(i,j)} \Hom_{\widetilde{Q}}(\alpha^{-1}.\widetilde{P}_{i,m},\widetilde{P}_{j,m+1})$ in Lemma~\ref{le:hom lifts}.
\end{definition}
\begin{corollary}
  For $i,j\in Q_0$ and $m\ge0$, the morphism spaces $\Irr(P_{j,m},P_{i,m})$ and $\Irr(P_{i,m},P_{j,m+1})$ are both $b_{ij}$-dimensional.
\end{corollary}
\begin{remark}
  The notation $\Irr(P_{j,m},P_{i,m})$ and $\Irr(P_{i,m},P_{j,m+1})$ is justified by the following observation: these spaces map isomorphically to the appropriate spaces $\Rad(M,N)/\Rad^2(M,N)$ of irreducible morphisms labeling arrows in the Auslander-Reiten quiver of $Q$.
  For the results that follow, it will be important to have honest morphisms and we fix these particular choices of lifts from such quotient spaces.
\end{remark}

For $k\in Q_0$, we take $P_{k,-1}=0$.
\begin{definition}
  For $i\in Q_0$ and $m\in\ZZ_{\ge-1}$, consider the tuple of representations 
  \[\cP_{i,m}=\big(P_{k,m}\big)_{k\in Q_0(-,i)} \smile\big(P_{j,m+1}\big)_{j\in Q_0(i,-)}\]
  and the tuple of vector spaces 
  \[\cI_{i,m}=\big(\Irr(P_{k,m},P_{i,m+1})\big)_{k\in Q_0(-,i)} \smile\big(\Irr(P_{j,m+1},P_{i,m+1})\big)_{j\in Q_0(i,-)}.\]
  We write $\cP_{i,m}\bullet\cI_{i,m}$ for the representation
  \[\bigoplus_{k\in Q_0(-,i)} P_{k,m}\otimes \Irr(P_{k,m},P_{i,m+1}) \oplus \bigoplus_{j\in Q_0(i,-)} P_{j,m+1}\otimes\Irr(P_{j,m+1},P_{i,m+1}).\]
  Call a tuple of subspaces $\cV\subsetneq\cI_{i,m}$ \emph{admissible} if $\cP_{i,m}\bullet(\cI_{i,m}/\cV)$ is not isomorphic to $P_{\ell,m}$ nor to $P_{\ell,m+1}$ for any vertex $\ell\in Q_0$ which is a limb with respect to $i$.
\end{definition}

\begin{lemma}
  \label{le:standard AR sequences}
  \mbox{}
  \begin{enumerate}
    \item For $i\in Q_0$, there exists an exact sequence
      \begin{equation}
        \label{eq:projective AR}
        \ses{\cP_{i,-1}\bullet\cI_{i,-1}}{P_{i,0}}{S_i}.
      \end{equation}
    \item For $i\in Q_0$ and $m\in\ZZ_{\ge0}$, there exists an Auslander-Reiten sequence
      \begin{equation}
        \label{eq:preprojective AR}
        \ses{P_{i,m}}{\cP_{i,m}\bullet\cI_{i,m}}{P_{i,m+1}}.
      \end{equation}
      %\begin{equation}
      %  \label{eq:preprojective AR}
      %  \ses{P_{i,m}}{\bigoplus\limits_{k\in Q_1(-,i)} P_{k,m}\otimes\Irr(P_{k,m},P_{i,m+1})\oplus\bigoplus\limits_{j\in Q_1(i,-)} P_{j,m+1}\otimes \Irr(P_{j,m+1},P_{i,m+1})}{P_{i,m+1}};
      %\end{equation}
  \end{enumerate}
\end{lemma}
\begin{proof}
  Part (2) can be reduced to the case $m=0$ by applying Auslander-Reiten translations and then to the case where vertex $i$ is a sink in $Q$ using reflection functors.
  Thus it suffices to establish the existence of an Auslander-Reiten sequence 
  \[\ses{P_{i,0}}{\bigoplus\limits_{k\in Q_0(-,i)} P_{k,0}\otimes\Irr(P_{k,0},P_{i,1})}{P_{i,1}}\]
  when $i\in Q_0$ is a sink.
  This sequence and the one from part (1) follow immediately from the definitions $P_{k,0}:=\Sigma_1\cdots\Sigma_{k-1} S_k$ and $P_{i,1}:=\Sigma_1\cdots\Sigma_n P_{i,0}$.
\end{proof}

The sequences \eqref{eq:projective AR} and \eqref{eq:preprojective AR} are essential for the results to follow.
In some cases, the left hand term $P_{i,m}$ in \eqref{eq:preprojective AR} surjects onto one of the summands of the middle term and the sequence \eqref{eq:preprojective AR} will not serve our purposes.
The next results describe situations when this will occur and provide the correct alternative sequences to consider.
\begin{corollary}
  \label{cor:thin sequences}
  Suppose $i\in Q_0$ is adjacent to a limb vertex $\ell$ with associated leaf $\ell'$.
  \begin{enumerate}
    \item If $\ell\in Q_0(i,-)$, then for any $m\ge h(\ell)$, we have an exact sequence
      \begin{equation}
        \label{eq:thin limb sequence 1}
        \ses{P_{\ell',m-h(\ell)}}{P_{i,m}}{P_{\ell,m+1}}.
      \end{equation}
    \item If $\ell\in Q_0(-,i)$, then for any $m\ge h(\ell)$, we have an exact sequence
      \begin{equation}
        \label{eq:thin source sequence}
        \ses{P_{\ell',m-h(\ell)}}{P_{i,m}}{P_{\ell,m}}.
      \end{equation}
  \end{enumerate}
\end{corollary}
\begin{proof}
  We work by simultaneous induction on $h(\ell)$ and $m$.
  When $h(\ell)=0$, the type A quiver $Q(\ell)$ is linearly oriented from vertex $i$ to the leaf $\ell'$, in particular $\ell\in Q_0(i,-)$ and $\ell'$ is a sink of $Q$.

  If $\ell'=\ell$, then Lemma~\ref{le:standard AR sequences} gives the desired exact sequence
  \[\ses{P_{\ell,m}}{P_{i,m}}{P_{\ell,m+1}}\]
  for $m\ge0$.
  When $\ell'\ne\ell$, there exists $\ell''\in Q_0(\ell,-)$ which is itself a limb vertex with respect to $\ell$ and by induction we obtain an exact sequence
  \[\ses{P_{\ell',m}}{P_{\ell,m}}{P_{\ell'',m+1}}\]
  for $m\ge0$.
  Lemma~\ref{le:standard AR sequences} gives us an exact sequence
  \[\ses{P_{\ell,m}}{P_{i,m}\oplus P_{\ell'',m+1}}{P_{\ell,m+1}}\]
  which reduces to the desired sequence using Lemma~\ref{le:sequence reducing}.

  Now assume $h(\ell)>0$.
  If $\ell'=\ell$, then $\ell\in Q_0(-,i)$ is a source of $Q$.
  In this case, Lemma~\ref{le:standard AR sequences} gives the desired exact sequence
  \[\ses{P_{\ell,m}}{P_{i,m+1}}{P_{\ell,m+1}}\]
  for $m\ge0$.
  
  Assume $\ell'\ne\ell$.
  Write $\ell''$ for the limb vertex with respect to $\ell$.
  Then there are several cases to consider.
  \begin{itemize}
    \item If $\ell\in Q_0(i,-)$ and $\ell''\in Q_0(\ell,-)$, then $h(\ell'')=h(\ell)$ and by induction we obtain an exact sequence
      \[\ses{P_{\ell',m-h(\ell)}}{P_{\ell,m}}{P_{\ell'',m+1}}\]
      for $m\ge h(\ell)$.
      Lemma~\ref{le:standard AR sequences} gives an exact sequence
      \[\ses{P_{\ell,m}}{P_{i,m}\oplus P_{\ell'',m+1}}{P_{\ell,m+1}}\]
      which reduces to the desired sequence using Lemma~\ref{le:sequence reducing}.
    \item If $\ell\in Q_0(i,-)$ and $\ell''\in Q_0(-,\ell)$, then $h(\ell'')=h(\ell)$ and by induction we obtain an exact sequence
      \[\ses{P_{\ell',m-h(\ell)}}{P_{\ell,m}}{P_{\ell'',m}}\]
      for $m\ge h(\ell)$.
      Lemma~\ref{le:standard AR sequences} gives an exact sequence
      \[\ses{P_{\ell,m}}{P_{i,m}\oplus P_{\ell'',m}}{P_{\ell,m+1}}\]
      which reduces to the desired sequence using Lemma~\ref{le:sequence reducing}.
    \item If $\ell\in Q_0(-,i)$ and $\ell''\in Q_0(\ell,-)$, then $h(\ell'')=h(\ell)-1$ and by induction we obtain an exact sequence
      \[\ses{P_{\ell',m-h(\ell)}}{P_{\ell,m-1}}{P_{\ell'',m}}\]
      for $m-1\ge h(\ell)-1$.
      Lemma~\ref{le:standard AR sequences} gives an exact sequence
      \[\ses{P_{\ell,m-1}}{P_{i,m}\oplus P_{\ell'',m}}{P_{\ell,m}}\]
      which reduces to the desired sequence using Lemma~\ref{le:sequence reducing}.
    \item If $\ell\in Q_0(-,i)$ and $\ell''\in Q_0(-,\ell)$, then $h(\ell'')=h(\ell)-1$ and by induction we obtain an exact sequence
      \[\ses{P_{\ell',m-h(\ell)}}{P_{\ell,m-1}}{P_{\ell'',m-1}}\]
      for $m-1\ge h(\ell)-1$.
      Lemma~\ref{le:standard AR sequences} gives an exact sequence
      \[\ses{P_{\ell,m-1}}{P_{i,m}\oplus P_{\ell'',m-1}}{P_{\ell,m}}\]
      which reduces to the desired sequence using Lemma~\ref{le:sequence reducing}.
  \end{itemize}
\end{proof}

For $i\in Q_0$ and $m\ge0$, given $\ell\in Q_0(-,i)\cup Q_0(i,-)$ we write $\cI_{i,m}^{\{\ell\}}\subset\cI_{i,m}$ for the tuple of subspaces with the only proper subspace at position $\ell$ and the zero subspace in that component.
\begin{corollary}\sayT{Is there a typo and $P_{i,m+1}$ maybe $P_{\ell,m+1}$? We have $|\cI_{i,m}^{\{\ell\}}|=1$ right?}
  \label{cor:reduced sequences}
  Suppose $i\in Q_0$ is adjacent to a limb vertex $\ell$ with associated leaf $\ell'$.
  Then there is an exact sequence
  \begin{equation}
    \label{eq:reduced AR}
    \ses{P_{\ell',m-h(\ell)}}{\cP_{i,m}\bullet\cI_{i,m}^{\{\ell\}}}{P_{i,m+1}}.
  \end{equation}
\end{corollary}
\begin{proof}
  This is immediate by applying Lemma~\ref{le:sequence reducing} to the sequences from Corollary~\ref{cor:thin sequences} and Lemma~\ref{le:standard AR sequences}.
\end{proof}

The following result is crucial to the constructions to follow.
\begin{lemma}
  Consider $i\in Q_0$ and $m\ge0$.
  \begin{enumerate}
    \item For any tuple of subspaces $\cV\subsetneq\cI_{i,-1}$, the natural evaluation map $\cP_{i,-1}\bullet\cV\to P_{i,0}$ is injective.\sayT{$\cV$ admissible in (1)?}
    \item For any admissible tuple $\cV\subsetneq\cI_{i,m}$, the natural evaluation map $\cP_{i,m}\bullet\cV\to P_{i,m+1}$ is injective.
    \item The sequence \eqref{eq:preprojective AR} induces isomorphisms 
      \[\Irr(P_{k,m},P_{i,m+1})^*\cong \Irr(P_{i,m},P_{k,m}) \qquad \text{and} \qquad \Irr(P_{j,m+1},P_{i,m+1})^*\cong \Irr(P_{i,m},P_{j,m+1})\]
      for each $k\in Q_0(-,i)$ and each $j\in Q_0(i,-)$.
  \end{enumerate}
\end{lemma}
\begin{proof}
  Part (1) immediately follows from the exactness of the sequence \eqref{eq:projective AR}.
  Thus for an admissible tuple~$\cV\subsetneq\cI_{i,-1}$, say with $V_j\subset\Irr(P_{j,0},P_{i,0})$ in the $j$-th component, we have an exact commutative diagram:
  \[\xymatrix{
      & 0 \ar[d] & 0 \ar[d] & & \\
      0 \ar[r] & \bigoplus\limits_{j\in Q_0(i,-)} P_{j,0}\otimes V_j \ar@{=}[r] \ar[d] & \bigoplus\limits_{j\in Q_0(i,-)} P_{j,0}\otimes V_j \ar[r]\ar[d] & 0 \ar[d] & \\
      0 \ar[r] & \bigoplus\limits_{j\in Q_0(i,-)} P_{j,0}\otimes\Irr(P_{j,0},P_{i,0}) \ar[r]\ar[d] & P_{i,0} \ar[r]\ar[d] & S_i \ar[r]\ar@{=}[d] & 0 \\
      0 \ar[r] & \bigoplus\limits_{j\in Q_0(i,-)} P_{j,0}\otimes\big(\Irr(P_{j,0},P_{i,0})/V_j\big) \ar[r]\ar[d] & C \ar[r]\ar[d] & S_i \ar[r]\ar[d] & 0 \\
      & 0 & 0 & 0 & 
    }
  \]
  
  Part (2) is trivial if $\cV$ has all components zero, so we assume this is not the case.
  First observe that after an APR-tilt (applying a sequence of reflection functors) it suffices to show in the case $m=-1$ with vertex $i=n$ the unique source of $Q$ that the representation $C$ in the diagram above contains no preinjective summands.

  There are two cases to consider.
  \begin{itemize}
    \item Assume $\bigoplus\limits_{j\in Q_0(i,-)} P_{j,0}\otimes\big(\Irr(P_{j,0},P_{i,0})/V_j\big)$ contains a summand $P_{\ell,0}$ for $\ell\in Q_0(n,-)$ which is not a limb with reprect to the source vertex $n$.
      Then the result follows by induction using Lemma~\ref{le:nonlimb extensions} and Lemma~\ref{le:nonpreinjective extensions}.
    \item Assume all summands of $\bigoplus\limits_{j\in Q_0(i,-)} P_{j,0}\otimes\big(\Irr(P_{j,0},P_{i,0})/V_j\big)$ correspond to vertices which are limbs with respect to $n$, say $V_j=0$ exactly for $j\in\{i_1,\ldots,i_q\}$.
      Since $\cV$ is admissible, we must have $q\ge2$.
  \end{itemize}

  A simple homological calculation shows that $\End(C)\cong\End(P_{i,0})$ is one-dimensional.
  Therefore $C$ must be indecomposable and hence preinjective.
  Thus there exists $k\in Q_0$ with $C\cong \tau^r\Sigma_n\cdots \Sigma_{k+1} S_k$ for some $r\ge0$.

  Note that the lower horizontal sequence in the diagram above shows that $C$ contains a one-dimensional space at vertex $n$.
  As $C$ has a one-dimensional space at vertex $n$, this implies there is a unique path in $Q$ from vertex $k$ to vertex $n$.
  Since $\cV\subsetneq\cI_{n,-1}$ is a proper subspace, $C$ contains a projective subrepresentation and so vertex $k$ must be a sink in $Q$.
  
  
  are two cases to consider.
  We need to verify that $\cV$ being admissible implies that $P_{i,0}/\cP_{i,-1}\bullet\cV$ contains no preinjective summands.
  This then implies the statement of part (2) by applying reflection functors.

  By Lemma~\ref{le:preprojective ext groups}, we have $\Ext(\cP_{i,m}\bullet\cV,P_{i,m})=0$.
  Thus, by taking pullbacks, we obtain a commutative diagram 
  \[\xymatrix{0 \ar[r] & P_{i,m}\ar@{=}[d]\ar[r] & P_{i,m}\oplus \cP_{i,m}\bullet\cV \ar@{}[dr]|(.7){\lrcorner} \ar[r]\ar[d]_{(\iota_{i,m},\mathrm{id}_{\cP_{i,m}}\bullet\iota_\cV)} & \cP_{i,m}\bullet\cV \ar[r]\ar_{ev_\cV}[d] & 0\\
  0 \ar[r] & P_{i,m}\ar[r]^{\iota_{i,m}} & \cP_{i,m}\bullet \cI_{i,m} \ar[r]^{ev} & P_{i,m+1} \ar[r] & 0}\]
  in which we set $P_0=0$ and $I_l=0$ for $l\leq 0$.
  Since $\mathrm{id}_{P_1}\otimes \iota_V$ is injective, the snake lemma shows that $\ker(ev_V)=0$ for $m=1$ and thus $ev_V$ is injective in this case.
  
  In view of Remark~\ref{rem:reflection recursion}, it is enough for the case $m>1$ to show that the cokernel of $ev_V$ does not have a preinjective direct summand when $m=1$.
  Clearly the cokernel of $\mathrm{id}_{P_1}\otimes \iota_V$ is isomorphic to $P_1\otimes \cH_1/V$.
  Thus we obtain the following commutative diagram induced by the cokernels of the above vertical maps, note that the vertical maps below are surjective:
  \[\xymatrix{0 \ar[r]  & P_1\otimes \cH_1 \ar[d]\ar[r]^{ev} &P_{2}\ar[d] \ar[r] & I_{1}\ar@{=}[d]\ar[r]&0\\ 0\ar[r] & P_1\otimes \cH_1/V \ar[r] &K\ar[r]& I_{1}\ar[r]&0.}\]

  We need to show that $K$ has no preinjective direct summand.
  As $\Hom(P_2,P_1)=0$ and as the vertical maps are surjective, the representation $K$ has no direct summand which is isomorphic to $P_1=S_1$.
  But this already shows that $K$ is indecomposable as $\udim K=(\dim \cH_1/V,1)$.
  Since $V$ is a proper subspace of $\cH_1$, $\udim K$ is not the dimension vector of a preinjective representation and the claim follows.
\end{proof}

\begin{definition}
  For $i\in Q_0$, $m\in\ZZ_{\ge0}$, and $\cV\subsetneq\cI_{i,m}$, define the \emph{truncated preprojective} representation $P_{i,m+1}^\cV=P_{i,m+1}/\cP_{i,m}\bullet\cV$.
\end{definition}


\begin{proposition}
  Consider $\cV\subsetneq\cI_{i,m}$ with $\codim_{\cI_{i,m}}(\cV)=1$, say $\cP_{i,m}\bullet\cI_{i,m}/\cP_{i,m}\bullet\cV\cong P_{\ell,n}$ with $n=m$ or $n=m+1$.
  Then $P_{i,m+1}^\cV\cong P_{\ell,n}^{\overline{\cV}}$, where $\overline{\cV}=..$.
\end{proposition}
\begin{proof}
  There is a commutative diagram of the following form:
  \[\xymatrix{
      & & 0 \ar[d] & 0 \ar[d] \\
      & & \cP_{i,m}\bullet\cV \ar[d]\ar@{=}[r] & \cP_{i,m}\bullet\cV \ar[d] \\
      0 \ar[r] & P_{i,m} \ar[r]\ar@{=}[d] & \cP_{i,m}\bullet\cI_{i,m}  \ar[d]\ar[r] & P_{i,m+1} \ar[r]\ar[d] & 0\\
      0 \ar[r] & P_{i,m} \ar[r] & P_{\ell,n} \ar[r]\ar[d] & P_{i,m+1}^\cV \ar[r]\ar[d] & 0 \\
      & & 0 & 0}\]
  It follows that we may identify $P_{i,m+1}^\cV$ with a truncated preprojective $P_{\ell,n}^{\overline{\cV}}$.
\end{proof}


\section{Appendix: Homological Lemmas}

\begin{lemma}
  \label{le:sequence reducing}
  Suppose there is a exact sequence 
  \[\ses{A}{X\oplus B}{C}\]
  for which the induced map $f:A\to B$ given by composing the left hand morphism with the natural projection is surjective.
  Then there is an induced short exact sequence 
  \[\ses{K}{X}{C},\]
  where $K=\ker(f)$.
\end{lemma}
\begin{proof}
  Consider the following exact commutative diagram:
  \[\xymatrix{
      & 0 \ar[d] & 0 \ar[d] & \\
      0 \ar[r] & K \ar[d]\ar[r]\ar@{}[dr]|(.3){\ulcorner} & X \ar[d] & & \\
      0 \ar[r] & A \ar[d]\ar[r] & X\oplus B \ar[d]\ar[r] & C \ar[r] & 0\\
      & B \ar@{=}[r]\ar[d] & B \ar[d] & \\
      & 0 & 0 & }\]
  The middle vertical sequence is naturally split which induces the upper horizontal morphism and shows that it is injective.
  Since the lower horizontal morphism is an equality, the upper left square is a pushout.
  It follows that the cokernel of the upper horizontal morphism coincides with the cokernel in the middle horizontal sequence, giving the claim.
\end{proof}


%%%%%%%%%%%%%%%%%%%%%%%%%%%%%%%%%%%%%%%%%%%%%%%%%%%

\section{$\CC^*$-equivariant Caldero-Chapoton Maps}
The ideas presented here assume that we have already constructed a cell decomposition by combining the Bia\l{}ynicki-Birula cells coming from $\CC^*$-actions with the inductive construction of cell decompositions for lifts to the universal covering quiver.




%%%%%%%%%%%%%%%%%%%%%%%%%%%

\begin{thebibliography}{10}

\bibitem{ass} 
  I.~Assem, D.~Simson, A.~Skowronski: Elements of the Representation Theory of Associative Algebras. Cambridge University Press, Cambridge 2007.

\bibitem{ars} 
  M.~Auslander, I.~Reiten, S.~O.~Smalo: Representation theory of Artin algebras {\bf 36}. Cambridge University Press, Cambridge 1997.

\bibitem{bb} 
  A.~Bia\l{}ynicki-Birula: Some theorems on actions of algebraic groups. Annals of Mathematics \textbf{98}, 480-497 (1973).

\bibitem{bgp} 
  I.~N.~Bernstein, I.~M.~Gelfand, V.~A.~Ponomarev: Coxeter functors, and Gabriel's theorem. Russian Mathematical Surveys \textbf{28}(2), 17-32 (1973).

\bibitem{brenner-butler} 
  S.~Brenner, M.~C.~R.~Butler: The equivalence of certain functors occurring in the representation theory of artin algebras and species. Journal of the London Mathematical Society \textbf{2}(1), 183-187 (1976).

\bibitem{cc} 
  P.~Caldero, F.~Chapoton: Cluster algebras as {H}all algebras of quiver representations. Commentarii Mathematici Helvetici \textbf{81}(3), 595-616 (2006).

\bibitem{ck} 
  P.~Caldero, B.~Keller: From triangulated categories to cluster algebras II.  Annales Scientifique de l'Ecole Normale Superi\'{e}ure (4) \textbf{39} (6), 983-1009 (2006).

\bibitem{ca}	J.B.~Carrell: Torus actions and cohomology. Algebraic quotients. Torus actions and cohomology. The adjoint representation and the adjoint action, 83-158. Encyclopaedia Mathematical Science \textbf{131}, Invariant Theory and Algebraic Transformation Groups, Springer, Berlin, 2002.

\bibitem{ce} 
  G.~Cerulli Irelli, F.~Esposito: Geometry of quiver Grassmannians of Kronecker type and applications to cluster algebras. Algebra \&  Number Theory \textbf{5}(6), 777-801 (2011).

\bibitem{cefr} 
  G.~Cerulli Irelli, F.~Esposito, H.~Franzen, M.~Reineke: Topology of Quiver Grassmannians. Preprint 2018.

\bibitem{cr} 
  P.~Caldero, M.~Reineke: On the quiver Grassmannian in the acyclic case. Journal of Pure and Applied Algebra \textbf{212}(11), 2369-2380 (2008).

\bibitem{CB}
William Crawley-Boevey. Lectures on representations of quivers. Unpublished lecture notes, online available at \url{http://www1.maths.leeds.ac.uk/~pmtwc/quivlecs.pdf}, 1992.
	

\bibitem{gab} 
  P.~Gabriel: The universal cover of a finite-dimensional algebra. Representations of algebras. Lecture Notes in Mathematics {\bf 903}, 68-105 (1981).

\bibitem{llz} 
  K.~Lee, L.~Li, A.~Zelevinsky: Greedy elements in rank 2 cluster algebras. Selecta Mathematica \textbf{20}(1), 57-82 (2014).

\bibitem{lw} 
  O.~Lorscheid, T.~Weist: Representation type via Euler characteristics and singularities of quiver Grassmannians. Preprint 2017, arXiv:1706.00860.

\bibitem{rin1} 
  C.~M.~Ringel. Exceptional modules are tree modules. Linear algebra and its Applications \textbf{275/276}, 471-493 (1998).

\bibitem{rupel} 
  D.~Rupel: Rank Two Non-Commutative Laurent Phenomenon and Pseudo-Positivity. Preprint 2017, arXiv:1707.02696.

\bibitem{sch} 
  A.~Schofield: General representations of quivers. Proceedings of the London Mathematical Society (3) \textbf{65}(1), 46-64 (1992).

\bibitem{wei} 
  T.~Weist: Localization of quiver moduli spaces. Representation Theory \textbf{17}(13), 382-425 (2013).

\end{thebibliography}

\end{document}
